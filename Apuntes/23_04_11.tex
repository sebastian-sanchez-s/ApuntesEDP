%! Tex Root = edp.tex
\documentclass[../edp.tex]{subfiles}

\begin{document}

{\scshape \hfill 11 de abril, 2023}

\subsection{Principio de Dirichlet o Método de energía}

Buscamos resolver el problema de Poisson
\begin{displaymath}\label{poisson:110423}
\begin{cases}
\begin{aligned}
	-\Delta u &= f &&, \text{ en } \Omega &&, f\in \CC(\Omega)\\
	u &= g &&, \text{ en } \partial\Omega &&, g\in \CC(\partial\Omega)
\end{aligned}
\end{cases}.
\tag{\(\star\)}
\end{displaymath}

\begin{Teorema}
	Si \(\Omega\) es acotado y \(\partial\Omega\in \CC^{1}\), el
	problema tiene a lo más una solución de clase
	\(\CC^{2}(\Omega)\cap \CC(\overline{\Omega})\).
\end{Teorema}
\begin{Demostracion}
\end{Demostracion}

\noindent\textbf{Principio de Dirichlet} Definimos el funcional de energía
\begin{displaymath}
	\II w 
	\coloneqq 
	\int_{\Omega}
		(\frac{1}{2} \abs{\nabla u}^{2} - w f)
\end{displaymath}
con dominio \(\mathcal{A} \coloneqq \left\{ w\in \CC^{2}(\Omega) \mid
w = g \text{ sobre } \partial\Omega \right\}\).

\begin{Teorema}
	Si \(u\in\CC^{2}(\Omega)\cap\CC(\overline{\Omega})\) es solución
	del problema~\eqref{poisson:110423} entonces 
	\begin{displaymath}
		\II u = \min_{w \in \mathcal{A}} \II w.
	\end{displaymath}
	Más aún, si \(u\in \mathcal{A}\) cumple la ecuación anterior, 
	entonces \(u\) es solución de~\eqref{poisson:110423}.
\end{Teorema}
\begin{Demostracion}
\end{Demostracion}

\section{Estrategias para resolver EDP's}

\subsection{La Transformada de Fourier}

\begin{Definicion}[Transformada de Fourier]
	Sea \(u\in L^{1}(\R^n)\). Definimos su transformada de Fourier,
	\(\FF(u) = u^{\wedge}\) como
	\begin{displaymath}
		u^{\wedge}(y)
		\coloneqq
		\frac{1}{\sqrt{2\pi}^n}
		\int_{\R^n}
			e^{-i x\cdot y}
			u(x)
		\, dx
	\end{displaymath}
	donde \(y\in \R^n\). Su inversa \(\FF^{-1}(u) = u^{\vee}\)
	como 
	\begin{displaymath}
		u^{\vee}(y)
		\coloneqq
		\frac{1}{\sqrt{2\pi}^n}
		\int_{\R^n}
			e^{i x\cdot y}
			u(x)
		\, dx
	\end{displaymath}
\end{Definicion}

Notar que la transformada y la transformada inversa están acotadas
(pues \(u\) es integrable). Más aún, si pedimos que \(u\) sea cuadrado
integrable, entonces las transformadas serán cuadrado integrables.
Esto se muestra en el siguiente teorema.

\begin{Teorema}[Plancherel]
	Si \(u\in L^{1}(\R^n) \cap L^{2}(\R^n)\) es integrable y cuadrado
	integrable sobre \(\R^n\), entonces sus transformada de Fourier (e
	inversa) son cuadrado integrables. Más aún,
	\begin{displaymath}
		\norm{u^{\wedge}}_{L^{2}}
		=
		\norm{u^{\vee}}_{L^{2}}
		=
		\norm{u}_{L^{2}}.
	\end{displaymath}
\end{Teorema}
\begin{Demostracion}
\end{Demostracion}

La gracia de plancerel es que podemos definir la transformada de
Fourier para funciones \(u\) que solo son cuadrado integrables. Esto lo
haremos mendiante aproximaciones por funciones \(\left\{ u_k
\right\}_k \subset L^{1} \cap L^{2}\) que convergen a \(u\). En
efecto, por Plancherel tenemos que
\begin{displaymath}
	\norm{u^{\wedge}_{k} - u^{\wedge}_{j}}
	=
	\norm{\left(u_k - u_j\right)^{\wedge}}
	=
	\norm{u_k - u_j}.
\end{displaymath} 
Así que la sucesión \(\left\{ u^{\wedge}_k \right\}_k\) es de Cauchy
en \(L^2\). Como \(L^2\) es completo, tenemos que es convergente; Y de
hecho converge a \(u^{\wedge}\).

\begin{Teorema}[Propiedades de la Transformada de Fourier]
	Si \(u, v \in L^2(\R^n)\) son cuadrado integrables, entonces
	\begin{enumerate}[itemsep=2pt,topsep=3pt]
		\item 
		\begin{displaymath}
			\int_{\R^n} u \overline{v} \, dx
			=
			\int_{\R^n} u^{\wedge} \overline{v}^{wedge} \, dx.
		\end{displaymath}

		\item Sea \(\alpha = (\alpha_1, \dots, \alpha_n)\) un
		multi-índice. Entonces 
		\begin{displaymath}
			\left(D^{\alpha} u\right)^{\wedge}
			=
			(iy)^{\alpha} u^{\wedge},
		\end{displaymath}
		donde \(y^{\alpha} = y_1^{\alpha_1} y_2^{\alpha_2} \cdots
		y_n^{\alpha_n}\) y \(D^{\alpha} u \in L^{2}\).

		\item \((u\ast v)^{\wedge} = (\sqrt{2\pi})^n\, u^{\wedge}
		v^{\wedge}\).

		\item \(u = \FF^{-1} \, \FF\, u\).
	\end{enumerate}
	
\end{Teorema}

\end{document}
