%! Tex Root = edp.tex
\documentclass[../edp.tex]{subfiles}

\begin{document}

{\scshape \hfill 30 de mayo, 2023}

\subsection{Teoremas de Extensión}

\begin{Teorema}
	Sea \(1\le p \le \infty\). Supongamos que \(\Omega\subset\R^n\) es
	un abierto acotado con frontera \(\CC^{1}\). Sea \(V\subset\R^n\)
	un conjunto abierto y acotado tal que \(\Omega\subset\subset V\).
	Entonces, existe un operador lineal acotado
	\begin{displaymath}
		E\colon W^{1,p}(\Omega) \to W^{1,p}(\Omega) 
	\end{displaymath}
	tal que \(\forall u\in W^{1,p}(\Omega)\) se cumple que
	\begin{enumerate}[itemsep=2pt,topsep=3pt,leftmargin=.2\textwidth]
		\item \(E(u) = u\) c.t.p en \(\Omega\);
		\item \(\supp E(u) \subset V\) y
		\item existe una constante \(C\) que sólo depende de
		\(p,\Omega\) y \(V\) tal que
		\begin{displaymath}
			\norm{E(u)}_{W^{1,p}(\R^n)} \le C
			\norm{u}_{W^{1,p}(\Omega)}.
		\end{displaymath}
	\end{enumerate}
	El operador \(E\) nos da una extensión de \(u\) a \(\R^n\).
\end{Teorema}

\subsection{Teoremas de Traza}

La noción de traza viene a sustituir la de restringir una función a la
frontera.

\begin{Teorema}[Traza]
	Sea \(\Omega\subset\R^n\) abierto y acotado con frontera
	\(\CC^{1}\) y \(1 \le p < \infty\). Entonces existe un operador
	lineal acotado 
	\begin{displaymath}
		T\colon W^{1,p}(\Omega) \to L^{p}(\partial\Omega)
	\end{displaymath}
	tal que
	\begin{enumerate}[itemsep=2pt,topsep=2pt,leftmargin=.2\textwidth]
		\item Si \(u\in W^{1,p}(\Omega)\cap\CC(\overline{\Omega})\)
		entonces \(T(u) = u\mid_{\partial\Omega}\);
		\item Para toda función \(u\in W^{1,p}(\Omega)\) existe una
		constante \(C = C(p, \Omega)\) tal que
		\begin{displaymath}
			\norm{T(u)}_{L^{p}(\partial\Omega)}
			\le 
			C \norm{u}_{W^{1,p}(\Omega)}.
		\end{displaymath}
	\end{enumerate}
\end{Teorema}

\begin{Teorema}[kernel de \(T\)]
	Sea \(\Omega\subset\R^n\) un abierto acotado con
	frontera \(\CC^{1}\). Si \(u\in W^{1,p}(\Omega)\), entonces
	\begin{displaymath}
		u\in W^{1,p}_{0}(\Omega)
		\iff
		T(u) = 0
		\text{ en } \partial\Omega.
	\end{displaymath}
\end{Teorema}

\subsection{Inyecciones de Sobolev}

Nos preguntamos si una función en un espacio de Sobolev
automáticamente pertenece a otro espacio con mayor regularidad.
Si analizamos el espacio \(W^{1,p}(\Omega)\) tenemos que la respuesta
es un sí y depende de los valores de \(p\).

Recordamos que \(\Omega\subset\R^n\) abierto.

\subsubsection{Caso \(1\le p < n\): Desigualdad de Gagliardo-Nirenberg-Sobolev}

Comenzamos analizando el caso \(1\le p < n\). Buscamos una desigualdad
de la forma:
\begin{displaymath}
	\norm{u}_{L^{q}(\R^n)}
	\le
	C \norm{Du}_{L^{p}(\R^n)},
\end{displaymath}
donde \(1\le q < \infty\) y \(C>0\) no dependen de \(u\).
Notése que la desigualdad implica de manera
directa que \(W^{1,p}(\R^n) \subset L^{q}(\R^n)\). En efecto,
\begin{displaymath}
	\norm{u}_{L^{q}(\R^n)}^{p}
	\le
	C^{p} \norm{Du}_{L^{p}(\R^n)}^{p}
	\le
	C^{p} \norm{Du}_{L^{p}(\R^n)}^{p}
	+
	C^{p} \norm{u}_{L^{p}(\R^n)}^{p}
	=
	C^{p} \norm{u}_{W^{1,p}(\R^n)}^{p}.
\end{displaymath}

Miremos el caso \(u \in \CC^{\infty}_{C}(\R^n)\) y veamos qué
condiciones de compatibilidad debe cumplir \(q\). Vale decir, si
suponemos la desigualdad cierta queremos saber qué debería cumplir
\(q\). Definamos 
\begin{displaymath}
	u_{\lambda}(x) 
	\coloneqq 
	u(\lambda x)
	\qquad x\in\R^{n}, \lambda > 0
\end{displaymath}
e insertemoslo en la desigualdad. Esto nos deja con
\begin{displaymath}
	\frac{1}{\lambda^{n/q}}
	\norm{u}_{L^{q}(\R^n)}
	\le
	\frac{1}{\lambda^{n/p}}
	C \norm{Du}_{L^{p}(\R^n)}.
\end{displaymath}
De esta forma,
\begin{displaymath}
	\norm{u}_{L^{q}(\R^n)}
	\le
	\lambda^{1 - n/p - n/q}
	C \norm{Du}_{L^{p}(\R^n)}.
\end{displaymath}
y por lo tanto \(1-n/p-n/q = 0\) o equivalentemente \(q = np/(n-p)\).
En particular, \(q > p\).

\begin{Definicion}[Exponente Conjugado de Sobolev]
	Sea \(1 \le p < n\). Definimos el exponente conjugado de Sobolev
	como 
	\begin{displaymath}
		p^{\ast} = \frac{np}{n-p}.
	\end{displaymath}
\end{Definicion}

Notar que 
\begin{displaymath}
	\frac{1}{p^{\ast}}
	=
	\frac{1}{p}
	-
	\frac{1}{n}
	\qquad\text{y}\qquad
	p^{\ast} > p.
\end{displaymath}

\begin{Teorema}[Desigualdad de Gagliardo-Nirenberg-Sobolev]
	Sea \(1\le p < n\). Entonces existe una constante \(C = C(n,p) > 0\), que
	sólo depende de \(p\) y \(n\) tal que
	\begin{equation}\label{GNS}
		\norm{u}_{L^{p^{\ast}}(\R^n)}
		\le
		C\norm{Du}_{L^{p}(\R^n)}
		\qquad \forall u \in \CC^{1}_{C}(\R^n).
		\tag{GNS}
	\end{equation}
\end{Teorema}

\begin{Teorema}[Estimaciones en \(W^{1,p}(\Omega),\, 1 \le p < n\)]
	Sea \(\Omega \subset\R^n\) abierto y acotado con frontera de clase
	\(\CC^1\). Sean \(1\le p < n\) y \(u\in W^{1,p}(\Omega)\). 
	Entonces \(u\in L^{p^{\ast}}(\Omega)\) y
	\begin{displaymath}
		\norm{u}_{L^{p^{\ast}}(\Omega)}
		\le
		C
		\norm{u}_{W^{1,p}(\Omega)},
	\end{displaymath}
	donde \(C = C(p,\Omega) > 0\). En particular, tenemos que
	\begin{displaymath}
		W^{1,p}(\Omega) \subset L^{p^{\ast}}(\Omega).
	\end{displaymath}
\end{Teorema}
\begin{Demostracion}
	La idea será aplicar~\eqref{GNS} y para ello necesitamos
	que la función esté definida sobre todo \(\R^n\). Ahora bien, dado
	que \(\Omega\) tiene frontera de clase \(\CC^{1}\) y es acotado, 
	el teorema de extensión nos dice que existe un operador lineal
	acotado \(E\) tal que extiende \(u\) a \(\R^n\). Denotemos a tal 
	extensión \(\bar{u} = E(u) \in W^{1,p}(\R^n)\). Por el mismo
	teorema tenemos que
	\begin{displaymath}
		\norm{\bar{u}}_{W^{1,p}(\R^n)}
		\le
		C \norm{u}_{W^{1,p}(\Omega)}.
	\end{displaymath}
	Aplicando el teorema de aproximación interior por funciones suaves
	tenemos una sucesión de funciones \((u_m)_{m\in\N} \in
	\CC^{\infty}_{C}(\R^n)\) tal que \(u_m \to \bar{u}\) en norma 
	\(W^{1,p}(\R^n)\). Aplicando~\eqref{GNS} tenemos que
	\begin{displaymath}
		\norm{u_m - u_{\ell}}_{L^{p^{\ast}}(\R^n)}
		\le
		C \norm{D(u_m - u_{\ell})}_{L^{p}(\R^n)}.
	\end{displaymath}
	Obsérvese que el lado derecho se va a cero, pues
	\((Du_{m})_{m\in\N}\) es una sucesión de Cauchy en \(L^{p}(\R^n)\).
	En consecuencia, concluimos que \((u_m)_{m\in\N}\) es una sucesión
	de Cauchy en \(L^{p^{\ast}}(\R^n)\). A su vez, esto implica que 
	\(u_{m} \to \bar{u}\) en norma \(L^{p^{\ast}}(\R^n)\), pues vive
	en un espacio completo. Luego,
	\begin{displaymath}
		\norm{u_m}_{L^{p^{\ast}}(\R^n)}
		\le
		C \norm{Du_m}_{L^{p}(\R^n)}.
	\end{displaymath}
	Dado que tenemos convergencia en ambos espacios, tomando \(m\to
	\infty\) nos deja con
	\begin{displaymath}
		\norm{\bar{u}}_{L^{p^{\ast}}(\R^n)}
		\le
		C \norm{D \bar{u}_m}_{L^{p}(\R^n)}
		\le
		C \norm{\bar{u}_m}_{W^{1,p}(\R^n)}
		\le
		C \norm{\bar{u}_m}_{W^{1,p}(\Omega)}.
	\end{displaymath}
	donde segunda desigualdad sale por la definición de la norma de
	los espacios de Sobolev y la última por el teorema de extensión.
	Por otro lado,
	\begin{displaymath}
		\norm{\bar{u}}_{L^{p^{\ast}}(\R^n)}
		\ge
		\norm{\bar{u}}_{L^{p^{\ast}}(\Omega)}
		\ge
		\norm{u}_{L^{p^{\ast}}(\Omega)}.
	\end{displaymath}
	donde la primera desigualdad es porque estamos integrando en un
	espacio más pequeño y la última por el teorema de extensión.
\end{Demostracion}

\begin{Teorema}[Estimaciones en \(W^{1,p}_{0}(\Omega),\, 1\le p < n\)]
	Sea \(\Omega \subset\R^n\) abierto y acotado. 
	Sean \(1\le p < n\) y \(u\in W^{1,p}_{0}(\Omega)\). 
	Entonces
	\begin{displaymath}
		\norm{u}_{L^{q}(\Omega)}
		\le
		C
		\norm{D u}_{L^{p}(\Omega)},
		\quad\forall q\in [1,p^{\ast}]
	\end{displaymath}
	donde \(C = C(p,q,\Omega) > 0\). En particular,
	\begin{displaymath}
		\norm{u}_{L^{p}(\Omega)}
		\le
		C
		\norm{D u}_{L^{p}(\Omega)},
		\quad\forall 1 \le p < \infty.
	\end{displaymath}
	Esta última se conoce como la \textit{desigualdad de Poincaré}.
\end{Teorema}

\end{document}
