%! Tex Root = edp.tex
\documentclass[../edp.tex]{subfiles}

\begin{document}

{\scshape \hfill 20 de abril, 2023}

\section{Ecuación del Calor}
 
Trabajamos en el problema
\begin{equation}\label{eq:calor}
\begin{caligned}
	u_{t} - \Delta_{x} u &= f && (t,x) \in \R^{+}\times \Omega
\end{caligned}
\end{equation}
donde \(\Omega\subset\R^n\) abierto y \(f\colon \R^{+}\times\Omega\to
\R\) es una función dada. Si \(f\equiv 0\) diremos que el problema es
homogéneo.

\subsection{Solución fundamental}

\textbf{Simetrías} Considerando el problema homogéneo, notamos que es
invariante bajo traslaciones y rotaciones en el espacio. Más aún, es
invariante bajo traslaciones de la forma 
\begin{displaymath}
	(t,x) \mapsto (\lambda^2 t, \lambda x)
	\qquad
	\lambda \in \R.
\end{displaymath}
Por esto último, vamos a buscar soluciones de la forma
\begin{equation}\label{separable:calor}
	u(t,x) = t^{-\alpha} v(x t^{-\beta})
	\qquad
	\alpha, \beta \in \R.
\end{equation}

Poniendo \(y = xt^{-\beta}\) y reemplazando en la ecuación
~\eqref{eq:calor} se tiene que
\begin{displaymath}
	\alpha t^{-(\alpha+1)} v(y)
	+
	\beta t^{-(\alpha+1)} y \cdot \nabla v(y)
	+
	t^{-(\alpha + 2\beta)} \Delta v(y)
	=
	0.
\end{displaymath}
Tomando \(\beta = 1/2\) nos queda
\begin{displaymath}
	\alpha v(y)
	+ 
	\frac{1}{2} y\cdot \nabla v(y)
	+
	\Delta v(y)
	= 0.
\end{displaymath}
Ahora la ecuación ``solo tiene derivadas en \(y\)", así que es
razonable (por lo discutido en la ecuación de Laplace) buscar
soluciones de la forma \(v(y) = w(\abs{y})\). De esta forma obtenemos,
\begin{displaymath}
	\alpha w 
	+ 
	\frac{1}{2} r w'
	+
	w''
	+
	\frac{n-1}{r} w
	= 0.
\end{displaymath}
Multiplicando por \(r^{n-1}\) vemos que la forma se asemeja a la regla
del producto. De hecho, poniendo \(\alpha = n/2\) nos da que
\begin{displaymath}
	\left( \frac{r^n}{2} w \right)'
	+
	\left( w' r^{n-1} \right)'
	= 0
	\overset{\int}{\implies}
	r^{n-1} w' + \frac{1}{2} r^n w = C
\end{displaymath}
donde \(C\) es una constante. Vamos a suponer que \(w\) decae de forma
brutal en infinito. Específicamente,
\begin{displaymath}
	\lim_{r\to \infty} r^{n-1} w' 
	=
	0
	=
	\lim_{r\to \infty} r^{n} w',
\end{displaymath}
de esta forma \(A = 0\) (¿por qué?) y podemos resolver la ecuación.
Dándonos que \(w(r) = B e^{-\frac{1}{4} r^{2}}\) con \(B\in \R\)
constante. Así, la ecuación~\eqref{separable:calor} nos queda
\begin{displaymath}
	u(t,x) = \frac{B}{t^{n/2}} e^{-\frac{\abs{x}^{2}}{4t}}.
\end{displaymath}

\end{document}
