%! Tex Root = edp.tex
\documentclass[../edp.tex]{subfiles}

\begin{document}

{\scshape \hfill 01 de junio, 2023}

\subsubsection{Caso \(n < p \le \infty\): Desigualdad de Morrey}

Terminado el caso \(1\le p < n\), comenzaremos a analizar el caso
\(n < p < \infty\). Específicamente, mostraremos que si
\(W^{1,p}(\Omega)\) entonces \(u\) es Hölder continua (salvo quizá,
conjuntos de medida nula). Recordar que una función \(u\) es
\(\alpha\)-Hölder continua si \(\norm{u(x)-u(y)} \le
C \norm{x-y}^{\alpha}\). El ínfimo de los \(C\) es el
\textit{coeficiente de Hölder}. El espacio de las funciones
\(\alpha\)-Hölder continuas con \(k\)-ésima derivada
\(\alpha\)-Hölder continua se denota por \(\CC^{k,\alpha}\).

\begin{Teorema}[Desigualdad de Morrey]
	Sea \(n < p < \infty\). Entonces existe una constante \(C = C(p,n)
	> 0\) tal que
	\begin{displaymath}
		\norm{u}_{\CC^{0,\gamma}(\R^n)}
		\le
		C
		\norm{u}_{W^{1,p}(\R^n)},
		\qquad\forall u \in \CC^{1}(\R^n)
	\end{displaymath}
	donde \(\gamma = 1 - (n/p)\) y 
	\begin{displaymath}
		\norm{u}_{\CC^{0, \gamma}(\R^n)}
		=
		\norm{u}_{L^{\infty}(\R^n)}	
		+
		\left[ u \right]_{\CC^{0,\gamma}(\R^n)},
	\end{displaymath}
	con 
	\begin{displaymath}
		\left[ u \right]_{\CC^{0,\gamma}(\R^n)}
		\coloneqq
		\sup_{x\ne y}
		\left\{
			\frac
			{\abs{u(x) - u(y)}}
			{\abs{x - y}^{\gamma}}
		\right\}.
	\end{displaymath}
\end{Teorema}

\begin{Definicion}[Versión]
	Diremos que \(u^{\ast}\) es una versión de \(u\) en \(\Omega\) si
	\(u = u^{\ast}\) c.t.p. en \(\Omega\).
\end{Definicion}

\begin{Teorema}[Estimaciones en \(W^{1,p}\), \(n < p \le \infty\)]
	Sea \(\Omega\subset\R^n\) un abierto acotado con frontera
	\(\CC^{1}\). Sea \(n < p \le \infty\) y \(u\in W^{1,p}(\Omega)\).
	Entonces \(u\) tiene una versión
	\(u^{\ast}\in\CC^{0,\gamma}(\overline\Omega)\), donde \(\gamma = 1
	-n/p\) con
	\begin{displaymath}
		\norm{u^{\ast}}_{C^{0,\gamma}(\overline\Omega)}
		\le
		\norm{u}_{W^{1,p}(\Omega)},
		\qquad
		C = C(p,n,\Omega) > 0.
	\end{displaymath}
\end{Teorema}

\end{document}
