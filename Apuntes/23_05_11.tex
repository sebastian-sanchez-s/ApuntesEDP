%! Tex Root = edp.tex
\documentclass[../edp.tex]{subfiles}

\begin{document}

{\scshape \hfill 11 de mayo, 2023}

Ahora veremos algunas consecuencias del principio del máximo.
\begin{enumerate}[topsep=3pt,itemsep=2pt]
	\item Velocidad infinita de propagación: Sea \(U\) es abierto
	conexo y acotado. Si \(u\in\CC^{(1,2)}(U_T)\cap\CC(\overline{U_T})\)
	satisface
	\begin{displaymath}
	\begin{caligned}
		u_t - \Delta_x u &= 0 &&, \text{ en } U_T\\
		u &= 0 &&, \text{ en } [0,T]\times\partial U\\
		u &= g &&, \text{ en } \lbrace 0\rbrace\times U
	\end{caligned}
	\end{displaymath}
	donde \(g\ge 0\), entonces \(u\) es estrictamente positiva en
	\(U_T\) si \(g\) es estrictamente positiva en alguna parte de
	\(U\).

	\item Unicidad en dominios acotados: Sea \(g\in \CC(\Gamma_{T})\)
	y \(f\in \CC(U_{T})\). Entonces existe a lo más una solución del
	problema
	\begin{displaymath}\label{110523:0}
	\begin{caligned}
		u_t - \Delta_x u &= f &&, \text{ en } U_T\\
		u &= g &&, \text{ en } \Gamma_T
	\end{caligned}\tag{0}
	\end{displaymath}

	\item Principio del máximo para el problema de Cauchy: Supongamos
	que \(u\in \CC^{(1,2)}((0,T]\times\R^n) \cap
	\CC([0,T]\times\R^n)\) satisface
	\begin{displaymath}
	\begin{caligned}
		u_t - \Delta_x u &= 0 &&, \text{ en } (0,T)\times\R^n\\
		u &= g &&, \text{ en } \lbrace 0 \rbrace \times \R^n
	\end{caligned}
	\end{displaymath}
	y satisface la estimación de crecimiento
	\begin{displaymath}
		u(t,x) \leq A\, e^{\alpha\abs{x}^2}
	\end{displaymath}
	donde \(A,\alpha > 0\) son constantes. Entonces,
	\begin{displaymath}
		\sup_{[0,T]\times\R^n} u
		=
		\sup_{\R^n} g.
	\end{displaymath}
\end{enumerate}

\subsection{Métodos de Energía}

\begin{Teorema}[Unicidad hacia atrás]
	Si \(u_1, u_2 \in \CC^{(1,2)}(\overline{U_{T}})\) son soluciones
	de \eqref{110523:0} y \(u_1(T,x) = u_2(T,x)\) para todo \(x\in U\)
   	entonces \(u_1 = u_2\) en \(U_{T}\).
\end{Teorema}
\begin{Demostracion}
	Sea \(w = u_1 - u_2\). Luego, \(w\) resuelve la ecuación del calor
	homogénea con condiciones de borde nulas. Vale decir,
	\begin{displaymath}
	\begin{caligned}
		w_t - \Delta_{x} w &= 0 &, \text{ en } U_{T}\\
		w &= 0 &, \text{ en } \Gamma_{T}
	\end{caligned}
	\end{displaymath}
	Definamos el funcional de energía
	\begin{displaymath}
		e(t) = \int_{U} w^{2}(t,x) \, dx
		\quad\text{ para } 0 \le t \le T.
	\end{displaymath}
	De la demostración anterior tenemos que 
	\begin{displaymath}
		\dot{e}(t) = -2 \int_{U} \abs{\nabla_{x} w}^2 \, dx.
	\end{displaymath}
	Se sigue que
	\begin{displaymath}
		\ddot{e}(t) 
		= 
		-2 \int_{U} \partial_{t}(\abs{\nabla_{x} w}^2) \, dx
		=
		-2 
		\int_{U} 
		2 \abs{\nabla_{x} w} 
		\frac{\nabla_{x} w \cdot \nabla_{x} w_{t}}{\abs{\nabla_{x} w}}\, dx
		=
		-4 \int_{U} \nabla_{x} w \cdot \nabla w_{t} \, dx.
	\end{displaymath}
	Haciendo integración por partes tenemos que
	\begin{displaymath}
		\ddot{e}(t) 
		=
		4 \int_{U} \Delta_{x} w\, w_{t} \, dx
		=
		4 \int_{U} (\Delta_{x} w)^2  \, dx,
	\end{displaymath}
	donde usamos que \(w_t\) se anula en \(\partial U\) y que resuelve
	la ecuación del calor homogénea. Por otro lado, dado que \(w\) se
	anula en \(\partial U\), tenemos que 
	\begin{displaymath}
		\int_{U} (\nabla_{x} w)^2  \, dx
		=
		- \int_{U} w\, \Delta_{x} w \, dx
		\le
		\left( \int_{U} w^2 \, dx \right)^{1/2}
		\left( \int_{U} (\Delta_{x} w)^2 \, dx \right)^{1/2},
	\end{displaymath}
	donde la última desigualdad es por Cauchy-Schwarz. Combinando los
	cálculos anteriores deducimos que
	\begin{displaymath}
		(\dot{e}(t))^{2} 
		= 4 \left( \int_{U} \abs{\nabla_{x} w}^2 \, dx \right)^{2}
		\le 
		\left( \int_{U} w^2 \, dx \right)^{1/2}
		\left(4 \int_{U} (\Delta_{x} w)^2 \, dx \right)^{1/2}
		=
		e(t) \ddot{e}(t).
	\end{displaymath}
	Si \(e(t) = 0\) para todo \(0\le t \le T\) el resultado se
	sigue de la siguiente manera: tendríamos que \(\nabla w = 0\)
	y por lo tanto \(w\) es una función constante. Dado que vale cero
	en la frontera, necesariamente tendría que ser la función
	constante cero (c.t.p.).

	Supongamos que \(e(t) > 0\) para \(t \in [t_1, t_2) \subset [0,
	T]\) donde \(t_2\) es tal que \(e(t_2) = 0\) (ie. \(t_2\) podría
	ser \(T\)). Definamos \(f(t) = \log(e(t))\) sobre \([t_1,t_2)\). 
	Notemos que \(f\) es convexa en \((t_1, t_2)\), pues
	\begin{displaymath}
		\ddot{f(t)} = 
		\frac{\ddot{e}(t)}{e(t)} - \frac{\dot{e}(t)^2}{e(t)^2}
		\ge 0.
	\end{displaymath}
	Luego, si \(\tau \in (0,1)\) y \(t \in (t_1, t_2)\) tenemos
	que
	\begin{displaymath}
		f((1-\tau) t_1 + \tau t)
		\le
		(1-\tau) f(t_1) + \tau f(t).
	\end{displaymath}
	Aplicando \(\exp\) nos queda
	\begin{displaymath}
		e((1-\tau) t_1 + \tau t)
		\le
		e(t_1)^{1-\tau}
		e(t)^{\tau}.
	\end{displaymath}
	Y por lo tanto
	\begin{displaymath}
		0 < e((1-\tau) t_1 + \tau t_2) \le
		e(t_1)^{1-\tau} e(t_2)^{\tau} = 0
	\end{displaymath}
	Obteniéndose una
	contradicción.
\end{Demostracion}

\end{document}
