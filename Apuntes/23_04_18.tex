%! Tex Root = edp.tex
\documentclass[../edp.tex]{subfiles}

\begin{document}

{\scshape \hfill 18 de abril, 2023}

\subsection{Separación de Variables}

Consideremos el problema sobre \(I \times \R^{+}\)
\begin{displaymath}
\begin{caligned}
	u_t &= k u_{xx} &&, x\in I, t > 0\\
	u(x,0) &= \phi(x)\\
	&\text{más condiciones de borde}
\end{caligned}
\end{displaymath}

Las condiciones de borde más comunes son
\begin{table}[h!]
\centering
\begin{tabular}{|c|c|}
	\hline
	Dirichlet & \(u(0,t) = u(\ell,t)\) \\
	\hline
	Neumann   & \(u_{x}(0,t) = u_{x}(\ell, t)\) \\
	\hline
	Robin	  & \(u_{x}(0,t) - a_0 u(0,t) = 0\) \\
			  & \(u_{x}(\ell,t) - a_1 u(\ell,t) = 0\) \\
	\hline
	Periódicas	& \(u(-\ell, t) = u(\ell, t)\) \\
				& \(u_{x}(-\ell, t) = u_{x}(\ell,t)\) \\
	\hline
\end{tabular}
\end{table}

Nos enfocaremos en la condiciones de tipo Dirichlet homogéneas (es
decir, en la frontera ambas valen cero).

En la separación de variables buscamos desacoplar las componentes que
modelan el problema. En el caso del problema que consideramos, \(u\)
depende del espacio y del tiempo, así que buscamos soluciones de la
forma
\begin{displaymath}
	u(t,x) = T(t) E(x).
\end{displaymath}

\textit{
Nótese que esto no quiere decir que todas las soluciones se vean así,
estamos buscando un tipo muy particular de solución.}

Si reemplazamos en la ecuación tenemos que
\begin{displaymath}
	\frac{T'}{k T}
	=
	\frac{E''}{E}
\end{displaymath}
Para que la igualdad se mantenga, es necesario que ambos cocientes
sean una constante (porque dependen de variables distintas). Dígamos
por conveniencia que la constante es \(-\lambda\). Luego,
\begin{displaymath}
	\frac{T'}{k T}
	=
	\frac{E''}{E}
	=
	-\lambda
	\implies
	\begin{cases}
		E'' = -\lambda E \\
		T' = -\lambda T
	\end{cases}.
\end{displaymath}

Ahora tenemos un par de ecuaciones diferenciales ordinarias, que son
más fáciles de resolver. Sin embargo, lo importante aquí es encontrar
\(\lambda\).

Por las condiciones de borde, \(E(0) = E(\ell) = 0\) y tenemos el
problema de valores y funciones propias:
\begin{displaymath}
\begin{caligned}
	-E''(x) &= \lambda E(x) &&, x\in I\\
	E(0) &= 0\\
	E(\ell) &= 0
\end{caligned}
\end{displaymath}
Si \(\lambda\) satisface el problema anterior para \(E\not\equiv 0\)
diremos que es un valor propio del operador \(-\partial_{x}^2\)
asociado a las condiciones de borde. De manera homóloga, diremos que
\(E\) es una función propia.

Tenemos distintos casos dependiendo del signo de \(\lambda\).
\begin{enumerate}
	\item \(\lambda > 0\): La solución de la EDO es una combinación
		lineal de la forma
		\begin{displaymath}
			E(x)
			=
			a_1 \cos(\sqrt{\lambda} x) 
			+
			a_2 \sin(\sqrt{\lambda} x).
		\end{displaymath}
		Imponiendo las condiciones de borde tenemos que \(a_1 = 0\) y
		\(a_2 \sin(\sqrt{\lambda} \ell) = 0\). Se sigue que
		\(\sqrt{\lambda} \ell = n \pi\) con \(n\) un entero positivo.
		Esto nos da un valor propio para cada \(n\) y a su vez una
		función propia por cada \(\lambda_{n}\). Estás son:
		\begin{displaymath}
			E_{n}(x)
			=
			A_n \sin(\sqrt{\lambda_{n}} x)
			\qquad
			n \ge 1,
			A_n \ne 0.
		\end{displaymath}
	\item \(\lambda = 0\): La función es lineal y por las condiciones
		de borde debe ser la constante nula.
	\item \(\lambda < 0\): Las soluciones son de la forma
		\begin{displaymath}
			E(x)
			=
			a_1 \cosh(\sqrt{-\lambda} x)
			+
			a_2 \sinh(\sqrt{-\lambda} x).
		\end{displaymath}
		Imponiendo las condiciones de borde se concluye que \(E \equiv
		0\).
\end{enumerate}

De esta forma, tenemos resuelta la parte espacial del problema y
obtenido el (los) \(\lambda\) podemos resolver la parte temporal.

Como \(T' = -\lambda_{n} k T\) tiene por solución \(T_{n}(t) = B_{n}
e^{-k \lambda_{n} t}\) con \(B_n \in \R\). Así, la solución \(u\) con
los componentes desacoplados se lee
\begin{displaymath}
	u_{n}(t,x)
	=
	T_{n}(t) E_{n}(x)
	=
	C_{n} e^{-k \lambda_{n} t} \sin(\sqrt{\lambda_n} x).
\end{displaymath}
Y satisface el problema expuesto al inicio con condiciones de borde de
dirichlet homogéneas.

Finalmente, nos queda el problema de imponer condiciones iniciales. La
afirmación es que se pueden escoger coeficientes convenientes tal que
la serie
\begin{displaymath}
	u(t,x)
	=
	\sum_{n=1}^{\infty}
	C_n \sin(\sqrt{\lambda_n} x)
	= 
	\phi(x)
\end{displaymath}
Es solución del mismo problema anterior, pero además satisface las
condiciones iniciales. Asumiremos cosas como la convergencia y
mostraremos que tales coeficientes sí existen.

\textbf{Ortogonalidad de Funciones Propias} Sean \(f,g\in
L^2(\Omega)\) con \(\Omega\subset\R^n\). Recordamos que el producto
interno es \(L^2\) es 
\begin{displaymath}
	\langle f,g \rangle
	=
	\int_{\Omega} f(x) g(x) \, dx,
\end{displaymath}
y la norma inducida es \(\norm{f}_{2} = \langle f, f\rangle\). Decimos
que \(f\) y \(g\) son \textit{ortogonales} si \(\langle f,g \rangle =
0\).

\textbf{Condiciones de Borde Simétricas} Diremos que las condiciones
de borde son simétricas si para todo par \(f,g\) que las satisface se
cumple que:
\begin{displaymath}
	\det
	\begin{pmatrix}
		f(x) & g(x) \\
		f'(x) & g'(x)
	\end{pmatrix}
	\Bigg|^{x=\ell}_{x=0}
	= 0.
\end{displaymath}

\begin{Proposicion}
	Consideremos el problema de valores propios
	\begin{displaymath}
	\begin{caligned}
		E'' + \lambda E &= 0 &&, x\in I\\
		E \text{ satisface ciertas condiciones de borde simétricas}.
	\end{caligned}
	\end{displaymath}
	Si \(E_n\) y \(E_m\) son funciones propias asociadas a valores
	propios distintos, entonces son ortogonales.
\end{Proposicion}
\begin{Demostracion}
\end{Demostracion}

Volviendo a nuestro ejemplo, teníamos que
\begin{displaymath}
	\phi(x)
	=
	\sum_{n\ge 0} A_n E_n(x)
\end{displaymath}
tiene condiciones de borde simétricas (verificar), así que
\begin{displaymath}
	\langle \phi(x), E_n \rangle
	=
	A_n \langle E_n, E_n \rangle
	\implies
	A_n
	=
	\frac
	{\langle E_n, \phi(x) \rangle}
	{\langle E_n, E_n \rangle}.
\end{displaymath}

\end{document}
