%! Tex Root = edp.tex
\documentclass[../edp.tex]{subfiles}

\begin{document}

{\scshape \hfill 18 de mayo, 2023}

\section{Espacios de Sobolev}

Consideremos el problema
\begin{displaymath}
\begin{caligned}
	-u'' + u &= f, &(a,b)\\
	u(a) = u(b) &= 0
\end{caligned}
\end{displaymath}
y multipliquemos por una función test \(\phi\in
\CC^{\infty}_{C}((a,b))\). Al hacer integración por partes nos queda
\begin{displaymath}
	\int_{a}^{b} u' \phi'
	+
	\int_{a}^{b} u \phi
	=
	\int_{a}^{b} f \phi.
\end{displaymath}
Esta es la formulación débil del problema y la gracia es que ahora
solo necesitamos que \(u\) sea continuamente diferenciable. Con
herramientas de análisis funcional buscaremos revolver el sistema para
estas funciones menos regulares, y a partir de estas deducir las más
regulares.

\begin{Definicion}[Derivada débil]	
	Sean \(u, v\in L^{1}_{\loc}(\Omega)\) con \(\Omega\subset\R^n\)
	abierto y \(\alpha\) un multiíndice (ie. \(\alpha = (\alpha_1,
	\dots, \alpha_n)\)). Decimos que \(v\) es la \(\alpha\)-derivada
	débil de \(u\) si se cumple que
	\begin{displaymath}
		\int_{\Omega} u D^{\alpha} \phi
		=
		(-1)^{\abs{\alpha}} 
		\int_{\Omega} v \phi
		\qquad
		\forall \phi \in \CC^{\infty}_{C}(\Omega).
	\end{displaymath}
	En tal caso denotamos \(v = D^{\alpha} u\).
\end{Definicion}

\begin{Proposicion}
	Si existe la \(\alpha\)-derivada débil de una función \(u\in
	L^{1}_{\loc}(\Omega)\), entonces es única salvo un conjunto de
	medida nula.
\end{Proposicion}

\end{document}
