%! Tex Root = edp.tex
\documentclass[../edp.tex]{subfiles}

\begin{document}

{\scshape \hfill 25 de abril, 2023}

\begin{Definicion}[Solución Fundamental]
	La solución fundamental para la ecuación del
	calor~\eqref{eq:calor} es la función:
	\begin{equation}\label{Phi:calor}
		\Phi(t,x)
		=
		\begin{caligned}
			\frac{1}{(\sqrt{4\pi t})^{n}} e^{-\abs{x}^2/4t}
			&&,\quad t > 0 \\
			0 &&,\quad t < 0
		\end{caligned}.
	\end{equation}
\end{Definicion}

¿Qué tipo de condición inicial satisface \(\Phi\)? Tenemos las
siguientes observaciones:
\begin{enumerate}[itemsep=2pt, topsep=5pt, leftmargin=.3\textwidth]
	\item Si \(x\ne 0\), entonces \(\Phi(t,x) \to 0\) cuando
		\(t\downarrow 0\).
	\item Si \(x = 0\), entonces \(\Phi(t,x) \to \infty\) cuando
		\(t\downarrow 0\).
\end{enumerate}
Esto sugiere que ``la energía" en un comienzo está concentrada en el
origen. Ahora veamos que la elección de coeficientes para la solución
fundamental no fue al azar, sino que está hecha de tal forma que
\begin{displaymath}
	\int_{\R^n} \Phi(t,x) \, dx = 1
\end{displaymath}
En efecto, un cálculo directo nos muestra que
\begin{align*}
	\frac{1}{(4\pi t)^{n/2}} \int_{\R^n} e^{-\abs{x}^2/4t} \, dx
	&\overset{y=x/\sqrt{4t}}{=}
	\frac{1}{\pi^{n/2}} 
	\int_{\R^n} e^{-\abs{y}^2} \, dy
	=
	\frac{1}{\pi^{n/2}} 
	\prod_{i=1}^{n} 
	\underbrace{\int_{\R} e^{-y_{i}^2} \, dy_{i}}_{\sqrt{\pi}}
	=
	1.
\end{align*}

En particular, tenemos que
\begin{displaymath}
	\lim_{t\downarrow 0}
	\int_{\R^n} \Phi(t,x) \, dx = 1.
\end{displaymath}
Vamos a probar, que en el sentido de las distribuciones, 
se tiene que
\begin{displaymath}
	\Phi(0,x) = \delta_{x=0}.
\end{displaymath}
Vamos a ello. Como \(\Phi(t,x) \in L^{1}(\R^n)\) para todo \(t>0\),
para cada \(t\) existe una distribución asociada \(u_{\Phi}^{t}\) dada
por
\begin{displaymath}
	\langle u_{\Phi}^{t}, \psi \rangle
	\coloneqq
	\int_{\R^n} \Phi(t,x) \psi(x) \, dx
	\qquad
	\psi \in \CC^{\infty}_{C}(\R^n).
\end{displaymath}
Así que la igualdad en el sentido de la distribuciones quiere decir
que
\begin{displaymath}
	\begin{array}{ccc}
	\Phi(0,x) &=& \delta_{x=0}\\
	\rotatebox{90}{=} && \rotatebox{90}{=}\\
	\lim_{t \downarrow 0}\, 
	\langle u_{\Phi}^{t}, \psi \rangle
	&&
	\langle \delta_{x=0}, \psi \rangle
	\\
	\rotatebox{90}{=} && \rotatebox{90}{=}\\
	\lim_{t \downarrow 0}
	\int_{\R^n} \Phi(t,x) \psi(x) \, dx
	&=&
	\psi(0)
	\end{array}
\end{displaymath}

Como \(\Phi\) integra uno, tenemos que
\begin{displaymath}
	\psi(0)
	=
	\int_{\R^n} \Phi(t,x) \psi(0) \, dx
	\qquad
	\forall t > 0.
\end{displaymath}
Luego,
\begin{displaymath}
	\int_{\R^n} \Phi(t,x) \psi(x) \, dx - \psi(0)
	=
	\int_{\R^n} \Phi(t,x) \left( \psi(x) - \psi(0) \right) \, dx
\end{displaymath}
Vamos a usar la regularidad de \(\psi\) para mostrar que la
diferencia se va a cero. Sea \(\epsilon > 0\). Luego existe un 
\(\sigma_0 = \sigma_0(\epsilon) > 0\) tal que 
\begin{displaymath}
	\abs{ \psi(x) - \psi(0) } < \epsilon/2
	\qquad
	\forall x\in B(0,\sigma_0) \eqqcolon B.
\end{displaymath}
Luego, 
\begin{displaymath}
	\underbrace{
		\abs{\int_{\R^n} \Phi(t,x) \left( \psi(x) - \psi(0) \right) \, dx}
	}_{I}
	\le
	\underbrace{
		\int_{B} \Phi(t,x) (\psi(x) - \psi(0)) \, dx
	}_{I_1}
	+
	\underbrace{
		\int_{\R^n \setminus B} \Phi(t,x) (\psi(x) - \psi(0)) \, dx
	}_{I_2}
\end{displaymath}

Claramente \(\abs{I_1} \le \epsilon/2\). Para acotar \(I_2\) notamos
que
\begin{displaymath}
	\abs{I_2}
	\le
	2 \norm{\psi}_{L^{\infty}}
	\int_{\R^n \setminus B} \Phi(t,x) \, dx
	\le
	2 \norm{\psi}_{L^{\infty}}
	\frac{1}{(4 t \pi)^{n/2}}
	\int_{\R^n \setminus \overline{B}} e^{-\abs{x}^2/4t} \, dx
\end{displaymath}
Notemos que \(e^{-\abs{x}^2/4t} = e^{-\abs{x}^2 / 8t} e^{-\abs{x}^2 /
8t} \le e^{-\sigma_0^2 / 8t} e^{-\abs{x}^2 / 8t}\). Se sigue que
\begin{displaymath}
	\abs{I_{2}}
	\le
	C 
	\underbrace{\frac{e^{-\sigma_0^2 / 4t}}{t^{n/2}}}_{(t\downarrow 0) \to 0}
	\underbrace{\int_{\R^n} e^{-\abs{x}^2 / 8t} \, dx}_{\le \infty}.
\end{displaymath}
En particular, si \(t_0\) es suficientemente pequeño podemos decir que
\(\abs{I_2} < \epsilon/2\). De esta forma,
\begin{displaymath}
	\abs{I} < \epsilon.
\end{displaymath}

\begin{Teorema}
	La solución fundamental satisface
	\begin{displaymath}
	\begin{caligned}
		\Phi_{t}(t,x) - \Delta_{x} \Phi(t,x) &= 0
		&&, (t,x) \in \R^{+}\times\R^n\\
		\lim_{t\downarrow 0} \Phi(t,x) &= \delta_{x=0}.
	\end{caligned}
	\end{displaymath}
	En el sentido de las distribuciones.
\end{Teorema}

\textbf{Problemas de condición inicial} Consideremos el problema
\begin{displaymath}
	\begin{caligned}
		u_{t} - \Delta_{x} u &= 0
		&&, \R^{+} \times \R^n \\
		u &= g &&, \left\{ 0 \right\}\times \R^n
	\end{caligned}
\end{displaymath}
donde \(g\) es una función dada.

\begin{Teorema}
	Supongamos que \(g\in \CC(\R^{n}) \cap L^{\infty}(\R^n)\). La
	función definida por
	\begin{displaymath}
		u(t,x)
		=
		\Phi(t,x) \ast_{x} g
		=
		\int_{\R^n} \Phi(t,x-y) g(y) \, dy,
	\end{displaymath}
	satisface
	\begin{enumerate}[topsep=2pt, itemsep=2pt, leftmargin=0.3\textwidth]
		\item \(u\in \CC^{\infty}(\R^{+} \times \R^n)\);
		\item \(u_t - \Delta_{x} u = 0\) en \(\R^{+} \times \R^n\);
		\item \(\lim_{t\downarrow 0, x\to x_0} u(t,x) = g(x_0)\) para
			todo \(x_0 \in \R^n\).
	\end{enumerate}
\end{Teorema}
\begin{Demostracion}
	1. (ejercicio) 

	Para el punto 2. notamos que por TCD (seguramente)
	\begin{displaymath}
		u_{t} - \Delta_{x} u 
		= 
		\int_{\R^n} \Phi_{t}(t,x-y) g(y) \, dy
		-
		\int_{\R^n} \Delta_{x} \Phi(t,x-y) \, g(y) \, dy
		=
		\int_{\R^n} (\Phi_{t}(t,x-y) - \Delta_{x} \Phi(t,x-y)) g(y) \, dy
		=
		0.
	\end{displaymath}

	Para el punto 3., fijando \(x_0\) notamos que
	\begin{displaymath}
		g(x_0) = \int_{\R^n} \Phi(t,x) g(x_0) \, dx
	\end{displaymath}
	para todo \(t > 0\). Sea \(\epsilon > 0\) y tomemos \(\sigma_0 =
	\sigma_0(\epsilon)\) tal que
	\begin{displaymath}
		\abs{g(y) - g(x_0)} < \epsilon/2
		\qquad
		\forall y\in B(x_0, \sigma_0) \eqqcolon B.
	\end{displaymath}
	Luego, si \(x \in B(x_0, \sigma_0/2)\) tenemos que
	\begin{align*}
		\abs{u(t,x) - g(x_0)}
		&\le
		\abs{\int_{\R^n} \Phi(t,x-y) (g(y) - g(x_0)) \, dy}
		\\&\le
		\underbrace{
			\int_{B} \Phi(t,x-y) \abs{g(y) - g(x_0)} \, dy
		}_{I_1}
		+
		\underbrace{
		\int_{\R^n \setminus B} \Phi(t,x-y) \abs{g(y) - g(x_0)} \, dy
		}_{I_2}
	\end{align*}
	Acotar \(\abs{I_1} < \epsilon/2\) es directo usando la regularidad de
	\(g\) y que \(\Phi\) integra uno. Por otro lado, para acotar \(I_2\)
	notemos que si \(\abs{x - x_0} < \sigma_0 / 2\), entonces para
	\(\abs{y-x_0} > \sigma_0\) se tiene que \(\abs{y-x_0} \le 2
	\abs{y-x}\). Así que 
	\begin{align*}
		\abs{I_2}
		&\le
		2 \norm{g}_{L^{\infty}}
		\int_{\R^n \setminus B} \Phi(t,x-y) \, dy
		\\&\le
		\frac{C}{t^{n/2}} 
		\int_{\R^n \setminus B} e^{-\abs{y-x_0}^2/16t} \, dy
	\end{align*}
	Haciendo el cambio de variables \(z = \frac{y-x_0}{\sqrt{t}}\)
	tenemos que
	\begin{displaymath}	
		\frac{C}{t^{n/2}} 
		\int_{\R^n \setminus B} e^{-\abs{y-x_0}^2/16t} \, dy
		\le
		C \int_{\R^n \setminus B(0, \sigma_0/\sqrt{t})}
		e^{-\abs{z}^2 / 16t} \, dz
		\le 
		\epsilon / 2
	\end{displaymath}
	Tomando \(0 < t < t_0\) para algún \(t_0\) suficientemente
	pequeño.
\end{Demostracion}


\end{document}
