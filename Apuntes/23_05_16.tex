%! Tex Root = edp.tex
\documentclass[../edp.tex]{subfiles}

\begin{document}

{\scshape \hfill 16 de mayo, 2023}

\begin{Definicion}[Cilindro Circular Parabólico]
	Sea \((t,x) \in \R^{+}\times\R^n\) y \(r > 0\). El cilindro
	circular parabólico (cerrado) de radio \(r\) y altura \(r^2\) con techo
	centrado en \((t,x)\) es 
	\begin{displaymath}
		C(t,x;r)
		\coloneqq
		\left\{ 
			(s,y) \in \R^{+}\times\R^n \colon
			\abs{x-y} \le r^2
			\land
			t-r^2 \le s \le t
	   	\right\}.
	\end{displaymath}
\end{Definicion}

\begin{Teorema}[Regularidad de la Ecuación del Calor]
	Sea \(u\in \CC^{(1,2)}(U_{T})\) una solución de la ecuación
	del calor en \(U_T\), entonces \(u\in \CC^{\infty}(U_T)\).
\end{Teorema}
\begin{Demostracion}
	Fijemos \((x_0, t_0) \in U_{T}\). Sea \(C_1 = C(x_0,t_0;r)\) el
	cilindro de radio \(r\) y altura \(r^2\) contenido en \(U_T\). 

	Sean \(C_2\) y \(C_3\) los cilindros contenidos en \(C_1\) que
	comparten el mismo techo de radio \(0.75r\) y \(0.5r\), respectivamente.
	Sea \(\psi \in \CC^{\infty}_{C}([0,t_0]\times\R^n)\) tal que
	(1) \(0 \le \psi \le 1\); (2) \(\psi = 1\) en \(C_2\) y (3) \(\psi
	= 0\) fuera de \(C_1\).

	Consideremos \(v^{\epsilon} = \psi\cdot (\eta_{\epsilon} \ast u) =
	\psi u^{\epsilon}\). Notar que
	\(v^{\epsilon}\) está definido sobre todo \([0,t_0]\times\R^n\).
	Por otro lado, 
	\begin{displaymath}
		v^{\epsilon}_{t}
		=
		\psi_{t} u^{\epsilon}
		+
		\psi u^{\epsilon}_{t}
	\end{displaymath}
	y
	\begin{align*}
		\nabla v^{\epsilon}
		=
		\nabla \psi \, u^{\epsilon}
		+
		\nabla u^{\epsilon} \, \psi
		\Rightarrow
		\Delta v^{\epsilon}
		=
		\Delta \psi\, u^{\epsilon}
		+
		2 \nabla \psi \, \nabla u^{\epsilon}
		+
		\Delta u^{\epsilon} \, \psi
	\end{align*}
	Así que \(v^{\epsilon}\) resuelve
	\begin{displaymath}
	\begin{caligned}
		v^{\epsilon}_{t} - \Delta u^{\epsilon}
		&=
		\psi_{t} u^{\epsilon}
		-
		2 \nabla \psi\, \nabla u^{\epsilon}
		-
		\Delta \psi\, u^{\epsilon}
		= \tilde{f}
		&, \text{ en } (0,t_0)\times\R^n
		\\
		v &= 0 
		&, \text{ en } \left\{ t=0 \right\}\times\R^n
	\end{caligned}
	\end{displaymath}
	Por la unicidad del problema de Cauchy se tiene que
	\begin{displaymath}
		v^{\epsilon}(t,x)
		=
		\int_{0}^{t}
		\int_{\R^n}
			\Phi(t - s, x - y)
			\tilde{f}(s,y)
		\, dy\, ds	
		\qquad
		(t,x) \in (0,t_0)\times\R^n
	\end{displaymath}

	Sea \((x,t) \in C_3\). Aquí \(\psi = 1\), por lo tanto
	\(v^{\epsilon} = u^{\epsilon}\). Se sigue que
	\begin{align*}
		u^{\epsilon}(t,x)
		&=
		\int_{0}^{t}
		\int_{\R^n}
			\Phi(t - s, x - y)
			\tilde{f}(s,y)
		\, dy\, ds	
		\\&=
		\int_{0}^{t}
		\int_{\R^n}
			\Phi(t - s, x - y)
			\left(
			(\psi_{t} - \Delta \psi)
			\, u^{\epsilon}
			-
			2 \nabla \psi\, \nabla u^{\epsilon}
			\right)
		\, dy \, ds 
		\\&=
		\int_{C_1}
			\Phi(t - s, x - y)
			\left(
			(\psi_{t} - \Delta \psi)
			\, u^{\epsilon}
			-
			2 \nabla \psi\, \nabla u^{\epsilon}
			\right)
		\, dy \, ds.
	\end{align*}
	Haciendo integración por partes en el segundo sumando tenemos
	\begin{displaymath}
		\int_{C_1}	
			\Phi(t - s, x - y)
			2 \nabla \psi\, \nabla u^{\epsilon}
		\, dy\, ds
		=
		-2
		\int_{C_1}	
		\left(
			\Phi(t - s, x - y)
			\Delta \psi
			+
			\nabla \Phi(t - s, x - y)
			\nabla \psi
		\right) u^{\epsilon}
		\, dy\, ds.
	\end{displaymath}
	Volviendo a la expresión anterior
	\begin{displaymath}
		u^{\epsilon}(t,x)
		=
		\int_{C_1}
		\left(
			\Phi(t - s, x - y)
			(\psi_{t} + \Delta \psi)
			+
			2
			\nabla \Phi(t - s, x - y)
			\nabla \psi
		\right) \, u^{\epsilon}
		\, dy \, ds.
	\end{displaymath}
	Dado que \(\psi \equiv 1\) en \(C_2\), tenemos que
	\begin{align*}
		u^{\epsilon}(t,x)
		&=
		\int_{C_1 \setminus C_2}
		\left(
			\Phi(t - s, x - y)
			(\psi_{t} + \Delta \psi)
			+
			2
			\nabla \Phi(t - s, x - y)
			\nabla \psi
		\right) \, u^{\epsilon}(s,y)
		\, dy \, ds
		\\&\eqqcolon
		\int_{C_1 \setminus C_2}
			\mathcal{K}(t,s,x,y)
			u^{\epsilon}(s,y)
		\, dy \, ds
	\end{align*}
	Dado que \(\kappa \in \CC^{\infty}\) en \(C_3\), tenemos que
	\(u^{\epsilon} \in \CC^{\infty}(C_3)\). Finalmente, dado que
	\(u\in \CC^{2}\), se tiene que \(u^{\epsilon} \to u\) cuando
	\(\epsilon \to 0\) de manera uniforme. De esta forma, concluimos
	que \(u\in \CC^{\infty}\).
\end{Demostracion}

\begin{Teorema}
	Existen constantes \(C_{k,\ell}\) con \(k,\ell = 0,1,2,\dots\) tal
	que 
	\begin{displaymath}
		\max_{C(t,x;r/2)} 
		\abs{D_{x}^{\alpha} D^{\ell}_{t} u}
		\le
		\frac{C_{k,\ell}}{r^{k+2\ell+n+2}} 
		\norm{u}_{L^1(C(t,x;r))}
		\qquad
		\abs{\alpha} = k
	\end{displaymath}
	para todo cilindro \(C(t,x;r/2) \subset C(t,x;r) \subset U_T\) 
	y toda solución \(u\) de la ecuación del calor en \(U_T\).
\end{Teorema}
\begin{Demostracion}
	Sin pérdida de generalidad digamos que \((t,x) = (0,0)\). Primero
	probaremos el resultado para \(r=1\). Llamemos \(C_1\) y \(C_2\)
	a los cilindros de radio \(1\) y \(1/2\), respectivamente. Por la
	demostración anterior, sabemos que para \((t,x) \in C_2\) tenemos
	la representación:
	\begin{displaymath}
		u(t,x)
		=
		\int_{C_1}
			\mathcal{K}(t,s,x,y)
			u(s,y)
		\, dy\, ds.
	\end{displaymath}
	Fijemos \(k\) y \(\ell\). Sea \(\abs{\alpha} = k\). Se sigue que
	\begin{align*}
		\abs{D^{\alpha}_{x} D^{\ell}_{t} u(t,x)}
		&\le
		\int_{C_1}
			\abs{D^{\alpha}_{x} D^{\ell}_{t}\, \mathcal{K}(t,s,x,y)}
			\abs{u(s,y)}
		\, dy\, ds
		\\&\le	
		C_{k,\ell} 
		\norm{u}_{L^{1}(C(1))}
		\qquad
		\forall (t,x) \in C(0,0;1/2).
	\end{align*}
	Para el resultado general, sea \(r > 0\) y definamos
	\begin{displaymath}
		v(t,x) = v(r^2 t, rx)
		\qquad
		(t,x) \in C(1).
	\end{displaymath}
	Es claro que \(v\) es solución de la ecuación del calor. Por otro
	lado,
	\begin{displaymath}
		D^{\alpha}_{x} D^{\ell}_{t} v(t,x)
		=
		r^{2\ell + k} 
		D^{\alpha}_{x} D^{\ell}_{t} u(r^2 t, r x)
	\end{displaymath}
	y
	\begin{align*}
		\norm{v}_{L^{1}(C_1)}
		&=
		\int_{0}^{1} 
		\int_{B(0,1)}
			\abs{u(r^2 t, r x)}
		\, dx\, dt
		\\&=
		\int_{0}^{r^2} 
		\int_{B(0,r)}
			\abs{u(s, y)}
			r^{n+2}
		\, dy\, ds
		\\&=
		\frac{1}{r^{n+2}}
		\norm{u}_{L^{1}(C(0,0;r)}.
	\end{align*}
	Aplicando el primer resultado nos queda
	\begin{displaymath}
		\abs{D^{\alpha}_{x} D^{\ell}_{t} u(s,y)}
		\le
		\frac{C_{k,\ell}}{r^{2\ell + k + n + 2}} 
		\norm{u}_{L^{1}(C(0,0;r)}
		\qquad
		\forall (s,y) \in C(0,0;r/2).
	\end{displaymath}
\end{Demostracion}

\end{document}
