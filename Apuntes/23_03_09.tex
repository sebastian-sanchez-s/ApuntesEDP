%! Tex Root = edp.tex
\documentclass[../edp.tex]{subfiles}

\begin{document}

{\scshape \hfill 09 de Marzo, 2023}

\section{Preliminares: Teoría de medida}

Para un conjunto \(X\), decimos que una colección de subconjuntos \(M\) de 
\(X\) es una sigma álgebra si contiene a \(X\) y es cerrado bajo complementos
y uniones numerables. En símbolos:
\begin{displaymath}
	X\in M
	;\quad
	A\in M \implies A^c \in M
	;\quad
	A_1, A_2, \dots, M \implies \bigcup_{n\ge 1} A_n \in M.
\end{displaymath}

El par \((X, M)\) se dice \textit{espacio medible} y 
los elementos de \(M\) son los \textit{conjuntos medibles}.

Los espacios topológicos tienen una sigma álgebra inducida por sus abiertos (la
sigma álgebra más pequeña que contiene a todos los abiertos),
llamada \textit{\(\sigma\)-álgebra de Borel} y denotada por \(\BB(X)\).
Para \(X,Y\) espacios topológicos con sus respectivas sigma álgebras, 
decimos que \(f\colon X \to Y\)
es una \textit{función medible} si \(f^{-1}(A)\) es un conjunto medible
para todo \(A\) abierto.

Decimos que \(m\colon X \to \R_{\ge 0}\) es una
\textit{función de medida} si el vacío tiene medida cero y la medida de una 
unión (numerable) disjunta de conjuntos medibles es la suma de la medida de 
los conjuntos. En símbolos:
\begin{displaymath}
	m(\varnothing) = 0
	\text{ y }
	m\left(\bigcup_{n\ge 1} A_i\right) = \sum_{n\ge 1} m(A_i)
	\text{ con } A_1, A_2, \dots \text{ disjuntos.}
\end{displaymath}
A la tripleta \((X,M,m)\) se le dice \textit{espacio de medida}.

Casi siempre trabajaremos en \(\R^n\) y la medida estándar para esta será la
medida de Lebesgue. Esta le asigna a los intervalos su largo (en \(\R\)) y se
extiende a \(\R^n\) como la medida que le asigna a los \(n\)-cubos su volumen.

Decimos que un conjunto es despreciable si tiene medida nula. Cuando una
propiedad se cumple salvo un conjunto de medida nula, decimos que la propiedad
se cumple casi en todas partes o \(m\)-ctp.

Una \textit{función indicatriz} \(\chi_{A}\colon X \to [0, 1]\) se define por 
\begin{displaymath}
	\chi_{A}(x) = 
	\begin{cases}
		1, x\in A\\
		0, x\not\in A
	\end{cases}.
\end{displaymath}

Una \textit{función simple} es una combinación lineal finita de funciones
indicatrices:
\begin{displaymath}
	s = \sum_{n=1}^{N} a_n \chi_{A_n}(x)
	,\quad A_n \in M, a_n \in \R_{\ge 0}.
\end{displaymath}

Definimos la integral de una función simple como
\begin{displaymath}
	\int s \d{m}
	=
	\sum_{n=1}^{N} a_n m( A_n ).
\end{displaymath}

Para una función positiva \(f\) la integral se define por
\begin{displaymath}
	\int f \d{m}
	=
	\sup_{s\le f} \int s \d{m},
\end{displaymath}
donde el supremo se toma sobre todas las funciones simples menores a \(f\).

Para una función con signo \(f\) definimos \(f^{+} = \max\left\{0, f\right\}\) y
\(f^{-} = \max\left\{0, -f\right\}\), de esta forma, \(f = f^{+} - f^{-}\) y 
la integral de \(f\) es 
\begin{displaymath}
	\int f \d{m} = \int f^{+} \d{m} - \int f^{-} \d{m}.
\end{displaymath}
Obsérvese que \(\abs{f} = f^{+} + f^{-}\). Decimos que una función es integrable
si \(\int \abs{f} < \infty\).

Al conjunto de funciones integrables lo denotamos por \(L^{1}\) y en general el
conjunto \(L^{p}\) se define como
\begin{displaymath}
	L^{p} \coloneqq 
	\left\{
		f \colon X\to \R
		\mid
		\Big(\int \abs{f}^{p} \d{m} \Big)^{1/p} < \infty
	\right\}.
\end{displaymath}

\end{document}
