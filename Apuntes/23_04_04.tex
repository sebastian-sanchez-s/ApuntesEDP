%! Tex Root = edp.tex
\documentclass[../edp.tex]{subfiles}

\begin{document}

{\scshape \hfill 04 de abril, 2023}

\subsection{Función de Green para una bola}

Consideremos la bola unitaria en \(B(0,1) \subset \R^n\) con \(n\ge 3\). 
Buscamos resolver el problema corrector
\begin{displaymath}
\begin{cases}
\begin{aligned}
	\Delta \varphi^{x} &= 0 &&,\text{ en } B(0,1)\\
	\varphi^{x} &= \Phi(\cdot -x) &&,\text{ en } \partial B(0,1)
\end{aligned}
\end{cases}.
\end{displaymath}
Nótese que si \(x = 0\), entonces \(\varphi^{0}(y) = \Phi(1)\) para \(y\in
\partial B(0,1)\) y el problema se satisface. 
Consideremos entonces \(x\ne 0\); 
Denotamos por \(\tilde{x} \coloneqq
\frac{x}{\abs{x}^2}\) a su \textbf{punto dual} o inverso de \(x\) con respecto a 
\(\partial B(0,1)\). Fijando \(x\), tenemos que el mapa 
\( y \to \Phi(y-\tilde{x}) \) es armónico para \(y\ne \tilde{x}\). Más aún, el
mapa \(y \to \frac{1}{\abs{x}^{n-2}} \Phi(y-\tilde{x})\) es armónico para \(y\ne
\tilde{x}\). Definimos entonces,
\begin{displaymath}
	\varphi^{x}(y) \coloneqq \Phi(\abs{x} (y-\tilde{x})).
\end{displaymath}
Este mapa es armónico si \(y\ne \tilde{x}\). En efecto,
\begin{displaymath}
	\partial^{2}_{y_i} \varphi^{x}(y)
	=
	\partial_{y_i}
	(
	\partial_{y_i} 
	\Phi(\abs{x}(y-\tilde{x})
	\abs{x}
	)
	= ...
\end{displaymath}

Nótese que si \(y\in \partial B(0,1)\) entonces,
\begin{align*}
	\abs{x-y}^2 
	&=
	\abs{x}^2 - 2 y\cdot x + \abs{y}^2
	\\&=
	\abs{x}^2 
	\left(
	1 - 2y\cdot \frac{x}{\abs{x}^2} + \frac{1}{\abs{x}^2}
	\right)
	\\&=
	\abs{x}^2 
	\left(
	\abs{y}^2 - 2y\cdot \tilde{x} + \abs{\tilde{x}}^2
	\right)
	\\&=
	\abs{x}^2 \abs{y-\tilde{x}}^2,
\end{align*}
y tenemos que \(\abs{x-y} = \abs{x} \abs{y-\tilde{x}}\). Se sigue que
\begin{displaymath}
	\varphi^{x}(y)
	=
	\Phi(\abs{x} (y-\tilde{x}))
	=
	\Phi(y-x)
\end{displaymath}
pues \(\Phi\) es radial. La función de Green para la bola se lee
\begin{displaymath}
	G(x,y)
	=
	\begin{cases}
	\begin{aligned}
		\Phi(y-x) - \Phi(\abs{x} (y-\tilde{x}))
		&\quad, x\ne y\in \overline{B(0,1)}, x\ne 0
		\\
		\Phi(y) - \Phi(1) 
		&\quad, y\in \overline{B(0,1)}\setminus \left\{ 0 \right\} x\ne 0 
	\end{aligned}.
	\end{cases}
\end{displaymath}

Luego, si \(u\in \CC^{2}(\Omega)\cap\CC(\overline{\Omega})\) es
solución del problema
\begin{displaymath}
\begin{cases}
\begin{aligned}
	\Delta u &= 0 &&, \text{ en } \Omega
	\\
	u &= g &&,\text{ en } \partial\Omega
\end{aligned}
\end{cases},
\end{displaymath}
por la fórmula de representación, 
\begin{displaymath}
	u(x) 
	=
	-\int_{\partial B(0,1)}
		g(y) 
		\partial_{\n} G(x,y) 
	\, dS(y)
\end{displaymath}

Identifiquemos a \(\partial_{\n} G(x,y)\), 
\begin{displaymath}
	\partial_{\n} G(x,y)
	=
	\sum_{i=1}^{n} y_i \partial_{y_i} G(x,y)
\end{displaymath}
Primero calculemos cada sumando,
\begin{align*}
	\partial_{\n} G(x,y) 
	&=
	\partial_{y_i} \Phi(y-x)
	-
	\partial_{y_i} \Phi(\abs{x} (y - \tilde{x}))
	\\&=
	\frac{1}{n \alpha(n)} \frac{x_i - y_i}{\abs{x-y}^n}
	-
	\frac{1}{n \alpha(n)} \frac{y_i \abs{x}^2 - x_i}{(\abs{x}
	\abs{y-\tilde{x}}^{n}}
	\\&=
	\frac{1}{n \alpha(n)} \frac{x_i - y_i}{\abs{x-y}^n}
	-
	\frac{y_i \abs{x}^2 - x_i}{\abs{x-y}^n}.
\end{align*}
Se sigue que
\begin{displaymath}
	\partial_{\n} G(x,y)
	=
	-\frac{1}{n\alpha(n)} (1-\abs{x}^2) \frac{1}{\abs{x-y}^n}.
\end{displaymath}
 Definimos el \textbf{kernel de Poisson} para la bola como 
 \begin{displaymath}
	 k(x,y)
	 \coloneqq
	 -\partial_{\n} G(x,y)
	 =
	 \frac{1}{n\alpha(n)} \frac{1-\abs{x}^2}{\abs{x-y}^n}
 \end{displaymath}

 Haciendo un cambio de variables la tenemos la fórmula para cualquier
 bola de radio \(R\). Esta se lee
 \begin{displaymath}\label{representacion:bola}
	 u(x)
	 =
	 \int_{\partial B(0,R)} g(y) k(x,y) \, dS(y),
	 \tag{FRB}
 \end{displaymath}
 con el kernel de Poisson 
 \begin{displaymath}\label{KPB}
	 k(x,y) \coloneqq
	 \frac{1}{n\alpha(n)} \frac{R^2-\abs{x}^2}{\abs{x-y}^n R}
	 \tag{KPB}
 \end{displaymath}

\begin{Teorema}
	Si \(g\in \CC(\partial B(0,R))\), entonces la fórmula de
	representación~\eqref{FRB} satisface:
	\begin{enumerate}
		\item \(u \in \CC^{\infty}(B(0,R))\);
		\item \(\Delta u = 0\) en \(B(0,R)\);
		\item \(\lim_{x\to x_0} u(x) = g(x_0)\) donde \(x_0\in
		\partial B(0,R)\) y \(x\in B(0,R)\).
	\end{enumerate}
\end{Teorema}
\begin{Demostracion}
\end{Demostracion}

\end{document}
