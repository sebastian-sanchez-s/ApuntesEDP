%! Tex Root = edp.tex
\documentclass[../edp.tex]{subfiles}

\begin{document}

{\scshape \hfill 21 de Marzo, 2023}

\begin{Teorema}[Caracterización de Funciones Armónicas]
	Si \(u\in\CC^{2}(\Omega)\) satisface la propiedad de la media, entonces es
	armónica.
\end{Teorema}
\begin{Demostracion}
	
\end{Demostracion}

\begin{Teorema}[Principio del Máximo]
	Sea \(\Omega\) un subconjunto de \(\R^n\) abierto y acotado. Si \(u\in
	\CC^{2}(\Omega) \cap \CC(\overline{\Omega})\) es una función armónica,
	entonces
	\begin{enumerate}[topsep=0pt,itemsep=0pt]
		\item El máximo de la función se alcanza en la frontera.
			\begin{displaymath}
				\max_{\overline\Omega} u 
				= 
				\max_{\partial\Omega} u.
			\end{displaymath}
		\item Si además \(\Omega\) es conexo y el máximo se alcanza dentro de
			\(\Omega\), entonces \(u\) es constante.
			\begin{displaymath}
				\exists x \in \Omega 
				\mid
				u(x) = \max_{\overline\Omega} u
				\implies
				u(x) = c
				\qquad c\in \R.
			\end{displaymath}
	\end{enumerate}
	El primer punto es el principio débil y el segundo el fuerte. Nótese que el
	segundo vale incluso si \(\Omega\) no es acotado.
\end{Teorema}
\begin{Demostracion}
	Primero veamos que el principio fuerte implica el débil. En efecto, si no
	fuera así, se tendría que \(\max_{\overline\Omega} u > \max_{\partial
	\Omega} u\). Es decir, existiría un \(x_0\in \Omega\) tal que \(u(x_0) =
	\max_{\overline\Omega}\). Consideremos \(\Omega_0\) la componente conexa que
	contiene a \(x_0\). Luego, por el principio del máximo fuerte 
	\begin{displaymath}
		\max_{\overline \Omega} u
		=
		\max_{\overline \Omega_0} u
		=
		\max_{\partial \Omega_0} u
		\le 
		\max_{\partial \Omega} u,
	\end{displaymath}
	donde la última igualdad se debe a que la frontera es conexa.
	Tenemos una contradicción, por lo tanto el Ppio fuerte implica el débil.

	Ahora queda probar el Ppio fuerte. Sea \(x_0\in \Omega\) tal que \(u(x_0) =
	\max_{\overline\Omega} u \eqqcolon M\). Consideremos el conjunto
	\begin{displaymath}
		A 
		\coloneqq
		\left\{
			x\in \Omega
			\mid
			u(x) = M
		\right\}.
	\end{displaymath}
	Vamos a probar que \(A = \Omega\) mostrando que es abierto y cerrado
	(relativamente a \(\Omega\)).

	\noindent \underline{\(A\) cerrado:} Basta notar que \(A = u^{-1}(M)\) y
	\(u\) es continua, así que \(A\) es cerrado.

	\noindent \underline{\(A\) abierto:} Sea \(x \in A\). Como \(u\) es
	armónica, satisface la propiedad de la media. Se sigue que
	\begin{displaymath}
		M
		=
		u(x)
		=
		\fint_{B(x,r)} u(y) \, dy,
		\quad
		\forall r > 0 \mid \overline{B(x,r)} \subset \Omega.
	\end{displaymath}
	Luego,
	\begin{displaymath}
		\fint_{B(x,r)} (M - u(y)) \, dy
		=
		0
	\end{displaymath}
	Por lo tanto \(u(y) = M\) para todo \(y \in B(x,r)\). Se sigue que \(B(x,r)
	\subset A\) y por lo tanto \(A\) es abierto.
\end{Demostracion}

\begin{Corolario}[Unicidad en Poisson para dominios acotados con datos de borde]
	Si \(\Omega \subset \R^n\) abierto y acotado, entonces existe a lo más una
	solución \(u\in \CC^2(\Omega) \cap \CC(\Omega)\) del problema
	\begin{displaymath}
		\begin{array}{rrc}
			-\Delta u &= f 
			&, \text{ en } \Omega
			\\
			u &= g
			&, \text{ en } \partial\Omega
		\end{array},
	\end{displaymath}
	donde \(f\) y \(g\) son funciones continuas en \(\Omega\) y
	\(\partial\Omega\) respectivamente.
\end{Corolario}
\begin{Demostracion}
	Supongamos que \(u_1, u_2\) son dos soluciones, entonces \(v = u_1 - u_2\)
	es armónica en \(\Omega\) y \(v = 0\) en \(\partial \Omega\). Por el 
	Ppio del máximo, \(v = 0\) en \(\Omega\). 
\end{Demostracion}

\begin{Ejemplo}
\begin{enumerate}
	\item En \(\Omega = \R^n \setminus \overline{B(0,1)}\), el problema
	\begin{alignat*}{2}
		\Delta u &= 0
		&, \text{ en }\Omega
		\\
		u &= 0
		&, \text{ en } \partial\Omega,
	\end{alignat*}
	No tiene solución única. En efecto, \(u=0\) y 
	\begin{displaymath}
		u(x)
		=
		\begin{cases}
			\log 1/\abs{x} &, n=2\\
			1/\abs{x}^{n-2} - 1, n \ge 3
		\end{cases},
	\end{displaymath}
	son dos soluciones.

	\item En \(\Omega = \R^n_{+} \coloneqq \left\{(x', x_n) \in \R^{n-1}\times
	\R_{+}\right\}\), el problema
	\begin{alignat*}{2}
		\Delta u &= 0
		&, \text{ en }\Omega
		\\
		u &= 0
		&, \text{ en } \partial\Omega,
	\end{alignat*}
	tampoco tiene solución única, pues \(u=0\) y \(u=x_n\) son solución.
\end{enumerate}
\end{Ejemplo}

\begin{Teorema}[de Liouville]
	Si \(u\colon \R^n \to \R\) es una función armónica y acotada, entonces es
	constante.
\end{Teorema}
\begin{Demostracion}
	Asumimos que \(u \in \CC^{\infty}(\R^n)\). Obsérvese que las derivadas
	parciales de \(u\) son armónicas, pues
	\begin{displaymath}
		\Delta (\partial_{x_i} u)
		=
		\partial_{x_i} \Delta u
		= 0.
	\end{displaymath}
	Veremos que las derivadas son nulas y por lo tanto \(u\) será constante.

	Aplicando la propiedad de la media a cada derivada parcial y usando el
	Teorema de la Divergencia, tenemos que
	\begin{displaymath}
		\partial_{x_i} u(x_0)
		=
		\fint_{B(x_0, r)} \partial_{x_i} u(x)\, dx
		=
		\frac{1}{\alpha(n) r^{n}} 
		\int_{\partial B(x_0, r)} u(x) \n^{i}(x)\, dx.
	\end{displaymath}
	Y entonces
	\begin{displaymath}
		\abs{\partial_{x_i} u(x_0)}
		\le
		\norm{u}_{L^{\infty}}
		\frac
			{\abs{\partial B(x_0, r)}}
			{\alpha(n) r^n}
		\le
		\norm{u}_{L^{\infty}}
		\frac{n}{r}.
	\end{displaymath}
	Tomando \(r \to \infty\) se concluye que \(\partial_{x_i} u(x) = 0\) para
	todo \(i = 1,\dots, n\) y para todo \(x\in \R^n\). Por lo tanto \(u\) es
	constante.
\end{Demostracion}

\begin{Corolario}
	Para \(n\ge 3\) y \(f\in \CC^{2}_{C}(\R^n)\), se tiene que toda solución
	acotada \(u\) del problema
	\begin{displaymath}
		-\Delta u = f,
	\end{displaymath}
	es de la forma \(u(x) = \Phi \ast f(x) + C\).
\end{Corolario}
\begin{Demostracion}
	Sabemos que \(\Phi \ast f\) es solución del problema. Dado que
	\begin{displaymath}
		\abs{\Phi \ast f(x)}
		\le
		\norm{f}_{K}
		\int_{K} \Phi(x-y) \, dy
		< \infty
		\quad
		\supp f \subset K \text{ compacto},
	\end{displaymath}
	se tiene que es acotada. Luego, si \(u\) es otra solución acotada, la
	función \(v = \Phi\ast f - u\) es armónica en \(\R^n\) y acotada.
	Por Liouville, es constante. Es decir, \(u = \Phi \ast f + C\) con \(C\in \R\).
\end{Demostracion}
\end{document}
