%! Tex Root = edp.tex
\documentclass[../edp.tex]{subfiles}

\begin{document}

{\scshape \hfill 23 de Marzo, 2023}

\subsection{Regularidad}

\noindent\textbf{Convolución y Regularización}

Para \(\Omega \subset \R^n\) abierto, definimos el conjunto 
\begin{displaymath}
	\Omega_{\epsilon}
	\coloneqq
	\left\{
		x\in \Omega
		\mid
		\dist(x, \partial\Omega) > \epsilon
	\right\}
	,\quad \epsilon > 0.
\end{displaymath}
Además, definimos \(\eta \in \CC^{\infty}_{C}(\R^n)\) como
\begin{displaymath}
	\eta(x)
	\coloneqq
	\begin{cases}
		C \exp( \frac{1}{\abs{x}^2 - 1}
		&, \abs{x} < 1
		\\
		0
		&, \abs{x} \ge 1
	\end{cases},
\end{displaymath}
donde \(C\) es una constante tal que
\begin{displaymath}
	\int_{\R^n}
		\eta(x)
	= 1.
\end{displaymath}
Con esto definimos la \textbf{función regularizante}
\begin{displaymath}
	\eta_{\epsilon}(x)
	\coloneqq
	\frac{1}{\epsilon^{n}}
	\eta(\frac{x}{\epsilon})
	,\quad \epsilon > 0.
\end{displaymath}
La gracia de la función regularizante es que \(\supp \eta_{\epsilon} =
\overline{B(0,\epsilon)}\) y \(\int_{\R^n} \eta_{\epsilon} = 1\).

Para una función \(f\in L^{1}_{\loc}(\Omega)\) y \(x\in\Omega_{\epsilon}\),
definimos
\begin{displaymath}
	f_{\epsilon}(x)
	=
	\eta_{\epsilon} 
	\ast
	f(x)
	=
	\int_{\R^n} \eta_{\epsilon}(x-y) f(y)\, dy.
\end{displaymath}
la regularización de \(f\) en \(\Omega_{\epsilon}\). Para la regularización, se
cumple que
\begin{enumerate}[itemsep=0pt, topsep=1ex]
	\item \(f_{\epsilon} \in \CC^{\infty}(\Omega_{\epsilon})\);
	\item \(f_{\epsilon} \to f\) cuando \(\epsilon \to 0\) ctp;
	\item Si \(f\in \CC(\Omega)\), \(f_{\epsilon}\) converge uniformemente a
		\(f\) sobre compactos;
	\item Si \(f\in L^{p}_{\loc} (\Omega)\) con \(1 \le p \le \infty\), entonces
		\(f_{\epsilon} \to f\) en \(L^{p}_{\loc}\).
\end{enumerate}

\begin{Teorema}[Regularidad de funciones armónicas]
	Si \(u\in \CC(\Omega)\) y satisface la propiedad de la media para toda bola
	\(\overline{B(x,r)}\subset\Omega\), entonces \(u\) es \(\CC^{\infty}(\Omega)\).
\end{Teorema}
\begin{Demostracion}
	Vamos a probar que \(u = u_{\epsilon}\) en \(\Omega_{\epsilon}\).
	\begin{align*}
		u_{\epsilon}(x)
		&=
		\int_{\R^n} \eta_{\epsilon} (x-y) u(y) \, dy
		\\&=
		\frac{1}{\epsilon^n}
		\int_{B(x,\epsilon)}
			\eta\left(\frac{x-y}{\epsilon}\right) u(y) \, dy
		\\&=
		\frac{1}{\epsilon^n}
		\int_{0}^{\epsilon}
		\int_{\partial B(x, r)}
			\eta\left(\frac{x-y}{\epsilon}\right) 
			u(y) \, dS(y)
		\, dr
		\\&=
		\frac{1}{\epsilon^n}
		\int_{0}^{\epsilon}
		\eta\left(\frac{r}{\epsilon}\right) 
		\underbrace{
		\int_{\partial B(x, r)}
			u(y) \, dS(y)
		}_{u(x) \abs{\partial B(x,r)}}
		\, dr
		\\&=
		\frac{u(x)}{\epsilon^n}
		\int_{0}^{\epsilon}
			\eta\left(\frac{r}{\epsilon}\right)
			\abs{\partial B(x,r)}
		\, dr
		\\&=
		\frac{u(x)}{\epsilon^n}
		\int_{0}^{\epsilon}
		\int_{\partial B(x,r)}
			\eta\left(\frac{r}{\epsilon}\right)
			\, dS(y)
		\, dr
		\\&=
		\frac{u(x)}{\epsilon^n}
		\int_{B(x,\epsilon)}
			\eta\left(\frac{x-y}{\epsilon}\right)
			\, dy
		\\&=
		u(x)
		\underbrace{
		\int_{\R^n}
			\eta_{\epsilon} \left(x-y\right)
			\, dy
		}_{=1}
		=
		u(x).
	\end{align*}
	Como \(u_{\epsilon}\) es \(\CC^{\infty}(\Omega_{\epsilon})\), también lo es 
	\(u\).
\end{Demostracion}

\begin{Teorema}
	Las funciones armónicas son analíticas.
\end{Teorema}

\begin{Teorema}[Desigualdad de Harnack]
	Sea \(U\subset \R^n\) abierto y \(V\) un dominio compactamente contenido 
	en \(U\). Si \(u\) es una función armónica no negativa en \(U\), entonces 
	\begin{displaymath}
		\sup_{V} u \le C \inf_{V} u.
	\end{displaymath}
	donde la constante \(C\) no depende de \(u\). En particular, se tiene que
	\begin{displaymath}
		\frac{1}{C} u(y)
		\le
		u(x)
		\le
		C u(y)
		\quad \forall x, y\in V.
	\end{displaymath}
\end{Teorema}
\begin{Demostracion}
	Vamos a probar la última desigualdad. Sea \(r = \dist(V, \partial U)\). 
	Como
	\(\overline V\) es compacto si cubrimos con bolas de radio \(r/2\), 
	existe un cubrimiento \(B_1, \dots, B_N\)
	finito, tal que \(V \subset B_1 \cup \cdots \cup B_N\). 
	Además, como \(V\) es conexo,
	se tiene (renombrando de ser necesario) que \(B_{i-1} \cap B_{i} \ne
	\varnothing\) para todo \(i=2, \dots, N\). 

	Denotemos por \(x_1, \dots, x_N\) a los centros de \(B_1, \dots, B_N\)
	respectivamente. Como \(u\) es armónica, vale la propiedad de la media, 
	así que
	\begin{displaymath}
		u(x_i)
		=
		\fint_{B(x_i,r)} u(y) \, dy
		=
		\frac{1}{\abs{B(x_i,r)}}
		\int_{B(x_i,r)} u(y) \, dy
		\quad \forall i=1, \dots, N.
	\end{displaymath}
	Ahora, por la elección del cubrimiento, se tiene que
	\begin{alignat*}{2}
		u(x_{1})
		&=
		\frac{1}{\abs{B(x_1,r)}}
		\int_{B(x_{1},r)} u(y) \, dy
		\\&\ge
		\frac{1}{\abs{B(x_1,r)}}
		\int_{B(x_2, r/2)} u(y) \, dy
		=
		\frac{1}{2^n} 
		\fint_{B(x_2, r/2)} u(y) \, dy
		=
		\frac{1}{2^n} 
		u(x_2).
	\end{alignat*}
	Se sigue que \(u(x_{1}) \ge \frac{1}{2^{n(N-1)}} u(x_N)\). Si \(x, y\in V\),
	entonces SPG \(x \in B_{i}\) y \(y \in B_{j}\) para algún par \(i \le j\). Bajo el
	procedimiento anterior en \(x\) tenemos \(u(x) \ge \frac{1}{2^n} u(x_i)\) y
	aplicando el procedimiento en \(x_j\) da que \(u(x_j) \ge \frac{1}{2^n} u(y)\). 
	Luego,
	\begin{displaymath}
		u(x) 
		\ge
		\frac{1}{2^n} u(x_i)
		\ge
		\frac{1}{2^{nN}} u(x_j)
		\ge
		\frac{1}{2^{n(N+1)}} u(y).
	\end{displaymath}
	Las desigualdades en la otra dirección son análogas. Con esto queda
	demostrado el resultado.
\end{Demostracion}

\end{document}
