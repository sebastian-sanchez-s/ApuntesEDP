%! Tex Root = edp.tex
\documentclass[../edp.tex]{subfiles}

\begin{document}

{\scshape \hfill 27 de abril, 2023}

\subsection{Ecuación del Calor No Homogénea}

Ahora tratamos con el problema
\begin{displaymath}
\begin{caligned}\label{NH}
	u_{t} - \Delta u &= f, && \text{ en } \R^{+}\times\R,\\
	u &= g, && \text{ en } \left\{ 0 \right\}\times \R^n
\end{caligned}\tag{NH}
\end{displaymath}

Para resolver esto usaremos un método que es común en EDO. La idea
consiste en resolver el problema homogéneo y desde ahí construir una
solución del problema no homogéneo. A esta idea se le conoce como el
\textit{principio de Duhamel}.

Recordemos que en EDO, para resolver problema de la forma 
\begin{displaymath}
\begin{caligned}
	v_{t} + v &= f 
	&&, t > 0\\
	v &= g 
	&&, t = 0,
\end{caligned}
\end{displaymath}
podíamos aplicar la técnica del propagador como se muestra a
continuación
\begin{align*}
	v'(t) \, e^{t}  + v(t) \, e^{t} &= f(t)\, e^{t} \\
	(v(t) \, e^{t})' &= f(t)\, e^{t} \\
	\Rightarrow
	v &= e^{-t}\, g(t) + \int f(s)\, e^{s-t} \, ds.
\end{align*}
donde el propagador lo podemos interpretar como el operador \(S(t)[F]
= e^{-t}\, F\). De esta forma, podemos ver que \(v\) se compone de la
solución del problema homogéneo (\(S(t)[g]\)) y una solución
particular del problema (\(\int S(t-s)[f]\)) no homogéneo. 

Volviendo al problema~\eqref{NH}, si las soluciones del problema
homogéneo son de la forma \(u_{h}(t,x) = S(t)[g(x)]\) para algún
propagador. ¿Será que \(u_{h}(t,x) + \int S(t-s)[f(s,x)]\) es solución
del problema no homogéneo?. Simbólicamente, tendríamos que
\begin{align*}
	u_t - \Delta_{x} u 
	&=
	[\partial_{t} - \Delta_{x}] S(t)[g(x)]
	+
	[\partial_{t} - \Delta_{x}] \int_{0}^{t} S(t-s)[f(s,x)]\, ds
	\\&=
	0 + S(t-t)[f(t, x)] + 
	\int_{0}^{t} [\partial_t - \Delta_x] S(t-s)[f(s,x)]\, ds
	\\&=
	f(t,x)
\end{align*}
Ahora, recordemos que 
\begin{displaymath}
	u_h(t,x)
	=
	\int_{\R^n} \Phi(t, x-y) g(y) \, dy
\end{displaymath}
y esto se ve exactamente como un propagador, así que un candidato a
solución particular es
\begin{displaymath}
	u_{p}(t,x)
	=
	\int_{0}^{t}
	\int_{\R^n} \Phi(t-s, x-y) f(s,y) \, dy\, ds.
\end{displaymath}
El siguiente teorema dice que, en efecto, este procedimiento da una
solución de~\eqref{NH}.

\begin{Teorema}
	Sea \(f\in \CC^{(1,2)}(\R_{\ge 0} \times \R^n)\) con soporte
	compacto. Definamos \(u = u_{p}\) como arriba. Entonces,
	\begin{enumerate}[topsep=5pt,itemsep=2pt]
		\item \(u\in \CC^{(1,2)}(\R_{+}\times\R^n)\),
		\item \(u_{t} - \Delta_{x} u = f(t,x)\) para todo \((t,x) \in
		\R_{+}\times \R^n\),
		\item Para cualquier \(x_0 \in \R^n\), se tiene que
		\begin{displaymath}
			\lim_{x\to x_0,\, t\downarrow 0} u(t,x) = 0.
		\end{displaymath}
	\end{enumerate}
\end{Teorema}
\begin{Demostracion}
\end{Demostracion}

\end{document}
