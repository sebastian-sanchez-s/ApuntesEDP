%! Tex Root = edp.tex
\documentclass[../edp.tex]{subfiles}

\begin{document}

{\scshape \hfill 23 de mayo, 2023}

Fijemos \(1\le p \le \infty\) y \(k\in \Z^{+}\). 
\begin{Definicion}[Espacio de Sobolev]
	El espacio de Sobolev \(W^{k,p}(\Omega)\) consiste de todas las
	funciones escalares \(k\)-veces debilmente diferenciables con derivadas
	\(p\)-integrables. En símbolos,
	\begin{displaymath}
		W^{k,p}(\Omega)
		\coloneqq
		\left\{ 
			u\colon \Omega \to \R
			\text{ tal que }
			\forall \abs{\alpha} \le k\quad
			D^{\alpha} u \text{ existe } \land\,
			D^{\alpha} u \in L^{p}(\Omega)
	   	\right\}
	\end{displaymath}
\end{Definicion}

Denotamos \(H^{k} = W^{k,2}\). 

\begin{Definicion}[Norma en \(W^{k,p}\)]
	Si \(u\in W^{k,p}(\Omega)\), definimos su norma por
	\begin{displaymath}
		\norm{u}_{W^{k,p}(\Omega)}
		=
		\begin{cases}
			\left( 
				\sum_{\abs{\alpha} \le k}\, 
				\norm{D^{\alpha} u}_{L^{p}(\Omega)}^{p}
			\right)^{1/p}
			& \text{ Si } 1\le p < \infty
			\\
			\sum_{\abs{\alpha}\le k}\,
				\norm{D^{\alpha} u}_{L^{\infty}(\Omega)}
			& \text{ Si } p = \infty.
		\end{cases}
	\end{displaymath}
\end{Definicion}

La convergencia la definimos de manera usual. Es decir, una sucesión
de funciones \((u_{m})_m\) en \(W^{k,p}(\Omega)\) converge a una
función \(u\in W^{k,p}(\Omega)\) si \(\norm{u_m - u}_{W^{k,p}(\Omega)} \to 0\).

De manera análoga a los espacios \(L^{p}_{\loc}\), definimos
\(W^{k,p}_{\loc}\) y decimos que una sucesión converge si lo hace en
compactos.

\begin{Teorema}
	Los espacios de Sobolev son espacios de Banach.
\end{Teorema}

\begin{Ejemplo}
\begin{enumerate}[itemsep=2pt,topsep=3pt]
	\item 
	Sea \(\Omega = B(0,1) \subset \R^n\) y \(\alpha > 0\).
	Estudiaremos las derivadas débiles de 
	\begin{displaymath}
		u(x) = \frac{1}{\abs{x}^{\alpha}}
		\qquad x\ne 0
	\end{displaymath}
	y deduciremos para qué valores de \(\alpha\) está en algún espacio
	de Sobolev.

	Dado que
	\begin{displaymath}
		\partial_{x_i} u(x)
		=
		- \alpha \abs{x}^{-\alpha - 1} \frac{x_i}{\abs{x}}
		=
		-\alpha \frac{x_i}{\abs{x}^{\alpha+2}}.
	\end{displaymath}
	Se tiene que
	\begin{displaymath}
		D u =
		\begin{bmatrix}
			\partial_{x_1} u &\cdots &\partial_{x_n} u
		\end{bmatrix}
		\implies
		\abs{Du} = \frac{\abs{\alpha}}{\abs{x}^{\alpha+1}}.
	\end{displaymath}
	Se sigue que \(\abs{Du}\) está en \(L^{1}(\Omega)\) si
	\(\alpha + 1 < n\).

	Consideremos \(\phi \in \CC^{\infty}_{C}(\Omega)\) y \(\epsilon >
	0\). Notese que
	\begin{displaymath}
		\int_{\Omega \setminus \overline{B(0,\epsilon)}}
			u \phi_{x_i}
		=
		- \int_{\Omega\setminus\overline{B(0,\epsilon)}}
			u_{x_i} \phi
		+
		\int_{\partial B(0,\epsilon)}
			u \phi \n_{i}.
	\end{displaymath}
	Dado que  
	\begin{align*}
		\abs{
		\int_{\partial B(0,\epsilon)}
			u \phi \n_{i}
		} 
		&\le
		\norm{\phi}_{L^{\infty}(\Omega)}
		\int_{\partial B(0,\epsilon)}
			\epsilon^{-\alpha}
		\\&\le
		\norm{\phi}_{L^{\infty}(\Omega)}
		n \alpha(n) \epsilon^{n-1}
		\epsilon^{-\alpha}
		\to 0
	\end{align*}
	se tiene que (cuando \(\epsilon \to 0\))
	\begin{displaymath}
		\int_{\Omega} u \phi_{x_i}
		=
		- \int_{\Omega} u_{x_i} \phi
	\end{displaymath}
	y por lo tanto \(u\in W^{1,1}(\Omega)\) si \(\alpha+1 < n\).

	En general, si \((\alpha+1)p < n\) se tiene que \(\abs{Du} \in
	L^{p}(\Omega)\) y \(u\in W^{1,p}(\Omega)\).

	\item 
	Sea \(\Omega = B(0,1) \subset \R^n\) y \(\left\{ r_k \right\}_{k\in\N}\)
	un subconjunto denso y numerable de \(\Omega\). La función
	definida por
	\begin{displaymath}
		u(x) = 
		\sum_{k\ge 1} 
			\frac{1}{2^{k}} 
			\frac{1}{\abs{x-r_k}^{\alpha}}
	\end{displaymath}
	está en \(W^{1,p}(\Omega)\) para \((\alpha+1)p < n\) (usando el
	argumento del ejemplo anterior). Sin embargo, \(u\) no
	es acotada en todo abierto de \(\Omega\) y por lo tanto \(u\not\in
	L^{1}_{\loc}(\Omega)\).
\end{enumerate}
\end{Ejemplo}

\begin{Proposicion}[Propiedades de Derivadas Débiles]
	En los siguientes incisos se entiende que
	\(u,v\in W^{k,p}(\Omega)\) y que \(\abs{\alpha} \le k\). 
	\begin{enumerate}[itemsep=2pt,topsep=3pt]
		\item
		Sean \(\beta\) un multi-índice tal que \(\abs{\alpha} +
		\abs{\beta} \le k\). Entonces
		\begin{displaymath}
			D^{\alpha} u \in W^{k-\abs{\alpha}, p}(\Omega)
			\quad\land\quad
			D^{\beta} (D^{\alpha} u)
			=
			D^{\alpha} (D^{\beta} u)
			=
			D^{\alpha + \beta} u.
		\end{displaymath}

		\item
		Si \(\lambda, \mu\in \R\), se tiene que
		\begin{displaymath}
			\lambda u + \mu v \in W^{k,p}(\Omega)
			\quad\land\quad
			D^{\alpha}(\lambda u + \mu v)
			=
			\lambda D^{\alpha} u + \mu D^{\alpha} v
		\end{displaymath}

		\item
		Si \(\Theta\subset \Omega\) es abierto, entonces \(u\in
		W^{k,p}(\Theta)\).

		\item
		Si \(\zeta \in \CC^{\infty}_{C}(\Omega)\), entonces \(\zeta u
		\in W^{k,p}(\Omega)\) y
		\begin{displaymath}
			D^{\alpha}(\zeta u)
			=
			\sum_{\abs{\beta} \le \abs{\alpha}}
				{\alpha \choose \beta}
				D^{\beta} \zeta
				\,
				D^{\alpha - \beta} u
		\end{displaymath}
	\end{enumerate}
\end{Proposicion}

Denotamos por \(W_{0}^{k,p}(\Omega)\) a la clausura de
\(\CC^{\infty}_{C}(\Omega)\) en \(W^{k,p}(\Omega)\). De esta forma,
toda función en \(W_{0}^{k,p}(\Omega)\) se puede aproximar por
funciones continuas y con soporte compacto. En símbolos:
\begin{displaymath}
	u\in W_{0}^{k,p}(\Omega)
	\iff
	(\exists\, (u_m)_{m\in\N}\in \CC^{\infty}_{C}(\Omega))
	\mid
	u_m \to u \in W^{k,p}(\Omega).
\end{displaymath}
Intuitivamente, las funciones en \(W_{0}^{k,p}(\Omega)\) son tales que
``\(D^{\alpha} u = 0\) en \(\partial\Omega\)" para \(\abs{\alpha} \le
k-1\).

\subsection{Teoremas de Aproximación}

\begin{Teorema}[Aproximación Interior por Funciones Suaves]
	Sea \(u\in W^{k,p}(\Omega)\) con \(1 \le p < \infty\). La función
	\begin{displaymath}
		u_{\epsilon} = \eta_{\epsilon}\ast u
	\end{displaymath}
	sobre \(\Omega_{\epsilon} = \left\{ x\in\Omega\mid
	\textrm{dist}(x,\partial\Omega) > \epsilon \right\}\) satisface:
	\begin{enumerate}[leftmargin=.3\textwidth, topsep=5pt, itemsep=2pt]
		\item
		\(u_{\epsilon} \in \CC^{\infty}(\Omega_{\epsilon)}\) para todo
		\(\epsilon > 0\).

		\item
		\(u_{\epsilon} \to u\) en \(W^{k,p}_{\loc}(\Omega)\) cuando
		\(\epsilon \to 0\).
	\end{enumerate}
\end{Teorema}

\begin{Teorema}[Aproximación Global por Funciones Suaves en \(\Omega\)]
	Supongamos que \(\Omega\) es acotado. Sea \(u\in W^{k,p}(\Omega)\)
	con \(1 \le p < \infty\). Entonces existe una sucesión de
	funciones \((u_m)_{m\in\N} \in \CC^{\infty}(\Omega)\cap
	W^{k,p}(\Omega)\) tal que \(u_m \to u \in W^{k,p}(\Omega)\).
\end{Teorema}

\begin{Teorema}[Aproximación Global por Funciones Suaves en \(\overline{\Omega}\)]
	Supongamos que \(\Omega\) es acotado y \(\partial\Omega\) es de
	clase \(\CC^{1}\). Sea \(u\in W^{k,p}(\Omega)\) con \(1\le p <
	\infty\). Existe una sucesión \((u_m)_{m\in\N} \in
	\CC^{\infty}(\overline\Omega)\) tal que \(u_m \to u\) en
	\(W^{1,p}(\Omega)\).
\end{Teorema}

\end{document}
