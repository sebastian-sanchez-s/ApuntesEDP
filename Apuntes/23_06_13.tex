%! Tex Root = edp.tex
\documentclass[../edp.tex]{subfiles}

\begin{document}

{\scshape \hfill 13 de junio, 2023}

\section{Operadores Elípticos de Segundo Orden}

Comenzamos a estudiar el problema con condición de borde
\begin{equation}\label{1306:eq1}
\begin{caligned}
	Lu &= f &&, \text{ en } \Omega\\
	u &= 0 &&, \text{ en }\partial\Omega
\end{caligned}
\end{equation}
donde \(\Omega\subset\R^n\) es abierto y acotado, \(u\colon
\overline{\Omega} \to \R\) es  la función incógnita, \(f\colon \Omega
\to \R\) es dada y \(L\) denota un operador diferencial de segundo
orden que tiene una de las siguientes formas:
\begin{align}
	\label{1306:FD}
	Lu 
	&= 
	- \sum_{i,j}^{n} (a^{ij}(x) u_{x_i})_{x_j} 
	+ \sum_{i=1}^{n} b^{i}(x) u_{x_i}
	+ c(x) u
	\\
	\label{1306:FND}
	Lu
	&=
	- \sum_{i,j}^{n} a^{ij}(x) u_{x_i\, x_j}
	+ \sum_{i=1}^{n} b^{i}(x) u_{x_i}
	+ c(x) u
\end{align}
Si \(L\) es de la forma~\eqref{1306:FD}, diremos que está en forma de
divergencia. En el otro caso~\eqref{1306:FND} diremos que está en
forma de no-divergencia. En ambos casos, las funciones coeficientes
(\(a^{ij}, b^{i}, c\)) satisfacen que:
\begin{enumerate}
	\item
	están en \(L^{\infty}(\Omega)\).

	\item
	\(a^{ij} = a^{ji}\).

	\item
	existe \(\Theta > 0\) tal que
	\begin{displaymath}
		\sum_{i,j}^{n} a^{ij}(x) \xi_{i} \xi_{j}
		\ge
		\Theta \abs{\xi}^{2}
		\quad
		\text{c.t.p. en } \Omega,
		\forall \xi\in\R^n
	\end{displaymath}
\end{enumerate}
Si \(L\) satisface las propiedades listadas arribas, diremos que es
uniformemente elíptico.

\subsection{Soluciones Débiles}

Vamos a estudiar el problema~\eqref{1306:eq1} en el caso que \(L\)
está en forma de divergencia~\eqref{1306:FD} y \(f\in L^{2}(\Omega)\).
Momentáneamente supongamos que \(u\in\CC^{\infty}_{C}(\Omega)\). Al
multiplicar por una función test e integrar por partes tendremos que
\begin{equation}\label{1306:eq2}
	\sum_{i,j}^{n}
	\int_{\Omega} a^{ij} u_{x_i} v_{x_j}
	+
	\sum_{i=1}^{n}
	\int_{\Omega} b^{i} u_{x_i} v
	+
	\int_{\Omega} cvu
	=
	\int_{\Omega} fv
	\quad\forall v\in \CC^{\infty}_{C}(\Omega).
\end{equation} 
Más aún, el resultado sigue valiendo para \(v\in H^{1}_{0}(\Omega)\),
pues podemos aproximar por funciones suaves con soporte compacto.

Por otro lado, la ecuación~\eqref{1306:eq2} hace sentido incluso para
\(u\in H^{1}(\Omega)\). No obstante, nos restringiremos a trabajar
sobre \(H^{1}_{0}(\Omega)\) pues necesitamos que se satisfaga la
condición de borde.

\begin{Definicion}[Forma Bilineal Asociada a \(L\)]
	La forma bilineal (lineal en ambas entradas) asociada al operador
	lineal diferencial \(L\) en forma de divergencia~\eqref{1306:FD} 
	está definida por
	\begin{displaymath}\label{1306:starB}
		B(u,v) 
		\coloneqq
		\sum_{i,j}^{n}
		\int_{\Omega} a^{ij} u_{x_i} v_{x_j}
		+
		\sum_{i=1}^{n}
		\int_{\Omega} b^{i} u_{x_i} v
		+
		\int_{\Omega} cvu,
		\quad u,v \in H^{1}_{0}(\Omega).
		\tag{\(\star_B\)}
	\end{displaymath}
\end{Definicion}

\begin{Definicion}[Solución Débil, \(f\in L^{2}(\Omega)\)]
	Diremos que \(u\in H^{1}_{0}(\Omega)\) es una solución débil
	de~\eqref{1306:eq1} para \(f\in L^{2}(\Omega)\) si 
	\begin{displaymath}
		B(u,v) = \langle f,v \rangle_{L^{2}(\Omega)}
		\quad\forall v \in H^{1}_{0}(\Omega),
	\end{displaymath}
	donde el lado derecho es el producto interno en \(L^{2}(\Omega)\).
	Esto se conoce como la formulación variacional
	de~\eqref{1306:eq1}.
\end{Definicion}

\begin{Definicion}[Solución Débil, \(f\in H^{-1}(\Omega)\)]
	Diremos que \(u\in H^{1}_{0}(\Omega)\) es una solución débil
	de~\eqref{1306:eq1} para \(f\in H^{-1}(\Omega)\) si 
	\begin{displaymath}
		B(u,v) = 
		\langle f,v \rangle_{H^{-1}(\Omega)\times H^{1}_{0}(\Omega)}
		\quad\forall v \in H^{1}_{0}(\Omega),
	\end{displaymath}
	donde el lado derecho es el producto dualidad entre \(H^{-1}\times
	H^{1}_{0}\).
\end{Definicion}

El concepto de solución débil para condiciones de borde no nulas se
puede transformar en uno con condiciones de borde nulas de la
siguiente manera: supongamos que tenemos el problema
\begin{displaymath}
\begin{caligned}
	Lu &= f &&, \Omega\\
	u &= g &&, \partial\Omega,
\end{caligned}
\end{displaymath}
donde \(f\in H^{-1}(\Omega)\), \(g \in T(H^{1}(\Omega))\) y \(\Omega\)
tiene ahora frontera \(\CC^{1}\). Poniendo \(u' = u - w\), donde
\(T(w) = g\) reduce el problema a 
\begin{displaymath}
\begin{caligned}
	Lu' &= f' &&, \Omega\\
	u' &= 0 &&, \partial\Omega,
\end{caligned}
\end{displaymath}
donde \(f' = f - Lw \in H^{-1}(\Omega)\).

\subsection{Existencia de Soluciones Débiles}

En esta parte \(H\) denotará un espacio de Hilbert, con producto
interno \(\langle \cdot, \cdot \rangle\) y norma \(\norm{\cdot}\).
\(H'\) denota el dual de \(H\) y \(\langle \cdot,\cdot \rangle_{\times}\)
denota el producto dual entre \(H'\) y \(H\).

\begin{Teorema}[Lax-Milgram]
	Sea \(B\colon H\times H \to \R\) una forma bilineal tal que
	existen constantes \(\alpha,\beta > 0\) que satisfacen
	\begin{alignat*}{2}
		\label{1306:B1}
		\abs{B(u,v)} &\le \alpha \norm{u} \norm{v}
		&&\quad\forall u,v\in H
		\tag{B.1}
		\\
		\label{1306:B2}
		\beta \norm{u}^{2} &\le B(u,u)
		&&\quad\forall u\in H
		\tag{B.2}
	\end{alignat*}
	entonces para \(f\in H'\) existe un único \(u\in H\) tal que
	\begin{displaymath}
		B(u,v) = \langle f,v \rangle_{\times}.
	\end{displaymath}

	La condición~\eqref{1306:B1} hace referencia a la continuidad del
	operador \(B\), mientras que la condición~\eqref{1306:B2} se le
	conoce como coercividad.
\end{Teorema}
\begin{Demostracion}
	La demostración la haremos por pasos.

	\framebox{Paso 1:} 
	Fijemos \(u\in H\). Luego, el mapa \(v\mapsto
	B(u,v)\) es continuo por~\eqref{1306:B1}. Así, el teorema de
	representación de Riesz implica la existencia de un único \(w =
	w(u) \in H\) tal que 
	\begin{displaymath}
		B(u,v) = \langle w, v \rangle
		\quad\forall v \in H.
	\end{displaymath}
	Denotemos
	por \(A\) al operador que a cada \(u\in H\) le asigna \(w(u) \in
	H\). Nótese que 
	\begin{displaymath}
		B(u,v) = \langle Au, v \rangle
		\quad\forall v\in H.
	\end{displaymath}

	\framebox{Paso 2:}
	Primero veamos \(A\) efectivamente define un operador lineal. 
	Para ver que \(A0 = 0\), notemos \(B(0,v) = B(\lambda \cdot 0, v)
	= \lambda B(0,v)\) para cualquier \(\lambda\in \R^{\times}\). En
	consecuencia, se sigue que \(B(0,v) = 0 = \langle 0, v \rangle\). Por
	lo tanto \(A0 = 0\). Sean ahora \(u, u'\in H\) y \(\lambda, \mu
	\in \R^{\times}\). Entonces
	\begin{align*}
		B(\lambda u + \mu u', v)
		&=
		\lambda B(u,v) + \mu B(u', v)	
		\\&=
		\lambda \langle w(u),v \rangle + \mu \langle w(u'),v \rangle
		\\&=
		\langle \lambda w(u) + \mu w(u'), v \rangle
		\\&=
		\langle w(\lambda u + \mu u'), v \rangle.
	\end{align*}
	Por lo tanto, \(A(u+u') = w(\lambda u + \mu u') = \lambda u + \mu
	u'\).

	Por otro lado, comprobamos que \(A\) es acotado (y por lo tanto
	continuo)
	\begin{displaymath}
		\langle Au, Au \rangle
		=
		\norm{Au}^{2}
		=
		B(u, Au)
		\overset{\eqref{1306:B1}}{\le}
		\alpha \norm{u} \norm{Au}
		\Rightarrow
		\norm{Au} \le \alpha \norm{u}.
	\end{displaymath}

	\framebox{Paso 3:}
	Usando~\eqref{1306:B2} y Cauchy-Schwarz nos da que
	\begin{displaymath}
		\beta \norm{u}^{2}
		\le
		B(u,u)
		=
		\langle Au, u \rangle
		\le
		\norm{Au}\, \norm{u}
		\Rightarrow
		\beta \norm{u} \le \norm{Au}.
	\end{displaymath}
	Se sigue que \(A\) es inyectiva y tiene rango cerrado. En efecto,
	como \(A0 = 0\) tenemos que \(\norm{A0} \le \norm{u} \beta\) y por
	lo tanto \(u = 0\) (con esto tenemos inyectividad, y en particular
	podemos hablar de la inversa). Por otro lado, si \((u_k)_{k\in\N}\) define 
	una sucesión en \(T(H) \subset H\) convergente a \(u\in H\) tenemos que
	\begin{displaymath}
		\beta \norm{A^{-1}(u_k - u_m)} 
		\le 
		\norm{u_k - u_m} \to 0.
	\end{displaymath}
	Así que \((A^{-1}u_k)_{k\in\N}\) es una sucesión de Cauchy en \(H\) y
	por lo tanto converge a \(A^{-1} u\). En particular \(u \in
	A(H)\).

	\framebox{Paso 4:}
	Ahora probaremos que \(A(H) = H\). Supongamos buscando una 
	contradicción que \(H\setminus A(H) \ne \varnothing\). Dado que
	\(A(H)\) es cerrado, tiene un complemento ortogonal y \(H = A(H)
	\oplus A^{\perp}(H)\). Sea entonces \(w\in A^{\perp}(H)\),
	entonces \(\langle Au, w \rangle = 0\) para cualquier \(u\in H\). 
	Por~\eqref{1306:B2} tenemos que
	\begin{displaymath}
		\beta \norm{w}^{2} \le B(w,w) = \langle Aw, w \rangle = 0.
	\end{displaymath}
	Se sigue que \(w = 0\). Pero \(w\in A(H)\). Contradicción.

	\framebox{Paso 5:} Recordamos que por el Teorema de Representación
	de Riesz, que si \(f\in H'\) existe un único \(w_0\in H\) tal que
	\(\langle f,v \rangle_{\times} = \langle w_0, v \rangle\) para
	todo \(v\in H\).

	Ahora bien, tenemos que \(A\) es un operador lineal biyectivo (un
	endomorfismo), así que existe \(u_0 \in H\) tal que \(Au_0 =
	w_0\). Se sigue que
	\begin{displaymath}
		B(u_0, v) 
		= \langle Au_0, v \rangle
		= \langle w_0, v \rangle
		= \langle f, v \rangle_{\times}.
	\end{displaymath}

	\framebox{Paso 6:} En los pasos anteriores mostramos la existencia
	de la representación. Queda ver que es única.
\end{Demostracion}

Obsérvese que si \(B\) es una forma bilineal simétrica (i.e. \(B(u,v)
= B(v,u)\)), entonces define un producto interno. Por el teorema de
representación de Riesz, el resultado del teorema anterior se sigue de
manera directa.

\begin{Proposicion}
	Existen constantes \(\alpha,\beta,\gamma \le 0\) tal que
	la forma bilineal en~\eqref{1306:starB} satisface:
	\begin{alignat}{6}
		\label{1306:starB:B1}
		\abs{B(u,v)}
		&\le
		\alpha 
		\norm{u}_{H^{1}_{0}(\Omega)}
		\norm{v}_{H^{1}_{0}(\Omega)}
		&&\quad \forall u,v\in H^{1}_{0}(\Omega).
		\tag{\(\star_{B}\).B1}
		\\
		\label{1306:starB:B2}
		\beta \norm{u}^{2}_{H^{1}_{0}(\Omega)}
		&\le
		B(u,u) + \gamma \norm{u}^{2}_{L^{2}(\Omega)}
		&&\quad \forall u \in H^{1}_{0}(\Omega).
		\tag{\(\star_{B}\).B2}
	\end{alignat}
\end{Proposicion}
\begin{Demostracion}
	Para ver la condición de continuidad~\eqref{1306:starB:B1}
	basta acotar todo brutalmente
	\begin{align*}
		\abs{B(u,v)}
		&\le
		\sum_{i,j}^{n} 
			\norm{a^{ij}}_{L^{\infty}(\Omega)} 
			\int_{\Omega} \abs{u_{x_i} v_{x_j}}
		+
		\sum_{i=1}^{n}
			\norm{b^{i}}_{L^{\infty}(\Omega)}
			\int_{\Omega} \abs{u_{x_j} v}
		+
		\norm{c}_{L^{\infty}(\Omega)} 
		\int_{\Omega} \abs{v}\abs{u}
		\\&\le
		\sum_{i,j}^{n} 
			\norm{a^{ij}}_{L^{\infty}(\Omega)} 
			\int_{\Omega} \abs{Du} \abs{D v}
		+
		\sum_{i=1}^{n}
			\norm{b^{i}}_{L^{\infty}(\Omega)}
			\int_{\Omega} \abs{Du} \abs{v}
		+
		\norm{c}_{L^{\infty}(\Omega)} 
		\int_{\Omega} \abs{v}\abs{u}
		\\&\le
		\alpha 
		\norm{u}_{H^{1}_{0}(\Omega)}
		\norm{v}_{H^{1}_{0}(\Omega)}.
	\end{align*}
	donde \(\alpha\) tiene que ser mayor o igual a 
	\begin{displaymath}
		\max
		\left(
		\sum_{i,j}^{n} \norm{a^{ij}}_{L^{\infty}(\Omega)}
		,
		\sum_{i=1}^{n} \norm{b^{i}}_{L^{\infty}(\Omega)}
		,
		\norm{c}_{L^{\infty}(\Omega)}
		\right).
	\end{displaymath}

	Ahora vamos a mostrar la pseudo-coercividad~\eqref{1306:starB:B2}. Usando que
	el operador es uniformemente elíptico tenemos que existe \(\Theta >
	0\) tal que
	\begin{displaymath}
		\Theta \abs{\nabla u}^{2}
		\le
		\sum_{i,j}^{n} a^{ij}(x) u_{x_i} u_{x_j}
	\end{displaymath}
	integrando tenemos que
	\begin{align*}
		\Theta \int_{\Omega} \abs{\nabla u}^{2}
		&\le	
		\sum_{i,j}^{n} \int_{\Omega} a^{ij}(x) u_{x_i} u_{x_j}
		\\&=
		B(u,u) 
		- \sum_{i=1}^{n} \int_{\Omega} b^{i}(x) u_{x_i} u
		- \int_{\Omega} c \abs{u}^{2}
		\\&\le
		B(u,u)
		+ 
		\sum_{i=1}^{n} 
			\norm{b^i}_{L^{\infty}(\Omega)}
			\int_{\Omega} \abs{\nabla u} \abs{u}
		+
		\norm{c}_{L^{\infty}(\Omega)}
		\int_{\Omega} \abs{u}^2.
	\end{align*}

	Por otro lado, recordando que \(ab \le \frac{a^2}{2} +
	\frac{b^2}{2}\) tenemos que
	\(ab \le \epsilon \frac{a^2}{2} + \frac{b^2}{4\epsilon}\) para
	cualquier \(\epsilon > 0\). Usando la
	desigualdad de Cauchy-Schwarz con esta idea nos da que
	\begin{displaymath}
		\int_{\Omega} \abs{\nabla u} \abs{u}
		\le
		\epsilon \int_{\Omega} \abs{\nabla u}^2
		+
		\frac{1}{4\epsilon}
		\int_{\Omega} \abs{u}^{2}
		\quad\forall \epsilon > 0.
	\end{displaymath}
	En particular, podemos tomar \(\epsilon\) suficientemente pequeño
	para que 
	\begin{displaymath}
		\epsilon \sum_{i=1}^{n} \norm{b^{i}}_{L^{\infty}(\Omega)}
		<
		\frac{\Theta}{2},
	\end{displaymath}
	de esta forma se cumple que
	\begin{displaymath}
		\frac{\Theta}{2}
		\int_{\Omega} \abs{\nabla u}^{2}
		\le
		B(u,u) + C \int_{\Omega} \abs{u}^2,
	\end{displaymath}
	donde \(C \ge \max(\norm{c}_{L^{\infty}(\Omega)}, 1/4\epsilon)\).
	Usando la desigualdad de Poincaré en \(H^{1}_{0}(\Omega)\): 
	\begin{displaymath}
		\norm{u}_{H^{1}_{0}(\Omega)}
		\le
		C'
		\norm{\nabla u}_{L^{2}(\Omega)},
	\end{displaymath}
	concluimos que
	\begin{displaymath}
		\frac{\Theta}{2C'} \norm{u}_{H^{1}_{0}(\Omega)}
		\le
		B(u,u) + C \norm{u}^{2}_{L^{2}(\Omega)}.
	\end{displaymath}
\end{Demostracion}

\begin{Teorema}[Primer Teorema de Existencia para Soluciones Débiles]
	Hay un número \(\gamma \ge 0\) tal que \(\forall \mu \ge \gamma\) y
	\(\forall f\in L^{2}(\Omega)\) existe una solución débil \(u\in
	H^{1}_{0}(\Omega)\) del problema
	\begin{displaymath}
	\begin{caligned}
		Lu + \mu u &= f &&, \text{ en } \Omega\\
		u &= 0 &&, \text{ en } \partial\Omega.
	\end{caligned}
	\end{displaymath}
\end{Teorema}
\begin{Demostracion}
\end{Demostracion}
\end{document}
