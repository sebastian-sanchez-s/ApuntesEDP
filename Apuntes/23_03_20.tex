%! Tex Root = edp.tex
\documentclass[../edp.tex]{subfiles}

\begin{document}

{\scshape \hfill 20 de Marzo, 2023}

\begin{Teorema}
	Para \(f\in \CC^{2}_{C}(\R^n)\), definimos su convolución como
	\begin{displaymath}
		\Phi \ast f (x)
		\coloneqq
		\int_{\R^n} \Phi(x-y) f(y) \, dy,
	\end{displaymath}
	donde \(\Phi\) es la solución fundamental de la ecuación de Laplace.
	Entonces,
	\begin{enumerate}[topsep=0pt,itemsep=0pt]
		\item \(\Phi \ast f \in \CC^2(\R^n)\) y 
		\item \(-\Delta (\Phi \ast f) = f\).
	\end{enumerate}
\end{Teorema}
\begin{Demostracion}
	Pongamos \(u = \Phi \ast f\). Haciendo un cambio de variables tenemos que
	\begin{displaymath}
		u(x) 
		=
		\int_{\R^n} \Phi(y) f(x-y) \, dy.
	\end{displaymath}
	Luego, 
	\begin{displaymath}
		\partial_{x_i} u(x)
		=
		\lim_{h\to 0}
		\int_{\R^n} 
		\Phi(y)
		\left[
			\frac{
				f(x - y + h e_i) - f(x - y)
			}{h} 
		\right]
		\, dy.
	\end{displaymath}
	Como \(f \in \CC^{2}_{C}\), se cumplen las hipótesis del Teorema de
	Convergencia Dominada (TCD). Explícitamente,
	\begin{displaymath}
		f_n(x) = 
			\frac{
				f(x + (1/n) e_i) - f(x)
			}{1/n} 
	\end{displaymath}
	converge puntualmente a \(\partial_{x_i} f(x)\) que es continua.
	Además, \(f_n(x)\) es acotada y tiene soporte compacto, así que
	\(\abs{f_n(x)} < C 1_{K}(x)\) donde \(K\) es un compacto 
	que contiene al soporte de \(f_n(x)\), \(1_{K}\) es la función indicatriz de
	\(K\) y \(C\) una constante positiva. 
	Como \(C 1_{K}(x)\) es integrable y domina a la sucesión, sigue que,
	\begin{displaymath}
		\partial_{x_i} u(x)
		=
		\int_{\R^n} 
			\Phi(y)
			\partial_{x_i} f(x-y)
			\, dy.
	\end{displaymath}
	Análogamente obtenemos que
	\begin{displaymath}
		\partial_{x_i}^2 u(x)
		=
		\int_{\R^n} 
			\Phi(y)
			\partial_{x_i}^2 f(x-y)
			\, dy.
	\end{displaymath}
	El lado derecho es continuo para cada \(i=1,\dots,n\), así que \(u\in
	\CC^{2}\). Más aún, \(\Delta u(x)\) se ve como es una distribución de
	\(\Phi\) y probamos anteriormente que actúa como \(\delta_{0}\), así que,
	\begin{displaymath}
		\Delta u(x)
		=
		\int_{\R^{n}}
		\Phi(y)
		\Delta_{x} f(x - y)
		\, dy
		=
		- f(x).
	\end{displaymath}
\end{Demostracion}

\subsection{Propiedades de Funciones Armónicas}

Para una función \(u\) en \(B(x,r)\) definimos 
\begin{displaymath}
	\fint_{B(x,r)}
		u(y) \, dy
	\coloneqq
	\frac{1}{\abs{B(x,r)}} 
	\int_{B(x,r)}
		u(y) \, dy
	=
	\frac{1}{\alpha(n) r^n}
	\int_{B(x,r)}
		u(y) \, dy.
\end{displaymath}
De manera análoga, si \(u\) está definida sobre \(\partial B(x,r)\) definimos
\begin{displaymath}
	\fint_{\partial B(x,r)}
		u(y) \, dS(y)
	\coloneqq
	\frac{1}{\abs{\partial B(x,r)}} 
	\int_{\partial B(x,r)}
		u(y) \, dS(y)
	=
	\frac{1}{n \alpha(n) r^{n-1}}
	\int_{\partial B(x,r)}
		u(y) \, dS(y).
\end{displaymath}

\begin{Teorema}[Fórmula de la Media]
	Sea \(\Omega\) un abierto en \(\R^n\). Si \(u\in \CC^{2}(\Omega)\) es
	armónica, entonces para toda bola \(\overline{B(x,r)} \subset \Omega\) se
	tiene que
	\begin{displaymath}
		u(x)
		=
		\fint_{\partial B(x,r)}
			u(y) \, dS(y)
		=
		\fint_{B(x,r)}
			u(y) \, dy.
	\end{displaymath}
	Además, si pedimos que \(u\) también sea continua 
	en la clausura de \(\Omega\) podemos tomar
	bolas abiertas \(B(x,r) \subset\Omega\).
\end{Teorema}
\begin{Demostracion}
	Sea \(x\in \Omega\).
	Para \(r > 0\) denotamos por \(B_r\) a la bola centrada
	en \(x\) de radio \(r\). Definimos
	\begin{displaymath}
		\phi(r)
		=
		\begin{cases}
			\fint_{\partial B_{r}} 
				u(y) \, dS(y) 
			&, r \ne 0\\
			u(x) 
			&, r = 0.
		\end{cases}
	\end{displaymath}
	Nótese que \(\phi\) es continua. Haremos un cambio de variables para que la
	integral esté sobre la bola unitaria. Consideremos
	\(T\colon \partial B(0,1) \to \partial B_r\) dado por 
	\(z \mapsto x + rz\). El Jacobiano es
	\begin{displaymath}
		\abs{\det DT(z)}
		=
		\abs{\begin{matrix}
			r & 0 & \cdots & 0\\
			0 & r & \cdots & 0\\
			\vdots & \vdots & \ddots & \vdots\\
			0 & 0 & \cdots & r
		\end{matrix}}
		=
		r^{n-1}.
	\end{displaymath}
	Luego, 
	\begin{align*}
		\phi(r)
		&=
		\frac{1}{n \alpha(n) r^{n-1}}	
		\int_{\partial B_r} 
			u(y)\, dS(y)
		\\&=
		\frac{1}{n \alpha(n) r^{n-1}}	
		\int_{\partial B(0,1)} 
			u(x + rz) r^{n-1}
			\, dS(z)
		\\&=
		\fint_{\partial B(0,1)}
			u(x + rz)
			\, dS(z).
	\end{align*}
	Ahora probaremos que \(\phi(r)\) es constante, y por lo tanto \(\phi(r) =
	\phi(0) = u(x)\). Vamos a ello.
	\begin{align*}
		\phi'(r)
		&\overset{TCD}{=}
		\fint_{\partial B(0,1)}
			\nabla u(x + rz)
			\cdot
			z
			\, dS(z)
		\\&=
		\fint_{\partial B_{r}}
			\nabla u(y)
			\cdot
			\frac{y-z}{r} 
			\, dS(y)
		\\&=
		\fint_{\partial B_{r}}
			\nabla u(y)
			\cdot
			\n
			\, dS(y)
		\\&\overset{\ref{teo:divergencia}}{=}
		\frac{1}{\abs{\partial B_{r}}} 
		\int_{B_{r}}
			\nabla^2 u(y)
			\, dS(y)
		= 0.
	\end{align*}
	Ahora, esto nos da la igualdad sobre las fronteras. Para pasar a la integral
	sobre el volumen notemos que
	\begin{align*}
		\fint_{B_r}
			u(y) \, dS(y)
		&=
		\frac{1}{\abs{B_{r}}}
		\int_{B_r}
			u(y) \, dS(y)
		\\&=
		\frac{1}{\abs{B_{r}}}
		\int_{0}^{r}
		\int_{\partial B(x,t)}
			u(y) \, dS(y)
			\, dt
		\\&=
		\frac{1}{\abs{B_{r}}}
		\int_{0}^{r}
			u(x) \abs{\partial B(x,t)}
			\, dt
		= u(x).
	\end{align*}
	Concluyéndose la demostración.
\end{Demostracion}

\end{document}
