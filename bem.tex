%! tex root = bem.tex
\documentclass[11pt]{article}

% Page set up
\usepackage[margin=3cm]{geometry}

% Document font and symbols
%\usepackage{mathptmx} % Times New Roman
\usepackage{amsmath,amsfonts,amsthm,amssymb}
\usepackage{esint}

% Language formatting
\usepackage[utf8]{inputenc}
\usepackage[T1]{fontenc}
\usepackage[spanish, es-noshorthands]{babel}

% Graph and Drawings
\usepackage{tikz,tikz-cd,float}
\usepackage{wrapfig}
\usetikzlibrary{babel}
\usepackage{pgfplots}
\usepackage[framemethod=default]{mdframed}
\usepackage{framed}

% Tools
\usepackage{multicol}
\usepackage{array}
\usepackage{hyperref}
\usepackage{xcolor}
\usepackage{etoolbox,mathtools}
\usepackage[shortlabels]{enumitem}

% Mdframed template
\mdfsetup{innertopmargin=5pt,%
	frametitlealignment=\raggedright,%
	frametitlefont=\texttt,%
	linewidth=2pt,%
	topline=false,rightline=false, bottomline=false,%
	frametitleaboveskip=\dimexpr-1\ht\strutbox\relax,}

% Problem Env
\newcounter{ProblemCounter}
\newenvironment{Problema}[1][]{%
	\vspace{1em}
	\refstepcounter{ProblemCounter}%
	\noindent\hspace{-12ex}{\texttt{PROBLEMA~\theProblemCounter}~}
	%\begin{tikzpicture}
	%	\draw[thick] (0,0) -- (\linewidth, 0)
	%	node[midway, fill=white] 
	%	{\texttt{PROBLEMA~\theProblemCounter}};
	%\end{tikzpicture}
	\fontfamily{ptm}\selectfont
	}{
	\ \newline
	%\begin{tikzpicture}
	%	\draw[thick] (0,0) -- (\linewidth, 0);
	%\end{tikzpicture}
	%\vspace{1ex}
	}

% Solution Env
\newenvironment{Solucion}[1][]
{%
	\vspace{1ex}
	\noindent\hspace{-10ex}{{\ttfamily SOLUCIÓN}~}
}%
{%
	%\hfill\(\blacksquare\)
}

% -- Aligned Cases
\newenvironment{caligned}
	{\begin{cases} \begin{aligned}}
	{\end{aligned} \end{cases}}

% Resize abs and norm
\DeclarePairedDelimiter{\abs}{\lvert}{\rvert}
\DeclarePairedDelimiter{\norm}{\|}{\|}
\makeatletter
\let\oldabs\abs
\def\abs{\@ifstar{\oldabs}{\oldabs*}}
\let\oldnorm\norm
\def\norm{\@ifstar{\oldnorm}{\oldnorm*}}
\makeatother

% Shortcuts
\def\V{\mathbb{V}}
\def\N{\mathbb{N}}
\def\Z{\mathbb{Z}}
\def\Q{\mathbb{Q}}
\def\R{\mathbb{R}}
\def\C{\mathbb{C}}
\def\CC{\mathcal{C}}
\def\F{\mathbb{F}}
\def\FF{\mathcal{F}}
\def\A{\mathbb{A}}
\def\n{\hat{\textbf{n}}}
\def\loc{\textrm{loc}}
\def\supp{\textrm{supp}\,}

\newcommand{\header}[2]
{
\begin{minipage}{.15\textwidth}
	\includegraphics[width=\textwidth,height=\textheight,keepaspectratio]{LogoUC}
\end{minipage}
\hspace{1em}
\begin{minipage}{.75\textwidth}
	\vspace{2em}
	{\scshape
	Pontificia Universidad Católica de Chile\\
	Facultad de Matemáticas\\
	Docente: Carlos Román\\
	Ayudante: Santiago González
	}
\end{minipage}

\medskip

\begin{center}
	{\textbf{MAT2505 - Ecuaciones Diferenciales Parciales}} \\[1ex]
	{{\large Tarea #1 - Sebastián Sánchez}}
\end{center}

\medskip
}


\title{\bfseries
Apuntes\\
``Boundary Element Methods"\\[5pt]
\small Sauter SA, Schwab C}
\author{Sebastián Sánchez}
\date{}

\begin{document}

\maketitle

\tableofcontents

\section{Ecuaciones Diferenciales Elípticas}

\subsection{Preliminares: Análisis Funcional}

\subsubsection{Espacios de Banach y Espacios de Hilbert}

\begin{Definicion}[Espacio normado]
	Sea \(X\) un espacio vectorial sobre un cuerpo \(\mathbb{K}\) (\(\mathbb{R}\) o
	\(\mathbb{C}\)) y sea \(\norm{\dotp}\colon X\to [0,\infty)\) una función tal
	que
	\begin{subequations}
	\begin{alignat}{3}
		\forall x \in X &\colon&&{}& \norm{x} &= 0 \implies x = 0
		\\
		\forall x \in X, \forall \lambda \in \mathbb{K}
		&\colon&&{}& \norm{\lambda x} &= \abs{\lambda} \norm{x}
		\\
		\forall x, y\in X &\colon&&{}& \norm{x+y} &\le \norm{x} + \norm{y}
	\end{alignat}
	\end{subequations}
	Decimos que \(\norm{\dotp}\) es una norma y la tupla \((X, \norm{\dotp})\)
	es un espacio normado. Si \(X\) no es claro del contexto, denotamos a la
	norma \(\norm{\dotp}_{X}\).
\end{Definicion}

Al definir una norma sobre \(X\), dotamos al espacio de una topología: Un
conjunto \(A\subset X\) se dice abierto si existe \(\varepsilon >0\) constante
tal que para todo \(x\in X\) la bola \(\left\{y\colon \norm{x-y} <
\varepsilon\right\}\) está contenida en \(A\). Más aún, para una sucesión
\((x_n)_n \subset X\) decimos que \(x_n\to x\) si
\begin{equation*}
	x = \lim_{n\to \infty} x_n
	\iff
	\lim_{n\to \infty} \norm{x - x_n} = 0
\end{equation*}

Por otro lado, cualquier norma es continua. En efecto, dado \(\epsilon > 0\) y
\(x,y\in X\), por la desigualdad triangular inversa
\begin{equation*}
	\abs{\norm{x} - \norm{y}} \le \norm{x-y},
\end{equation*}
tomando \(\norm{x-y} \le \epsilon\) se concluye la continuidad.

\begin{Definicion}[Operadores]
Sean \((X,\norm{\dotp}_X), (Y,\norm{\dotp}_Y)\) dos espacios normados, un mapa
\(T\colon X\to Y\) se dice \textit{operador}. Definimos la norma de un operador
\(T\) como:
\begin{equation*}
	\norm{T}_{Y\leftarrow X}
	\coloneqq
	\sup_{0_X\ne x \in X} \frac{\norm{Tx}_{Y}}{\norm{x}_{X}} < \infty.
\end{equation*}
Si no hay ambigüedad, usaremos la notación \(\norm{T}\). Al conjunto de todos
los operadores (de \(X\) a \(Y\)) que son lineales y acotados los denotamos por
\(L(X,Y)\). Este último con la norma operador es un espacio normado. Si \(X=Y\)
y definimos \((T_1 T_2) x \coloneqq T_1(T_2x)\) entonces \(L(X)\) es un álgebra.
\end{Definicion}

\begin{Ejercicio}
	\begin{enumerate}[topsep=0pt]
		\item Muestre que para todo \(x\in X\) y \(T\in L(X,Y)\) se tiene que
		\[
			\norm{Tx}_Y \le \norm{T} \norm{x}_X
		.\]

		\begin{Solucion}
			Supongamos que \(x\ne 0\), entonces
			\[
				\norm{Tx}_{Y}
				=
				\frac{\norm{Tx}_Y}{\norm{x}_{X}}
				\norm{x}_{X}
				\le
				\norm{T}
				\norm{x}_{X}
			.\]
			Para \(x=0_{X}\), \(Tx = 0_{Y}\) y por lo tanto \(\norm{Tx}_{Y} = 0 \le
			0 = \norm{T} \norm{x}_{X}\).
		\end{Solucion}

		\item Muestre que si \(T_1\in L(Y,Z)\) y \(T_2 \in L(X,Y)\) entonces
		\(T_1 T_2 \in L(X,Z)\) y
		\[
			\norm{T_1 T_2}_{Z\leftarrow X}
			\le
			\norm{T_1}_{Z\leftarrow Y}
			\norm{T_2}_{Y\leftarrow X}
		.\]

		\begin{Solucion}
			Para lo primero: si \(x\in X\) entonces \(T_2 x \in Y\) y por
			lo tanto \(T_1 T_2 x = T_1 (T_2x) \in Z\). Para lo segundo,
			ocuparemos el ejercicio anterior (omitiremos subindices para ahorrar
			notación).
			\[
				\norm{T_1 T_2}
				=
				\sup_{0_X\ne x\in X} \frac{\norm{T_1 (T_2 x)}}{\norm{x}}
				\le
				\sup_{0_X\ne x\in X} \norm{T_1} \frac{\norm{T_2x}}{\norm{x}}
				=
				\norm{T_1} \norm{T_2}
			.\]
		\end{Solucion}
	\end{enumerate}
\end{Ejercicio}

\begin{Definicion}[Convergencia de operadores]
	La sucesión \((T_n)_n \subset L(X,Y)\) converge uniformemente a \(T\) si
	\[
		T_n \to T
		\iff
		\norm{T - T_n}_{Y\leftarrow X} \to 0,
		\quad n\to \infty
	.\]
	Por otro lado, la convergencia es puntual si
	\[
		\forall x\in X\colon \norm{Tx - T_n x}_{Y} \to 0, n\to \infty
	.\]
\end{Definicion}

\begin{Definicion}[Espacio de Banach]
	Una sucesión \((x_n)_n\) de un espacio normado \(X\) se dice de Cauchy si
	\[
		\sup_{m,n\ge k} \norm{x_n - x_m} \to 0,
		\quad k\to \infty
	.\]
	Obsérvese que esto no implica convergencia en mismo espacio \(X\). De hecho,
	si todas las sucesiones de Cauchy son convergentes, \(X\) se dice un espacio
	de Banach.
\end{Definicion}

\begin{Proposicion}[$L(X,Y)$ es un espacio Banach]\label{prop:L_es_banach}
	Si \(X\) es un espacio normado cualquiera y \(Y\) es un espacio de Banach,
	entonces \(L(X,Y)\) es un espacio de Banach
\end{Proposicion}
\begin{Demostracion}
Sea \((T_n)_n\subset L(X,Y)\) una sucesión de Cauchy, es decir,
\[
	\sup_{n,m\ge k} \norm{T_n - T_m} \to 0, \quad k\to \infty
.\]
Fijando \(x\in X\), la sucesión \((T_n x)_n \subset Y\) es de Cauchy y como
\(Y\) es un espacio de Banach, la sucesión converge a un \(y_x\coloneqq
\lim_{n\to \infty} T_n x \in Y\). Definamos \(T\colon X \to Y\) como la
aplicación \(Tx \coloneqq y_x\), debemos probar que es lineal y acotada.

Para la linealidad: Si \(x=0_X\) entonces \(T_n x = 0_Y\) para todo \(n\in \mathbb{N}\), así
que \(Tx = 0_Y\). Por otro lado, si \(\alpha,\beta\in \mathbb{K}\) y \(x_1,x_2\in
X\), entonces
\[
	T(\alpha x_1 + \beta x_2)
	=
	\lim_{n\to \infty} T_n (\alpha x_1 + \beta x_2)
	=
	\lim_{n\to \infty} \alpha T_n x_1 + \beta T_n x_2
	=
	\alpha Tx_1 + \beta Tx_2
,\]
concluyéndose la linealidad. Para la cota:
\[
	\norm{T}
	=
	\sup_{0_X\ne x\in X} \frac{\norm{Tx}}{\norm{x}}
	=
	\sup_{0_X\ne x\in X} \frac{\norm{\lim_{n\to \infty} T_n x}}{\norm{x}}
	=
	\lim_{n\to \infty} \sup_{0\ne x\in X} \frac{\norm{T_n x}}{\norm{x}}
	< \infty
,\]
donde en la última igualdad usamos la continuidad de la norma.
\end{Demostracion}

\begin{Definicion}[Espacio cociente]
	Sea \(X\) un espacio de Banach y \(Z\subset X\) un subespacio cerrado.
	El espacio cociente \(X/Z\) consiste de la clases
	\[
		\tilde x
		\coloneqq
		\left\{x + z\colon z\in Z\right\}
		\quad
		\forall x\in X
	.\]
\end{Definicion}

\begin{Proposicion}\label{prop:cociente_es_banach}
	El espacio cociente \(X/Z\) con la norma
	\(
		\norm{\tilde x} \coloneqq \inf_{z\in Z} \norm{x + z}
	,\)
	es un espacio de Banach.
\end{Proposicion}
\begin{Demostracion}
Primero veamos que es una norma:
\begin{itemize}[topsep=0pt,itemsep=0pt]
	\item Definida positiva:
	\[
		\forall x\in X\colon
		\norm{\tilde x} = 0
		\iff
		\inf_{z\in Z} \norm{x + z} = 0
	.\]
	Se sigue que \(x + z = 0\) para algún \(z\in Z\), se sigue que \(x\in Z\),
	es decir, \(\tilde x = \tilde 0\). La positividad es directa porque estamos
	tomando el ínfimo de una norma.

	\item Homogeneidad absoluta:
	\[
		\forall \lambda\in \mathbb{K}, \forall x\in X\colon
		\norm{\lambda \tilde x}
		= \inf_{z\in Z} \norm{\lambda (x + z)}
		= \abs{\lambda} \inf_{z\in Z} \norm{x+z}
		= \abs{\lambda} \norm{\tilde x}
	.\]

	\item Desigualdad triangular:
	\begin{align*}
		\forall x,y\in X\colon
		\norm{\tilde x + \tilde y}
		&\le
		\inf_{z,w\in Z} \norm{(x + z) + (y + w)}
		\\&\le
		\inf_{z,w\in Z} \norm{x + z} + \norm{y + w}
		\\&\le
		\inf_{z\in Z} \norm{x + z} + \inf_{w\in Z} \norm{y + w}
		\\&=
		\norm{\tilde x} + \norm{\tilde y}
	.\end{align*}
\end{itemize}
Ahora veremos que es completo. Sea \((\tilde x_{n})_{n} \subset X/Z\) una sucesión de Cauchy y
\(\tilde x\) su límite. Queremos probar que \(\tilde x \in X/Z\), es decir, que
existe \(x\in X\) tal que \(x + Z = \tilde x\).

Nótese que si para cada \(n\) denotamos por \(x_n\) a un representante de
\(\tilde x_n\), entonces que la última sea una sucesión de Cauchy no implica
que la primera lo sea. Sin embargo, dado que \(X\) es un espacio de Banach,
existe una subsucesión \((x_{n_k})_k\) que es convergente a un \(x\in X\). Se
sigue que \((\tilde x_{n_{k}})_{k}\) converge a \(\tilde x\). Como la sucesión
es \((\tilde x_n)_n\) es de Cauchy y tiene una subsucesión convergente,
concluimos que \(\tilde x_n\) es convergente.
\end{Demostracion}

Recordamos que un conjunto \(A\subset X\) es \textit{denso} en \(X\) si \(\overline{A} =
X\). En particular, esto nos dice que para todo elemento \(x\) de \(X\) existe una
sucesión en \(A\) que converge a \(x\). Un espacio de Banach se dice
\textit{separable} si contiene un subconjunto contable y denso.

Si \((X,\norm{\dotp}_X)\) no es un espacio de Banach pero \(X\) es denso en
\(\tilde X\), entonces \((\tilde X, \norm{\dotp}_{\tilde X})\) es un espacio de
Banach llamado la \textit{completación} de \(X\) donde \(\norm{x}_{\tilde X}
= \norm{x}_X \forall x \in X\). La completación es única salvo isomorfismo. Más
aún, los operadores con dominio en \(X\) se extienden de manera única a \(\tilde
X\). La siguiente proposición entrega más detalles.

\begin{Proposicion}\label{prop:extension_operador}
	Sea \(X\) un espacio normado y \(X_0\) un subcojunto denso de \(X\). Un
	operador \(T_0\in L(X_0, Y)\) tiene una única extensión \(T\in L(X,Y)\) y
	satisface
	\begin{enumerate}[topsep=0pt,itemsep=0pt]
		\item \(T|_{X_0} = T_0\);
		\item Para toda sucesión \((x_n)_n\subset X_0\) tal que \(x_n \to x\in X\)
		se tiene que \(Tx = \lim_{n\to \infty} T_0\, x_n\);
		\item \(\norm{T}_{Y\leftarrow X} = \norm{T_0}_{Y\leftarrow X_0}\).
	\end{enumerate}
\end{Proposicion}
\begin{Demostracion}
Definamos \(Tx = T_0 x\) si \(x\in X_0\) y \(Tx = \lim_{n\to \infty} T_0 x_n\)
para alguna sucesión \(x_n\to x\). Como \(L(X,Y)\) es Banach y \(x_n\) es de
Cauchy, el mapa \(T\) está bien definido, pues
\begin{displaymath}
	\norm{Tx_n - Tx_m}
	=
	\norm{T_0 (x_n - x_m)}
	\to 0,
\end{displaymath}
para \(n,m \to \infty\). Por otro lado, la definición de \(T\) es
independiente de la sucesión escogida. En efecto, sea \(Sx = \lim_{n \to \infty}
T_0 y_n\) para \(y_n\to x\) una sucesión distinta de la que toma \(Tx\). Luego,
\begin{displaymath}
	\norm{Tx - Sx}
	=
	\lim_{n\to \infty} \norm{T_0 (x_n - y_n)}
	\le
	\lim_{n\to \infty}\left( \norm{T_0 (x_n - x)} + \norm{T_0 (x - y_n)} \right)
	\to 0
	, \quad n\to \infty.
\end{displaymath}
Finalmente, vemos la doble desigualdad
\begin{displaymath}
	\norm{T_0}_{Y\leftarrow X_0}
	=
	\sup_{x\ne 0 \in X_0} \frac{\norm{T_0 x}}{\norm{x}}
	=
	\sup_{x\ne 0 \in X_0} \frac{\norm{T x}}{\norm{x}}
	\le
	\sup_{x\ne 0 \in X} \frac{\norm{T x}}{\norm{x}}
	=
	\norm{T}_{Y\leftarrow X}
\end{displaymath}
y
\begin{displaymath}
	\norm{T}_{Y\leftarrow X}
	=
	\sup_{x\ne 0\in X} \frac{\norm{T x}}{\norm{x}}
	=
	\sup_{x_n \to x\ne 0 \subset \in X_0}
		\lim_{n\to \infty} \frac{\norm{T_0 x_n}}{\norm{x_n}}
	\le
	\sup_{x\ne 0\in X_0} \frac{\norm{T_0 x_n}}{\norm{x_n}}
	=
	\norm{T_0}_{Y\leftarrow X_0}.
\end{displaymath}
\end{Demostracion}

\begin{Teorema}
	Sean \(X, Y\) espacios de Banach y \(T\in L(X,Y)\). Si \(T\) es
	biyectiva, entonces su inversa \(T^{-1}\) existe y \(T^{-1}\in L(Y,X)\).
\end{Teorema}
\begin{Demostracion}
Recordar que \(T^{-1}y = \left\{ x\in X\colon Tx = y \right\} \). Como \(T\) es
biyectiva, \(T^{-1}\) está bien definida sobre todo \(Y\). Además, \(T^{-1}\) es
lineal. En efecto,
\begin{itemize}
	\item \(T^{-1}0 = 0\) pues \(T\) es lineal e inyectiva.
	\item Sea \(x_1, x_2 \in X\) y \(\alpha \in \mathbb{K}\). Luego, \(\alpha x_1
	+ x_2 = \alpha T^{-1} y_1 + T^{-1}y_2\) para algún par \(y_1,y_2\in Y\).
	Aplicando \(T\) y usando su linealidad nos deja
	\begin{displaymath}
		T(\alpha x_1 + x_2) = \alpha y_1 + y_2
	\end{displaymath}
	Si ahora aplicamos \(T^{-1}\) queda
	\begin{displaymath}
		\alpha x_1 + x_2 = T^{-1}(\alpha y_1 + y_2)
	\end{displaymath}
	Juntando todas las ecuaciones se obtiene lo buscado
	\begin{displaymath}
		T^{-1}(\alpha y_1 + y_2)
		=
		\alpha x_1 + x_2
		=
		\alpha T^{-1}(y_1) + T^{-1}(y_2).
	\end{displaymath}
\end{itemize}

Que \(T^{-1}\) esté acotado es una consecuencia del Teorema del Mapeo Abierto.
\end{Demostracion}

\begin{Corolario}
	Si \(X,Y\) son espacios de Banach y \(T\in L(X,Y)\) es inyectiva, las
	siguientes condiciones son equivalentes
	\begin{enumerate}[topsep=0pt]
		\item \(Y_0 \coloneqq \left\{ Tx\colon x\in X \right\}\) con la norma
		\(\norm{\dotp}_{Y}\) es un subespacio cerrado de \(Y\).
		\item \(T^{-1}\) existe en \(Y_0\). Más aún, \(T^{-1} \in L(Y_0, X)\).
	\end{enumerate}
\end{Corolario}
\begin{Demostracion}
Asumiendo 1. tenemos que la imagen de \(T\) es un espacio de Banach y por lo
tanto aplica el Teorema anterior. Asumiendo 2. y en busca de una contradicción,
si \(Y_0\) es abierto, tomamos \(y\in \partial Y_0\) y una sucesión
\((y_{n})_{n} \subset Y_0\) que converge a \(y\). Como \(T\) es biyección,
existe una sucesión \((x_n)_n \in X\) tal que \(x_n = T^{-1} y_n\). Más aún.
Dado \(T, T^{-1}\) son lineales, la sucesión \(x_n\) es de Cauchy y por lo tanto
convergente a \(T^{-1} y\). Luego, \(\lim_{n\to \infty} T x_n = y\not\in Y_0\). Contradicción.
\end{Demostracion}

\begin{Definicion}[Incrustación]
	Sean \(X,Y\) espacios de Banach. Si existe \(\iota \in L(X,Y)\) inyectiva,
	esta se dice \textit{incrustación} o \textit{encaje} de \(X\) en \(Y\). Si
	\(\iota\) es continua y \(\iota(X)\) es denso en \(Y\), decimos que
	\(X\) está \textit{continua y densamente incrustado} en \(Y\).
\end{Definicion}

\begin{Definicion}[Espacio de Hilbert]
	Sea \(X\) un espacio vectorial y \(\langle \dotp, \dotp \rangle\colon
	X\times X \to \mathbb{K}\) un producto interno en \(X\), es decir, \(\langle
	\dotp, \dotp \rangle\) satisface
	\begin{subequations}
	\begin{alignat}{3}
		\forall x\in X\setminus \left\{ 0 \right\} &\colon
		&&{}& \langle x,x \rangle &> 0,
		\\
		\forall x,y,z\in X, \lambda\in \mathbb{K} &\colon
		&&{}& \langle \lambda x + y, z \rangle
		&=
		\lambda \langle x,z \rangle + \langle y,z \rangle,
		\\
		\forall x,y\in X &\colon
		&&{}&\overline{\langle x,y \rangle} &= \langle y,x \rangle.
	\end{alignat}
	\end{subequations}
	Si la norma inducida por \(\langle \dotp,\dotp \rangle \) definida por
	\(
		\norm{x} \coloneqq \langle x,x \rangle^{1/2}
	\)
	hace a \(X\) un espacio de Banach, decimos que \(X\) es un espacio de Hilbert.
\end{Definicion}

De la definición anterior se tiene la desigualdad de Cauchy-Schwarz
\begin{equation}
	\forall x,y\in X\colon \abs{\langle x,y \rangle} \le \norm{x}\,\,\norm{y}.
\end{equation}

\begin{Definicion}[Ortogonalidad]
	Sea \((X,\langle \dotp, \dotp \rangle)\) un espacio de Hilbert. Los vectores \(x,y\in X\)
	se dicen ortogonales, denotado por \(x\perp y\), si \(\langle x,y \rangle =
	0\). Extendemos la definición para subespacios \(A\) de \(X\) como
	\[
		A^{\perp}
		\coloneqq
		\left\{
			x\in X\colon \forall a\in A ,
			\, \langle x,a \rangle = 0
			\,\forall x\in X
		\right\}
	.\]
\end{Definicion}

\begin{Proposicion}\label{prop:Aperp_es_cerrado}
	Si \(A\subset X\) entonces \(A^{\perp}\) es un subespacio cerrado de \(X\).
\end{Proposicion}
\begin{Demostracion}
Que sea subespacio es directo por la bilinealidad del producto interno, así que
probaremos que es cerrado. Sea \((x_n)_{n} \subset A^{\perp}\) tal que \(x_n \to
x\in X\). Queremos ver que \(x\in A^{\perp}\). Dado que
\begin{displaymath}
	\forall a \in A\colon
	\langle x, a \rangle
	=
	\langle \lim_{n\to \infty} x_n, a \rangle.
\end{displaymath}
Basta probar que el mapa \(\langle \cdotp, a \rangle\) es continuo para todo
\(a\ne 0 \in A^{\perp}\). Esto es directo de la desigualdad de Cauchy-Schwarz.
En efecto, sea \(\epsilon >0\) y \(y,z \in X\) tal que \(\norm{y-z} <
\epsilon/\norm{a}\). Luego
\begin{displaymath}
	\abs{\langle y, a \rangle
	-
	\langle z, a \rangle}
	=
	\abs{\langle y-z, a \rangle}
	\le
	\norm{y-z}\,\, \norm{a}
	<
	\epsilon.
\end{displaymath}
De esta forma tenemos que
\begin{displaymath}
	\forall a \in A\colon
	\langle x, a \rangle
	=
	\langle \lim_{n\to \infty} x_n, a \rangle.
	=
	\lim_{n\to \infty} \langle x_n, a \rangle
	=
	0.
\end{displaymath}
\end{Demostracion}

\begin{Proposicion}[Descomposición Ortogonal]%
\label{prop:descomposicion_ortogonal}
	Sea \(X\) un espacio de Hilbert y \(U\) un subespacio cerrado de \(X\).
	Luego, \(X = U \oplus U^{\perp}\).
\end{Proposicion}
\begin{Demostracion}
Sea \(x\in X\). Sea \(u\in U\) y definamos \(\lambda = \langle x,u
\rangle/\norm{u}\) y \(u^{\perp} = x - \lambda u\). Nótese que
\begin{displaymath}
	\langle x - \lambda u, u \rangle = 0.
\end{displaymath}
De esta forma, \(x = \lambda u + u^{\perp} \in U\oplus U^{\perp}\).
\end{Demostracion}

\begin{Definicion}[Base Ortonormal]
	Una base ortonormal de un espacio de Hilbert \(X\), es una sucesión de
	vectores \((v_j)_{j\in I}\) tales que la expansión de Fourier en
	cada punto es convergente, en símbolos:
	\begin{equation}
		\forall x\in X\colon
		x = \sum_{j\in I} \langle x,v_j \rangle v_j
	\end{equation}
\end{Definicion}

\begin{Teorema}
	Todo espacio de Hilbert tiene una base ortonormal
\end{Teorema}
\begin{Demostracion}
\end{Demostracion}

\subsubsection{Espacios Duales}

En esta parte \(X\) denotará un espacio vectorial normado sobre un cuerpo
\(\mathbb{K} = \mathbb{C}, \mathbb{R}\).

\begin{Definicion}[Espacio Dual]\label{def:espacio_dual}
	El espacio dual \(X\) se define como el conjunto de los mapas lineales
	acotados de \(X\) a su campo escalar \(\mathbb{K}\).
	\begin{equation*}
		X' = L(X, \mathbb{K}).
	\end{equation*}
	Los elementos de \(X'\) son llamados funcionales lineales.
\end{Definicion}

\begin{Proposicion}\label{prop:espacio_dual_es_banach}
	\(X'\) es un espacio de Banach con la norma
	\(
		\norm{x'}_{X'} \coloneqq \norm{x'}_{\mathbb{K}\leftarrow X}.
	\)
\end{Proposicion}
\begin{Demostracion}
Una aplicación de Proposición~\ref{prop:L_es_banach}.
\end{Demostracion}

Nótese que \(x'(x)\) se puede escribir como
\begin{equation*}
	x'(x)
	=
	\langle x, x' \rangle_{X\times X'}
	=
	\langle x', x \rangle_{X'\times X}.
\end{equation*}
Estas son llamadas \textit{formas duales} o \textit{pares duales}.

\begin{Proposicion}
	Si \(X\) está continuamente incrustado en \(Y\), entonces \(Y'\) está
	continuamente incrustado en \(X'\).
\end{Proposicion}
\begin{Demostracion}
Sea \(\iota\) la incrustación de \(X\) en \(Y\). Definamos \(j\colon Y' \to \iota(X)'\)
por \(y' = y' \circ \iota\). El mapa está bien definido pues para \(x\in X\),
\(\iota(x) \in Y\) y por lo tanto \(y'(\iota(x)) \in \K\). El mapa es lineal y
continuo por ser composición de mapas lineales y continuos.
\end{Demostracion}

\begin{Definicion}[Espacio Bidual]\label{def:bidual}
	El espacio bidual de \(X\) está definido como
	\begin{equation*}
		X'' = L(X', \mathbb{K})
	\end{equation*}
\end{Definicion}

En general \(X \cong X''\). En tales casos \(X\) se dice \textit{reflexivo}.

\begin{Proposicion}\label{prop:hilbert_reflexivo}
	Todos los espacios de Hilbert son reflexivos.
\end{Proposicion}
\begin{Demostracion}
Para cada \(x' \in X'\) tenemos el mapeo evualuación \(E_x\colon X' \to \K\)
dado por \(E_x(x') = x'(x)\). Notése que \(E_{x} \in X''\). Debemos ver que el
mapa \(x\mapsto E_x\) es inyectivo y sobreyectivo.

Para la inyectividad supogangamos que \(x_1 \ne x_2\in X\) y que \(E_{x_1} =
E_{x_2}\). Luego,
\begin{displaymath}
	\forall x'\in X'\colon
	x'(x_1) = x'(x_2)
	\implies
	x'(x_1 - x_2) = 0,
\end{displaymath}
es decir, \(x_1 - x_2 \in \ker x'\) para todo \(x'\in X'\). Afirmamos que
\(X_0 = \bigcap_{x'\in X'} \ker x' = 0\). En efecto, si \(x_0 \ne 0 \in X_0\),
entonces definamos \(x'_{0}(x) = \langle x, x_0 \rangle\). Dado que \(X\) es un
espacio de Hilbert, \(x'_0\) es un mapa lineal y continuo, además, \(x'(x_0) >
0\). Por lo tanto, \(X_0 = 0\).

Para la sobreyectividad, tomemos \(x'' \in X''\). Queremos ver que \(x'' =
E_{x}\) para algún \(x\in X\).
\end{Demostracion}

\begin{Teorema}[Hanh-Banach para espacios de Banach]%
\label{teo:hahn_banach_normado}
	Sea \(X\) un espacio de Banach y \(M\) un subespacio de \(X\). Si \(f_0\) es
	un funcional lineal continuo sobre \(M\), entonces existe un funcional
	lineal continuo \(f\) definido sobre \(X\) tal que
	\begin{displaymath}
		\begin{array}{c c c}
		(1)\,\,f|_{M} = f_0 &
		\textrm{y} &
		(2)\,\,\norm{f_{0}}_{\mathbb{C}\leftarrow M} =
		\norm{f}_{\mathbb{C}\leftarrow X}.
		\end{array}
	\end{displaymath}
\end{Teorema}
\begin{Demostracion}
	Momentáneamente supondremos que \(X\) es un espacio normado real.

	En primer lugar, consideremos \(x_1\in X\setminus M\) y el subespacio \(M_1 = \spn
	\left\{M, x_1\right\}\). Nótese que \(x\in M_1\) se puede escribir de
	manera única como \(x = x_M + \lambda x_1\) con \(x_M \in M\) y \(\lambda\in
	\R\). Para que \(f_1\) sea una extensión lineal de \(f_0\) a \(M_1\), es necesario que
	\begin{displaymath}
		f_1(x) = f_0(x_M) + \lambda f_1(x_1),
	\end{displaymath}
	Por lo que \(f_1(x_1)\) determina al mapa. Más aún, para que sea continuo, es
	necesario que sea acotado. Como queremos mantener la norma, es necesario que
	\(\abs{f_1(x)} \le \norm{f_0} \norm{x}\) para todo \(x\in M_1\). La pregunta
	yace entonces en si existe una asignación \(f_1(x_1)\) que cumpla con lo que
	queremos.

	Para \(\lambda = 0\) la elección de \(f_1(x_1)\) es indiferente, por lo tanto,
	supondremos \(\lambda \ne 0\). Para \(\lambda > 0\) tenemos que
	\begin{displaymath}
		\abs{f_0(x_M/\lambda) + f_1(x_1)}
		\le
		\norm{f_0} \norm{x_M/\lambda + x_1}
	\end{displaymath}
	mientras que para \(\lambda < 0\) se tiene
	\begin{displaymath}
		\abs{f_0(x_M/(-\lambda)) - f_1(x_1)}
		\le
		\norm{f_0} \norm{x_M/(-\lambda) - x_1}
	\end{displaymath}
	
	Como \(M\) es subespacio, las condiciones anteriores para \(f_1(x_1)\) se
	reducen a cumplir
	\begin{displaymath}
		f_0(m_1) - \norm{f_0} \norm{m_1 - x_1}
		\le
		f_1(x_1) 
		\le 
		\norm{f_0} \norm{m_2 + x_1} - f_0(m_2 x) 
		\qquad \forall m_1, m_2 \in M.
	\end{displaymath}

	Luego, es suficiente para \(f_1(x_1)\) cumplir con:   
	\begin{displaymath}
		\sup_{m_1 \in M} f_0(m_1) - \norm{f_0} \norm{m - x_1}
		\le
		f_1(x_1)
		\le
		\inf_{m_2 \in M} \norm{f_0} \norm{m_2 + x_1} - f_0(m_2)
	\end{displaymath}
	Tal elección es posible, pues
	\begin{displaymath}
		f_0(m_1) + f_0(m_2)
		=
		f_0(m_1 + m_2)
		\le
		\norm{f_0} \norm{m_1 + m_2}
		\le
		\norm{f_0} \left( \norm{m_1 - x_1} + \norm{m_2 + x_1} \right) 
	\end{displaymath}

	En resumen, hemos probado que existe una extensión de \(f_0\) que es lineal,
	continua y mantiene la norma.

	Si \(X\) es finito dimensional, basta aplicar
	inductivamente el procedimiento. El caso infinito dimensional es un poco más
	trabajoso; Debemos definir un order parcial sobre las extensiones de \(f_0\) y 
	usar el Lema de Zorn para concluir que el elemento máximal es el espacio
	\(X\).

	Con lo primero, decimos que \(h\succ g\) si \(h\) es una extensión de \(g\).
	Luego, por el Lema de Zorn existe un elemento máximal \(f\) que extiende a
	\(f_0\) y tiene por dominio el espacio \(X\). En efecto, si el dominio no
	fuera todo \(X\) podemos aplicar el procedimiento de arriba y concluir que
	hay un elemento \(F\) que extiende a \(f\), lo cual contradice la
	máximalidad del último.  
\end{Demostracion}

\begin{Corolario}
	Sea \(X\) un espacio de Banach y \(x_0\ne 0\in X\). Entonces existe un
	funcional lineal continuo \(f_0\) sobre \(X\) tal que
	\begin{displaymath}
		\begin{array}{c c c}
			f_0(x_0) = \norm{x_0}_{X}
			& \textrm{ y } &
			\norm{f_0}_{X'} = 1.
		\end{array}
	\end{displaymath}
\end{Corolario}
\begin{Demostracion}
	Consideremos \(M = \spn(x_0)\). Definamos \(f_0(\alpha x_0) = \alpha \norm{x_0}\) para 
	\(\alpha\in \C\). Se verifica que \(f_0\) es un funcional lineal continuo y que
	\(\norm{f_0}_{X'} = 1\). Lo demás es directo por el
	Teorema~\ref{teo:hahn_banach_normado}.
\end{Demostracion}

\end{document}
