%! tex root = A10.tex
\documentclass[11pt]{article}

% Page set up
\usepackage[margin=3cm]{geometry}

% Document font and symbols
%\usepackage{mathptmx} % Times New Roman
\usepackage{amsmath,amsfonts,amsthm,amssymb}
\usepackage{esint}

% Language formatting
\usepackage[utf8]{inputenc}
\usepackage[T1]{fontenc}
\usepackage[spanish, es-noshorthands]{babel}

% Graph and Drawings
\usepackage{tikz,tikz-cd,float}
\usepackage{wrapfig}
\usetikzlibrary{babel}
\usepackage{pgfplots}
\usepackage[framemethod=default]{mdframed}
\usepackage{framed}

% Tools
\usepackage{multicol}
\usepackage{array}
\usepackage{hyperref}
\usepackage{xcolor}
\usepackage{etoolbox,mathtools}
\usepackage[shortlabels]{enumitem}

% Mdframed template
\mdfsetup{innertopmargin=5pt,%
	frametitlealignment=\raggedright,%
	frametitlefont=\texttt,%
	linewidth=2pt,%
	topline=false,rightline=false, bottomline=false,%
	frametitleaboveskip=\dimexpr-1\ht\strutbox\relax,}

% Problem Env
\newcounter{ProblemCounter}
\newenvironment{Problema}[1][]{%
	\vspace{1em}
	\refstepcounter{ProblemCounter}%
	\noindent\hspace{-12ex}{\texttt{PROBLEMA~\theProblemCounter}~}
	%\begin{tikzpicture}
	%	\draw[thick] (0,0) -- (\linewidth, 0)
	%	node[midway, fill=white] 
	%	{\texttt{PROBLEMA~\theProblemCounter}};
	%\end{tikzpicture}
	\fontfamily{ptm}\selectfont
	}{
	\ \newline
	%\begin{tikzpicture}
	%	\draw[thick] (0,0) -- (\linewidth, 0);
	%\end{tikzpicture}
	%\vspace{1ex}
	}

% Solution Env
\newenvironment{Solucion}[1][]
{%
	\vspace{1ex}
	\noindent\hspace{-10ex}{{\ttfamily SOLUCIÓN}~}
}%
{%
	%\hfill\(\blacksquare\)
}

% -- Aligned Cases
\newenvironment{caligned}
	{\begin{cases} \begin{aligned}}
	{\end{aligned} \end{cases}}

% Resize abs and norm
\DeclarePairedDelimiter{\abs}{\lvert}{\rvert}
\DeclarePairedDelimiter{\norm}{\|}{\|}
\makeatletter
\let\oldabs\abs
\def\abs{\@ifstar{\oldabs}{\oldabs*}}
\let\oldnorm\norm
\def\norm{\@ifstar{\oldnorm}{\oldnorm*}}
\makeatother

% Shortcuts
\def\V{\mathbb{V}}
\def\N{\mathbb{N}}
\def\Z{\mathbb{Z}}
\def\Q{\mathbb{Q}}
\def\R{\mathbb{R}}
\def\C{\mathbb{C}}
\def\CC{\mathcal{C}}
\def\F{\mathbb{F}}
\def\FF{\mathcal{F}}
\def\A{\mathbb{A}}
\def\n{\hat{\textbf{n}}}
\def\loc{\textrm{loc}}
\def\supp{\textrm{supp}\,}

\newcommand{\header}[2]
{
\begin{minipage}{.15\textwidth}
	\includegraphics[width=\textwidth,height=\textheight,keepaspectratio]{LogoUC}
\end{minipage}
\hspace{1em}
\begin{minipage}{.75\textwidth}
	\vspace{2em}
	{\scshape
	Pontificia Universidad Católica de Chile\\
	Facultad de Matemáticas\\
	Docente: Carlos Román\\
	Ayudante: Santiago González
	}
\end{minipage}

\medskip

\begin{center}
	{\textbf{MAT2505 - Ecuaciones Diferenciales Parciales}} \\[1ex]
	{{\large Tarea #1 - Sebastián Sánchez}}
\end{center}

\medskip
}


\begin{document}

\header{10}

\begin{Problema}
	Sea \(\Omega \subset \R^n\) un dominio acotado. Muestre que 
	\(W^{k+1,p}\) es denso en \(W^{k,p}\).
\end{Problema}
\begin{Solucion}
	Dado que \(\Omega\) es acotado, tenemos que \(\CC^{\infty}
	\subset W^{m, p}\) para cualquier \(m\). Por otro lado,
	sabemos que \(W^{k+1,p} \subset W^{k,p}\) y que por el teorema
	de aproximación por funciones suaves se tiene que
	\(\CC^{\infty}\) es denso en \(W^{m,p}\). En resumen:
	\begin{displaymath}
		\CC^{\infty}
		\subset
		W^{k+1, p}
		\subset 
		W^{k,p}
		\Rightarrow
		\underbrace{ 
			\overline{\CC^{\infty}}
		}_{=W^{k,p}}
		\subset
		\overline{W^{k+1, p}}
		\subset 
		W^{k,p},
	\end{displaymath}	
	donde la clausura la tomamos en \(W^{k,p}\).
\end{Solucion}

\begin{Problema}
	Sea \(\Omega = [0,2\pi)\).

	\noindent(1) Encuentre una base ortonormal para
	\(H^{k}(\Omega)\). ¿Qué relación puede establecer con una base
	de \(L^{2}(\Omega)\)?. 

	\noindent(2) Use lo anterior para deducir que el operador lineal
	de inclusión \(H^{k}(\Omega) \hookrightarrow L^{2}(\Omega)\) es
	compacto.
\end{Problema}
\begin{Solucion}
	\noindent(1) Sabemos que \(\left\{ \sqrt{2\pi}^{-1} e^{i\,nx}
	\right\}_{n\in\N}\) es base ortonormal de \(L^2\). Usaremos esta
	para construir una base de \(H^k\).
\end{Solucion}

\end{document}
