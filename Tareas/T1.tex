%! tex root = T1.tex
\documentclass[11pt]{article}

% Page set up
\usepackage[margin=3cm]{geometry}

% Document font and symbols
%\usepackage{mathptmx} % Times New Roman
\usepackage{amsmath,amsfonts,amsthm,amssymb}
\usepackage{esint}

% Language formatting
\usepackage[utf8]{inputenc}
\usepackage[T1]{fontenc}
\usepackage[spanish, es-noshorthands]{babel}

% Graph and Drawings
\usepackage{tikz,tikz-cd,float}
\usepackage{wrapfig}
\usetikzlibrary{babel}
\usepackage{pgfplots}
\usepackage[framemethod=default]{mdframed}
\usepackage{framed}

% Tools
\usepackage{multicol}
\usepackage{array}
\usepackage{hyperref}
\usepackage{xcolor}
\usepackage{etoolbox,mathtools}
\usepackage[shortlabels]{enumitem}

% Mdframed template
\mdfsetup{innertopmargin=5pt,%
	frametitlealignment=\raggedright,%
	frametitlefont=\texttt,%
	linewidth=2pt,%
	topline=false,rightline=false, bottomline=false,%
	frametitleaboveskip=\dimexpr-1\ht\strutbox\relax,}

% Problem Env
\newcounter{ProblemCounter}
\newenvironment{Problema}[1][]{%
	\vspace{1em}
	\refstepcounter{ProblemCounter}%
	\noindent\hspace{-12ex}{\texttt{PROBLEMA~\theProblemCounter}~}
	%\begin{tikzpicture}
	%	\draw[thick] (0,0) -- (\linewidth, 0)
	%	node[midway, fill=white] 
	%	{\texttt{PROBLEMA~\theProblemCounter}};
	%\end{tikzpicture}
	\fontfamily{ptm}\selectfont
	}{
	\ \newline
	%\begin{tikzpicture}
	%	\draw[thick] (0,0) -- (\linewidth, 0);
	%\end{tikzpicture}
	%\vspace{1ex}
	}

% Solution Env
\newenvironment{Solucion}[1][]
{%
	\vspace{1ex}
	\noindent\hspace{-10ex}{{\ttfamily SOLUCIÓN}~}
}%
{%
	%\hfill\(\blacksquare\)
}

% -- Aligned Cases
\newenvironment{caligned}
	{\begin{cases} \begin{aligned}}
	{\end{aligned} \end{cases}}

% Resize abs and norm
\DeclarePairedDelimiter{\abs}{\lvert}{\rvert}
\DeclarePairedDelimiter{\norm}{\|}{\|}
\makeatletter
\let\oldabs\abs
\def\abs{\@ifstar{\oldabs}{\oldabs*}}
\let\oldnorm\norm
\def\norm{\@ifstar{\oldnorm}{\oldnorm*}}
\makeatother

% Shortcuts
\def\V{\mathbb{V}}
\def\N{\mathbb{N}}
\def\Z{\mathbb{Z}}
\def\Q{\mathbb{Q}}
\def\R{\mathbb{R}}
\def\C{\mathbb{C}}
\def\CC{\mathcal{C}}
\def\F{\mathbb{F}}
\def\FF{\mathcal{F}}
\def\A{\mathbb{A}}
\def\n{\hat{\textbf{n}}}
\def\loc{\textrm{loc}}
\def\supp{\textrm{supp}\,}

\newcommand{\header}[2]
{
\begin{minipage}{.15\textwidth}
	\includegraphics[width=\textwidth,height=\textheight,keepaspectratio]{LogoUC}
\end{minipage}
\hspace{1em}
\begin{minipage}{.75\textwidth}
	\vspace{2em}
	{\scshape
	Pontificia Universidad Católica de Chile\\
	Facultad de Matemáticas\\
	Docente: Carlos Román\\
	Ayudante: Santiago González
	}
\end{minipage}

\medskip

\begin{center}
	{\textbf{MAT2505 - Ecuaciones Diferenciales Parciales}} \\[1ex]
	{{\large Tarea #1 - Sebastián Sánchez}}
\end{center}

\medskip
}


\begin{document}

\header{1}{}

\fontfamily{lmss}\selectfont
\noindent\framebox{AVISO} El vector normal lo denoto por \(\n\) en vez de \(\nu\).

% Problema 1
\begin{Problema}
	Pruebe que \(u(x) = \dfrac{e^{-\abs{x}}}{ 4\pi \abs{x}} \) satisface, en el
	sentido de las distribuciones,
	\begin{displaymath}
		-\Delta u + u = \delta_{0}
		\text{ en } \R^3.
	\end{displaymath}
\end{Problema}

\begin{Solucion}
	La igualdad en el sentido de distribuciones significa que:
	\begin{displaymath}
		\langle -\Delta f_{u} + f_{u}, \psi \rangle 
		=
		\psi(0)
		\quad
		\forall \psi \in \mathcal{D}(\R^3),
	\end{displaymath}
	donde \(f_{u}\) es la distribución asociada a \(u\).
	Debemos probar esto (que existe y vale la igualdad).

	En primer lugar, veamos que \(u\in L^{1}_{loc}(\R^3)\). Como \(u\) es
	continua lejos de cero, en particular es acotada y por lo tanto su integral
	es finita en cualquier compacto. Así que en realidad basta ver que la 
	integral es finita cerca de cero. Consideremos \(B_{1}\) la bola unitaria
	abierta. Luego,
	\begin{displaymath}
		\int_{B_{1}} \abs{u(x)} dx
		\le
		\int_{B_{1}} \frac{1}{4\pi \abs{x}^{1}} dx.
	\end{displaymath}
	Recordamos que \(f(x) = 1/\abs{x}^{\alpha}\) tiene integral finita sobre la
	bola unitaria si y solo si \(\alpha < n\). En este caso \(\alpha = 1\) y
	\(n=3\), así que la integral es finita. Con esto tenemos que \(u\) es
	localmente integrable en \(\R^3\).

	Dado que \(u\) es la localmente integrable, existe una distribución
	\(f_{u}\) asociada. Más aún, existen \(\partial_{x_i} f_{u}\) distribuciones
	asociadas para cada \(i=1,\dots, n\) definidas como
	\begin{displaymath}
		\langle \partial_{x_i} f_{u}, \psi \rangle 
		=
		- \langle f_{u}, \partial_{x_i} \psi \rangle 
		\quad\forall \psi \in \mathcal{D}(\R^3).
	\end{displaymath}
	Se sigue que \(\Delta f_{u}\) satisface,
	\begin{displaymath}
		\langle -\Delta f_{u}, \psi \rangle
		=
		- \langle f_{u},  \Delta \psi \rangle.
	\end{displaymath}
	Y de hecho, \((-\Delta f_{u} + f_{u})\) también es una 
	distribución definida como,
	\begin{displaymath}
		\langle -\Delta f_{u} + f_{u}, \psi \rangle 
		\coloneqq
		\langle f_u, -\Delta \psi + \psi \rangle.
	\end{displaymath}
	En efecto, si \(K\) es un compacto y \(\psi\in \mathcal{D}\) es tal que
	\(\supp \psi \subset K\), entonces:
	\begin{displaymath}
		\abs{\langle -\Delta f_{u} + f_{u}, \psi \rangle}
		\le
		\int_{K} \abs{(-\Delta \psi + \psi) u} 
		\le
		\sup_{K} \abs{-\Delta \psi + \psi}
		\underbrace{\int_{K} \abs{u}}_{ < \infty}.
	\end{displaymath}
	Nótese que la distribución es de orden 2 con constante \(\int_{K} \abs{u}\).

	Ahora debemos probar que \(-\Delta f_{u} + f_{u}\) opera como \(\delta_0\).
	Es decir, debemos probar que
	\begin{displaymath}
		\langle -\Delta f_{u} + f_{u}, \psi \rangle 
		=
		\int_{\R^n} (-\Delta \psi + \psi) u 
		=
		\psi(0)
		=
		\langle \delta_0, \psi \rangle.
	\end{displaymath}
	Vamos a ello. Sea \(1 > \epsilon > 0\) un real positivo menor uno y 
	\(B \coloneqq B(0, \epsilon)\) la bola centrada en el origen de radio
	\(\epsilon\). Luego,
	\begin{displaymath}
		\int_{\R^{n}} (-\Delta \psi + \psi) u
		=
		\underbrace{
			\int_{B} (-\Delta \psi + \psi) u
		}_{I_1}
		+
		\underbrace{
			\int_{\R^{n}\setminus B} (-\Delta \psi + \psi) u
		}_{I_2}.
	\end{displaymath}
	Primero lidiemos con \(I_1\).
	\begin{alignat*}{2}
		\norm{I_1}
		&\le
		\norm{-\Delta\psi + \psi}_{\infty}
		\abs{
			\int_{B} u
		}
		\\&\le
		\norm{-\Delta\psi + \psi}_{\infty}
		\abs{
			\int_{0}^{\pi}
			\int_{0}^{2\pi}
			\int_{0}^{\epsilon}
				u(r, t_1, t_2)
				r^2
				\, dr \, dt_1 \, dt_{2} 
		}.
	\end{alignat*}
	Luego,
	\begin{alignat*}{3}
		\norm{I_1}
		&\le
		\norm{-\Delta\psi + \psi}_{\infty}
		\abs{
			2\pi^2
			\int_{0}^{\epsilon}
				\frac{e^{-r}}{4\pi r} r^2 \, dr
		}
		\\&=
		\norm{-\Delta\psi + \psi}_{\infty}
		\abs{
			\frac{\pi}{2}
			\int_{0}^{\epsilon}
				e^{-r} r \, dr
		}
		\\&=
		\frac{\pi}{2} 
		\norm{-\Delta\psi + \psi}_{\infty}
		\big(
			e^{-\epsilon} (\epsilon + 1)
			-
			1
		\big).
	\end{alignat*}
	Cuando \(\epsilon \to 0\) la expresión se va a \(0\).

	Ahora veamos \(I_2\). Consideremos \(R \gg 1\) tal que \(\supp \psi \subset
	B(0,R) \eqqcolon B_{R}\). Luego,
	\begin{alignat*}{2}
		I_2
		&=
		\int_{B_{R}\setminus B} (-\Delta \psi + \psi) u
		\\&=
		\int_{B_{R}\setminus B} -u\Delta \psi 
		+ 
		\int_{B_{R}\setminus B} u \psi
		\\&=
		\int_{B_{R}\setminus B} 
			-\Delta u \psi
		+
		\int_{\partial B_{R}\setminus B} 
			\psi \partial_{\n} u 
			-
			u \partial_{\n} \psi 
		+
		\int_{B_{R}\setminus B}
			u \psi.
	\end{alignat*}
	Notando que \(-\Delta u + u = 0\) nos queda
	\begin{displaymath}
		I_2 
		= 
		\underbrace{
		\int_{\partial B_{R}\setminus B} 
			\psi \partial_{\n} u 
		}_{J_2}
		- 
		\underbrace{
		\int_{\partial B_{R}\setminus B} 
			u \partial_{\n} \psi 
		}_{J_1}.
	\end{displaymath}

	Comenzamos por \(J_1\). 
	\begin{displaymath}
		\abs{J_1}
		\le
		\norm{\partial_{\n} \psi}_{\infty} 
		\abs{ \int_{\partial B_{R} \setminus B} u}.	
	\end{displaymath}
	Ya vimos que la integral sobre la bola se va a cero si \(\epsilon \to 0\),
	en particular lo hace su borde. Además, el borde
	exterior es donde \(\psi\) se anula, así que \(J_1 \to 0\)

	Para \(J_2\). Primero notamos el borde exterior es donde \(\psi\) se anula,
	así que solo quedamos con el borde interior. Ahora 
	debemos observar que \(\n(x) = - x/\abs{x}\) y \(\partial_{x_i}
	u(x) = -\frac{e^{-\abs{x}} (\abs{x} + 1)}{4 \pi \abs{x}^2}
	\frac{x_i}{\abs{x}}\), así que
	\begin{displaymath}
		\partial_{\n} u
		=
		\nabla u \cdot \n
		=
		\sum_{i=1}^{3} 
			\frac{e^{-\abs{x}} (\abs{x} + 1)}{4 \pi \abs{x}^2}
			\frac{x_{i}^2}{\abs{x}^2}
		=
		\frac{e^{-\abs{x}} (\abs{x} + 1)}{4\pi \abs{x}^2}.
	\end{displaymath}
	Luego, 
	\begin{alignat*}{3}
		J_2
		&=
		\int_{\partial B}
			\frac{e^{-\abs{x}} (\abs{x} + 1)}{4\pi \abs{x}^2}
			\psi(x)
			\, dS(x)
		\\&=
		\frac{e^{-\epsilon} (\epsilon + 1)}{4\pi \epsilon^2}
		\int_{\partial B}
			\psi
		\\&=
		e^{-\epsilon} (\epsilon + 1)
		\fint_{\partial B}
			\psi
		\\&
		\xrightarrow{\epsilon \to 0} \psi(0).
	\end{alignat*}
	En conclusión, tenemos que 
	\begin{displaymath}
		\langle -\Delta f_{u} + f_{u}, \psi \rangle 
		=
		\psi(0).
	\end{displaymath}
	Que es lo que queriamos probar.
\end{Solucion}

% Problema 2
\begin{Problema}
	Sea \(u\) una función armónica en un conjunto \(\Omega\) abierto y conexo en
	\(\R^n\). Sean \(a \le b \le c\), con \(b^2 = ac\) y suponga que
	\(\overline{B(x_0, c)} \subset \Omega\). Demuestre que 
	\begin{displaymath}
		\int_{\abs{w} = 1} 
			u(x_0 + a w) u(x_0 + cw)
			\, dw
		=
		\int_{\abs{w} = 1}
			u(x_0 + bw)^2
			\, dw.
	\end{displaymath}
	Deduzca de lo anterior que si \(u\) es constante en algún subconjunto
	abierto de \(\Omega\), entonces \(u\) es constante en \(\Omega\).

	\noindent\textbf{Sugerencia:} Considere \(a\) como un parámetro.
\end{Problema}

\begin{Solucion}
	Consideremos \(f(a) = \int_{\abs{w}=1} u(x_0 + aw) u(x_0 + \frac{b^2}{a}w)
	dw\). De esta forma si \(f\) es una función constante tendremos lo pedido.
	En específico, probaremos que \(f' = 0\).

	Como \(u\in L^{1}\big(\overline{B(x_0, c)}\big)\) y es de hecho \(\mathcal{C}^2\) 
	por ser armónica.
	En particular, se cumple el TCD, así que podemos intercambiar límites entre
	la derivadas y la integral. Especificamente, se tiene que:
	\begin{align*}
		f'(a)
		&=
		\int_{\abs{w} = 1}
		(\nabla u(x_0 + aw)\cdot w) u(x_0 + \frac{b^2}{a} w)
		-
		(\nabla u(x_0 + \frac{b^2}{a} w) \cdot w) \frac{b^2}{a^2} u(x_0 + aw)
		\, dw
		\\&=
		\underbrace{
		\int_{\abs{w} = 1}
			u(x_0 + \frac{b^2}{a} w)
			(\nabla u(x_0 + aw)\cdot w)
		\, dw
		}_{I_1}
		-
		\underbrace{
		\int_{\abs{w} = 1}
			\frac{b^2}{a^2} u(x_0 + aw)
			(\nabla u(x_0 + \frac{b^2}{a} w) \cdot w)
		\, dw
		}_{I_2}.
	\end{align*}
	Considerando la región de integración como la bola unitaria vemos que
	\(w\) juega el papel del vector normal exterior, sin embargo, en \(I_1\) por
	ejemplo \(\nabla u(x_0 + aw)\cdot w\) no es la derivada normal, pues falta un 
	factor \(a\). Lo análogo pasa para \(I_2\), en este caso falta el factor 
	\(b^2/a\).
	Agregando dichos factores, podemos aplicar una identidad de Green.
	\begin{align*}
		I_1
		&=
		\frac{1}{a} 
		\int_{\abs{w} < 1}
			\nabla u(x_0 + \frac{b^2}{a} w)
			\cdot
			\nabla u(x_0 + a w)
			+
			u(x_0 + \frac{b^2}{a} w)
			\Delta u(x_0 + aw) 
		\, dw
		\\&=
		\frac{1}{a} 
		\int_{\abs{w} < 1}
			\nabla u(x_0 + \frac{b^2}{a} w)
			\cdot
			\nabla u(x_0 + a w)
		\, dw
		\\
		I_2 
		&=
		\frac{1}{a} 
		\int_{\abs{w} < 1}
			\nabla u(x_0 + \frac{b^2}{a} w)
			\cdot
			\nabla u(x_0 + a w)
			+
			u(x_0 + aw) 
			\Delta u(x_0 + \frac{b^2}{a} w)
		\, dw
		\\&=
		\frac{1}{a} 
		\int_{\abs{w} < 1}
			\nabla u(x_0 + \frac{b^2}{a} w)
			\cdot
			\nabla u(x_0 + a w)
		\, dw.
	\end{align*}
	Luego,
	\begin{displaymath}
		f'(a)
		=
		I_1 - I_2
		=
		0.
	\end{displaymath}
	Como \(a\) era un real positivo arbitrario, concluimos que \(f\) es constante
	(en los positivos) y por lo tanto
	\begin{displaymath}
		f(a) 
		=
		\int_{\abs{w} = 1}
			u(x_0 + aw)
			u(x_0 + cw)
			\,dw
		=
		\int_{\abs{w} = 1}
			u(x_0 + bw)^2
			\,dw
		=
		f(b).
	\end{displaymath}

	\begin{wrapfigure}{r}{.3\textwidth}
		\hspace{2em}
		\begin{tikzpicture}[scale=0.8]
			\coordinate (x1) at (0,0);
			\coordinate (x2) at (-160:0.8);
			\draw (x1) circle(2) node[fill=white] {\(x_1\)};
			\draw (x1) -- (45:2) node[right] {\(r_1\)};
			\draw (x1) circle(1);
			\draw (x1) -- (-45:1) node[right] {\(r_1'\)};
			\draw[dashed] (x2) circle(.7) node {\(x_2\)};
			\draw (x2) -- +(280:.7) node[below] {\(r_2\)};
		\end{tikzpicture}
	\end{wrapfigure}
	Para ver lo segundo, denotemos por \(A\) a una componente conexa de
	\(\Omega\) donde
	la función \(u\) se anula. Probaremos que \(A\) es todo \(\Omega\) 
	mostrando que es \(A\) es cerrado. En efecto, si no lo fuera,
	existiría un punto límite
	\(x_1 \in \partial A\) tal que \(x_1 \not\in A\). Sea \(r_1 > 0\) algún real
	positivo. Consideremos la bola de radio \(r_1' <
	r_1/2\) centrada en \(x_1\) y sea \(x_2 \in A\cap B(x_1, r_1')\). Se sigue que 
	para \(r_2 < r_1'\) se cumple que \(B(x_2, r_2) \subset A \cap B(x_1, r_1)\). 
	Por la igualdad anterior, tendremos que
	\begin{displaymath}
		\int_{\abs{w} = 1}
		u(x_1 + r_2 w) u(x_1 + r_1' w)
		=
		\int_{\abs{w} = 1}
		u^2 (x_1 + \sqrt{r_1' r_2} w).
	\end{displaymath}
	La integral de la izquierda da cero, puesto que la bola de radio \(r_2\)
	está completamente contenida en \(A\). Se sigue que \(u\) se anula
	sobre \(\partial B(x_2, \sqrt{r_1' r_2})\). Si tomamos \(r_2 \to r_1'\), por
	continuidad tendremos que 
	\begin{displaymath}
		\int_{\abs{w} = 1}
		u^2 (x_1 + r_1' w)
		= 0.
	\end{displaymath}
	De lo que se sigue que \(u = 0\) en \(\partial B(x_2, r_1')\) y por lo tanto
	\(x_1 \in A\) (contradicción). Se concluye así que \(A\) es abierto y 
	cerrado a la vez en un
	conjunto conexo, por lo que \(A = \Omega\).
\end{Solucion}

% Problema 3
\begin{Problema}
	Sea \(\R^{n}_{+} = \left\{ x\in \R^{n} \mid x_n > 0\right\}\). Suponga que
	\(u \in \mathcal{C}^{2}(\overline{\R^{n}_{+}})\) es armónica en
	\(\R^{n}_{+}\) y tal que \(u = 0\) sobre \(\partial \R^{n}_{+} = \left\{
		x\in \R^n \mid x_n = 0\right\}\). Demuestre que la función
	\begin{displaymath}
		v(x)
		\coloneqq
		\begin{cases}
			u(x) &,\text{Si } x_n \ge 0\\
			-u(x_1, \dots, x_{n-1}, -x_n) &,\text{Si } x_n < 0
		\end{cases}
	\end{displaymath}
	es de clase \(\mathcal{C}^{2}\) y armónica en \(\R^n\).
\end{Problema}

\begin{Solucion}
	\begin{wrapfigure}{r}{.3\textwidth}
		\vspace{-2em}\hspace{2em}
		\begin{tikzpicture}[scale=4]
			\draw[->, thin] (-.1,0) -- (1,0) node[right] {\(x'\)};	
			\draw[->, thin] (0,-.25) -- (0,.5) node[above] {\(x_n\)};	

			\draw[<-, red, thick] (.25,0) arc(180:0:.25) 
				node[midway,above] {\(\partial B(x_0, r)^{+}\)};
			\draw[->, blue, thick] (.25,0) arc(180:360:.25)
				node[midway,below] {\(\partial B(x_0, r)^{-}\)};
		\end{tikzpicture}
	\end{wrapfigure}
	Primero notemos que \(v\) es armónica en \(\R^n\setminus\R^{n}_{0}
	\) donde \(\R^n_{0} = \left\{(x', x) \in \R^{n-1}\times 0\right\}\), 
	por lo que para probar el resultado basta ver que es armónica en 
	\(\R^{n}_{0}\). Como \(u\in\mathcal{C}^2\) hasta la frontera, es claro 
	que \(v\) es continua (de hecho, \(\mathcal{C}^2\)) en \(\R^{n}\). 
	Veremos que \(v\) cumple la propiedad de la media y con eso 
	concluiremos que es armónica.

	Sea \(x_0 \in \R^{n}_{0}\) y \(r>0\) arbitrario. Luego, 
	\begin{displaymath}
		\int_{\partial B(x_0, r)}
			v(y) \, dS(y)
		=
		\int_{\partial B(x_0, r)^{+}}
			v(y) \, dS(y)
		+
		\int_{\partial B(x_0, r)^{-}}
			v(y) \, dS(y)
	\end{displaymath}
	donde \(\partial B(x_0, r)^{\pm}\) es el borde superior/inferior de la bola.
	Notando que 
	\begin{displaymath}
		\int_{\partial B(x_0, r)^{-}}
			v(y) \, dS(y)
		=
		\int_{\partial B(x_0, r)^{+}}
			-u(y) \, dS(y),
	\end{displaymath}
	vemos que la integral sobre el borde de la bola se anula. 
	Es decir, para cualquier bola con centro en \(\R^{n}_{0}\) se 
	cumple que la integral sobre el borde da cero. 
	En particular vale para las integrales promedio, por lo tanto, vale la
	propiedad de la media, pues
	\begin{displaymath}
		0 
		= 
		v(x_0) 
		=
		\fint_{\partial B(x_0, r)} v(y) \, dS(y)
		,\quad
		\forall x_0 \in \R^{n}_{0}
		\quad
		\forall r > 0\footnote{Mientras sigan en el dominio.}
	\end{displaymath}
	Por lo tanto, \(v\) es armónica en \(\R^n\) y de paso
	\(\mathcal{C}^{2}\).
\end{Solucion}

% Problema 4
\begin{Problema}
	Sea \(u\) una función armónica en un conjunto \(\Omega\) de \(\R^n\) abierto
	y conexo. Suponga que \(\partial \Omega\) posee una porción abierta
	\(\Gamma\) (en la topología relativa) de clase \(\mathcal{C}^1\), tal que
	\(u = \partial_{\n} u = 0\) en \(\Gamma\). Demuestre que \(u\) es
	idénticamente nula en \(\Omega\). 
\end{Problema}
\begin{Solucion}
	\begin{wrapfigure}{l}{0.3\textwidth}
		\centering
		\begin{tikzpicture}[scale=3]
			\draw[->, thin] (-.1,0) -- (1,0) node[right] {\(x'\)};	
			\draw[->, thin] (0,-.1) -- (0,1) node[above] {\(x_n\)};	
			\draw (-.1,.2) -- (1,.75)
				node[below] {\(\partial\Omega\)};
			\draw[very thick, red] 
				(.25, .375) -- (.75,.625)
				node[xshift=-1.4cm, yshift=-1cm] {\(\Gamma\)};
			\draw[very thick, red] 
				(.34, .42) [rotate=26.56] arc(180:0:.2)
				node[midway, right] {\(B_r^{+}\)};
		\end{tikzpicture}
	\end{wrapfigure}
	Sea \(x_0 \in \Gamma\). Para \(0 < r < R \coloneqq \textrm{dist}(x, \partial\Omega
	\setminus \Gamma)\), consideremos la
	región \(B_r \coloneqq B(x_0, r) \cap \Omega\). Luego, por la primera
	fórmula de Green tenemos que
	\begin{displaymath}
		\int_{B_r} \Delta u(y) \, dy
		=
		\int_{\partial B_r} \partial_{\n} u(y) \, dS(y).
	\end{displaymath}
	Como \(u\) es armónica en \(\Omega\), la integral de la izquierda es nula.
	Por otro lado, la frontera de \(B_r\) se puede descomponer en la porción que
	toca a \(\Gamma\) y la que está contenida en \(\Omega\). La primera es nula
	por hipótesis. Si denotamos a \(\partial B_r^{+}\) la región del borde que no
	es \(\Gamma\), nos queda
	\begin{displaymath}\label{cero}
		\int_{\partial B_r^{+}} \partial_{\n} u(y) \, dS(y) = 0.
		\tag{\(\dag\)}
	\end{displaymath}
	Si ahora usamos la tercera fórmula de Green, obtenemos que
	\begin{displaymath}
		\int_{B_r} \abs{\nabla u}^2 \, dy
		=
		\int_{\partial B_r^{+}} u \partial_{\n} u \, dS(y)
		\le
		\sup_{B_r} \norm{u}
		\int_{\partial B_r^{+}} \partial_{\n} u \, dS(y)
		= 0,
	\end{displaymath}
	de donde deducimos que \(\nabla u = 0\) en \(B_r\). Ahora, tenemos una 
	función constante y armónica en un abierto \(B_r\) que es conexo y acotado, 
	así que aplica el principio del máximo. En particular tenemos que \(u = 0\)
	en \(B_{r}\) (pues debe ser menor al máximo y mayor al mínimo, que es cero).
	De esta forma, por el Problema 2 concluimos que \(u\equiv 0\) en \(\Omega\)
	(pues es cero en una abierto contenido y \(\Omega\) es conexo).
\end{Solucion}

\end{document}
