%! tex root = T3.tex
\documentclass[11pt]{article}

% Page set up
\usepackage[margin=3cm]{geometry}

% Document font and symbols
%\usepackage{mathptmx} % Times New Roman
\usepackage{amsmath,amsfonts,amsthm,amssymb}
\usepackage{esint}

% Language formatting
\usepackage[utf8]{inputenc}
\usepackage[T1]{fontenc}
\usepackage[spanish, es-noshorthands]{babel}

% Graph and Drawings
\usepackage{tikz,tikz-cd,float}
\usepackage{wrapfig}
\usetikzlibrary{babel}
\usepackage{pgfplots}
\usepackage[framemethod=default]{mdframed}
\usepackage{framed}

% Tools
\usepackage{multicol}
\usepackage{array}
\usepackage{hyperref}
\usepackage{xcolor}
\usepackage{etoolbox,mathtools}
\usepackage[shortlabels]{enumitem}

% Mdframed template
\mdfsetup{innertopmargin=5pt,%
	frametitlealignment=\raggedright,%
	frametitlefont=\texttt,%
	linewidth=2pt,%
	topline=false,rightline=false, bottomline=false,%
	frametitleaboveskip=\dimexpr-1\ht\strutbox\relax,}

% Problem Env
\newcounter{ProblemCounter}
\newenvironment{Problema}[1][]{%
	\vspace{1em}
	\refstepcounter{ProblemCounter}%
	\noindent\hspace{-12ex}{\texttt{PROBLEMA~\theProblemCounter}~}
	%\begin{tikzpicture}
	%	\draw[thick] (0,0) -- (\linewidth, 0)
	%	node[midway, fill=white] 
	%	{\texttt{PROBLEMA~\theProblemCounter}};
	%\end{tikzpicture}
	\fontfamily{ptm}\selectfont
	}{
	\ \newline
	%\begin{tikzpicture}
	%	\draw[thick] (0,0) -- (\linewidth, 0);
	%\end{tikzpicture}
	%\vspace{1ex}
	}

% Solution Env
\newenvironment{Solucion}[1][]
{%
	\vspace{1ex}
	\noindent\hspace{-10ex}{{\ttfamily SOLUCIÓN}~}
}%
{%
	%\hfill\(\blacksquare\)
}

% -- Aligned Cases
\newenvironment{caligned}
	{\begin{cases} \begin{aligned}}
	{\end{aligned} \end{cases}}

% Resize abs and norm
\DeclarePairedDelimiter{\abs}{\lvert}{\rvert}
\DeclarePairedDelimiter{\norm}{\|}{\|}
\makeatletter
\let\oldabs\abs
\def\abs{\@ifstar{\oldabs}{\oldabs*}}
\let\oldnorm\norm
\def\norm{\@ifstar{\oldnorm}{\oldnorm*}}
\makeatother

% Shortcuts
\def\V{\mathbb{V}}
\def\N{\mathbb{N}}
\def\Z{\mathbb{Z}}
\def\Q{\mathbb{Q}}
\def\R{\mathbb{R}}
\def\C{\mathbb{C}}
\def\CC{\mathcal{C}}
\def\F{\mathbb{F}}
\def\FF{\mathcal{F}}
\def\A{\mathbb{A}}
\def\n{\hat{\textbf{n}}}
\def\loc{\textrm{loc}}
\def\supp{\textrm{supp}\,}

\newcommand{\header}[2]
{
\begin{minipage}{.15\textwidth}
	\includegraphics[width=\textwidth,height=\textheight,keepaspectratio]{LogoUC}
\end{minipage}
\hspace{1em}
\begin{minipage}{.75\textwidth}
	\vspace{2em}
	{\scshape
	Pontificia Universidad Católica de Chile\\
	Facultad de Matemáticas\\
	Docente: Carlos Román\\
	Ayudante: Santiago González
	}
\end{minipage}

\medskip

\begin{center}
	{\textbf{MAT2505 - Ecuaciones Diferenciales Parciales}} \\[1ex]
	{{\large Tarea #1 - Sebastián Sánchez}}
\end{center}

\medskip
}


\begin{document}

\header{3}{}

\begin{Problema}
	Sea \(T>0\), \(U\subset\R^n\) un abierto y
	\(u\in\CC^{(1,2)}(U_{T})\) una solución de la ecuación
	del calor en \(U_{T}\).
	Pruebe que para todo par de enteros \(k,\ell\) no negativos
	existen constantes \(C = C(k,\ell,n)\) tales que 
	\begin{displaymath}
		\max_{C(x,t;r/2)} \abs{D^{k}_{x} D^{\ell}_{t} u}
		\le
		\frac{C}{r^{k+2\ell}} \max_{C(x,t;r)} \abs{u},
	\end{displaymath}
	para todos los cilindros \(C(x,t;r/2) \subset C(x,t;r) \subset
	U_{T}\).

	Deduzca que si \(u\in \CC^{(1,2)}(\R\times\R^n)\) es una solución
	de la ecuación del calor tal que \(u = \mathcal{O}(\abs{t}^{\ell}
	+ \abs{x}^{k})\), entonces \(u\) es un polinomio de grado a lo
	más \(k + 2\ell\).
\end{Problema}
\begin{Solucion}
	Por el teorema de estimación sobre las derivadas tenemos que
	\begin{displaymath}
		\max_{C(r/2)} \abs{D^{\alpha}_{x} D^{\ell}_{t} v}
		\le
		\frac{C_{k,\ell}}{r^{2\ell + k + n+2}} 
		\norm{u}_{L^1(C(r))}.
	\end{displaymath}
	Dado que \(u\) es continua y \(C(r)\) es compacto, \(u\) alcanza
	su máximo. Se sigue que
	\begin{displaymath}
		\norm{u}_{L^1(C(r))}
		\le
		r^{n+2} \alpha(n)
		\max_{C(r)} \abs{u}.
	\end{displaymath}
	donde \(r^{n+2}\alpha(n)\) es el volumen del cilindro \(C(r)\).
	Juntando ambas desigualdades se sigue el resultado:
	\begin{displaymath}
		\max_{C(r/2)} \abs{D^{\alpha}_{x} D^{\ell}_{t} u}
		\le
		\frac{C_{k,\ell} \alpha(n)}{r^{2\ell + k}} 
		\max_{C(r)} \abs{u}
		=
		\frac{C_{k,\ell,n}}{r^{2\ell + k}} 
		\max_{C(r)} \abs{u}.
	\end{displaymath}

	Supongamos ahora que \(u\in\CC^{(1,2)}(\R_{\ge0}\times\R^n)\).
\end{Solucion}

\begin{Problema}
	Sea \(\Omega \subset\R^n\) un abierto acotado con frontera
	de clase \(\CC^{1}\) y \(u\in H^{1}(\Omega)\).
	\begin{enumerate}[label=(\alph*)]
		\item
		Demuestre que \(\abs{u} \in H^{1}(\Omega)\) y que
		\begin{displaymath}
			\nabla \abs{u}(x)
			=
			\text{signo}(u(x))\, \nabla u(x)
			\qquad
			\text{c.t.p. en } \Omega, 
		\end{displaymath}
		donde \(\text{signo}(s) = 1, 0, -1\) si \(s>0, s=0\) o
		\(s<0\), respectivamente.

		\item
		Demuestre que \(u^{+}(x) \coloneqq \max \left\{ u(x), 0
		\right\}\) y \(u^{-}(x) \coloneqq \min \left\{ u(x), 0 \right\}\)
		pertenecen a \(H^{1}(\Omega)\).

		\textit{Sugerencia: considere la sucesión de funciones
		\(u_{\epsilon}(x) \coloneqq \sqrt{\epsilon^2 + u(x)^2} -
		\epsilon\)}.
		
		\item
		Demuestre que \(\nabla u(x) = 0\) para casi todo \(x\in
		\Omega\) tal que \(u(x) = 0\), y que si \(u\in
		H^{1}_{0}(\Omega)\) entonces \(\abs{u}, u^{\pm} \in
		H^{1}_{0}(\Omega)\).
	\end{enumerate}
\end{Problema}
\begin{Solucion}
\begin{enumerate}[label=(\alph*)]
	\item
	Primero notemos que \(u\in H^{1}(\Omega)\) si y solo si
	\begin{displaymath}
		\left(
		\sum_{\abs{\alpha} \le 1} \norm{D^{\alpha} u}_{L^2(\Omega)}^{2}
		\right)^{1/2}
		< \infty.
	\end{displaymath}
	Dado que \(\alpha = (0,0,\dots,0), (1,0,\dots,0), \dots, (0,
	\dots, 1)\) son las
	únicas posibilidades para \(\alpha\), podemos reescribir la
	condición como
	\begin{displaymath}
		\norm{u}_{L^2(\Omega)}^{2} 
		+ 
		\norm{\abs{\nabla u}}_{L^2(\Omega)}^{2}
	   	< \infty.
	\end{displaymath}
	De esta forma, solo debemos mostrar que \(\abs{u}\) está
	controlado por las estimaciones de \(u\). 
	
	Probaremos el resultado para \(u\in\CC^{\infty}\cap H^{1}(\Omega)\) y luego usaremos
	densidad para extenderlo. Sea entonces \(u\in\CC^{\infty}(\Omega)\cap
	H^{1}(\Omega)\).
	Tenemos que
	\begin{align*}
		\norm{\abs{u}}_{L^2(\Omega)}^{2}
		=
		\int_{\Omega} \abs{u}^{2}
		&=
		\int_{\Omega} u^{2}
		=
		\norm{u}_{L^2(\Omega)}^{2}
		\intertext{y (tomando la derivada c.t.p.)}
		\partial_{x_i} \abs{u}
		=
		\text{signo}(u) \partial_{x_i} u
		&\Rightarrow
		\nabla \abs{u}
		=
		\text{signo}(u) \nabla u
		\\
		&\Rightarrow
		\norm{\abs{\nabla \abs{u}}}_{L^{2}(\Omega)}^{2}
		=
		\int_{\Omega} \abs{\nabla \abs{u}}^{2}
		\\&=
		\int_{\Omega} \abs{\text{signo}(u) \nabla u}^{2}
		=
		\int_{\Omega} \abs{\nabla u}^{2}
		=
		\norm{\abs{\nabla u}}_{L^2(\Omega)}^{2}.
	\end{align*}
	Así que \(\abs{u} \in H^{1}(\Omega)\) para funciones
	\(u\in \CC^{\infty}(\Omega)\cap H^{1}(\Omega)\). Si \(u\in
	H^{1}(\Omega)\), existe una sucesión
	de funciones suaves\footnote{Aproximación global por funciones
	suaves} \((u_m)_{m\in\N}\) tal que \((u_m) \to u\) en
	\(H^{1}(\Omega)\). Luego,
	\begin{displaymath}
		\norm{\abs{u}}_{L^{2}(\Omega)}^{2}
		=
		\norm{\abs{\lim_{m\to\infty} u_m}}_{L^{2}(\Omega)}^{2}
		=
		\lim_{m\to \infty} \norm{\abs{u_m}}_{L^{2}(\Omega)}^{2}
		=
		\lim_{m\to \infty} \norm{u_m}_{L^{2}(\Omega)}^{2}
		=
		\norm{u}_{L^{2}(\Omega)}^{2}.
	\end{displaymath}
	De manera análoga concluimos que \(\norm{\abs{\nabla
	\abs{u}}}_{L^2(\Omega)}^{2} = \norm{\abs{\nabla u}}_{L^{2}(\Omega)}^{2}\).

	\item
	Debemos probar que \(\norm{u^{\pm}}_{L^{2}(\Omega)}^{2} +
	\norm{D u^{\pm}}_{L^{2}(\Omega)} < 0\). Para ello necesitamos obtener la
	derivada débil de \(u^{\pm}\).

	Sea \(\Omega^{+}\) el dominio donde \(u(x) \ge 0\) o
	equivalentemente \(u^{+} = u\). Un candidato natural a derivada
	débil es la función
	\begin{displaymath}
		v = \begin{cases}
			D u &, \text{ en }\Omega^{+}\\
			0 &, \text{ fuera de }\Omega^{+}.
		\end{cases}
	\end{displaymath}
	Y en efecto, se tiene que
	\begin{displaymath}
		\int_{\Omega} u^{+} D \psi
		=
		\int_{\Omega^{+}} u D \psi
		=
		-\int_{\Omega^{+}} D u \psi
		=
		-\int_{\Omega} v\psi
		\quad\forall \psi \in \CC^{\infty}_{C}(\Omega).
	\end{displaymath}
	Así que podemos denotar con propiedad \(Du^{+} = v\).
	De manera análoga definimos \(Du^{-}\). 

	Para finalizar, notemos que
	\begin{displaymath}
		\norm{u^{\pm}}_{L^2(\Omega)}^{2} 
		\le 
		\norm{u}_{L^{2}(\Omega)}^2 < \infty
		\quad\text{ y }\quad
		\norm{Du^{\pm}}_{L^2(\Omega)}^2 
		\le 
		\norm{D u}_{L^2(\Omega)}^{2} < \infty.
	\end{displaymath}
	Así que \(u^{\pm} \in H^{1}(\Omega)\).

	\item
	Supongamos que \(u = 0\) c.t.p. en \(\Omega\). Luego,
	\begin{displaymath}
		0
		=
		\int_{\Omega} u \, D\psi
		=
		-\int_{\Omega} Du \, \psi
		\qquad\forall\psi\in\CC^{\infty}_{C}(\Omega)
	\end{displaymath}
	Así deducimos que \(Du = 0\) c.t.p. en \(\Omega\). Por otro
	lado, si \(u\in H^{1}_{0}(\Omega)\), por los apartados anteriores
	tenemos que todas las estimaciones sobre \(\abs{u}\) y \(u^{\pm}\)
	están controladas por \(u\), así que necesariamente obtienen la
	regularidad de esta.
\end{enumerate}
\end{Solucion}

\begin{Problema}
	Sea \(\Omega\subset\R^n\) un abierto conexo acotado con frontera
	\(\CC^1\). Demuestre la desigualdad de tipo Poincaré
	\begin{equation}\label{eq1}
		\norm{v}_{L^{2}(\Omega)}
		\le
		C \left( 
			\norm{\nabla v}_{L^{2}(\Omega)}
			+
			\norm{T(v)}_{L^{2}(\partial\Omega)}
		\right)
		\quad\forall v\in H^{1}(\Omega).
	\end{equation}
	donde \(T(v)\) denota la traza de \(v\) sobre \(\partial\Omega\).

	\textit{Sugerencia: utilice el argumento de compacidad visto en
	clases. Puede usar \textbf{sin demostrar} que el operador de traza 
	\(T\colon H^{1}(\Omega) \to L^{2}(\partial\Omega)\) es compacto.}
\end{Problema}
\begin{Solucion}
	Supongamos que la desigualdad no vale. Entonces para cada
	\(k\in\N\) existe una función \(u_k\in H^{1}\) tal que
	\begin{displaymath}\label{star2}
		\norm{u_{k}}_{L^2(\Omega)}
		>
		k
	   	\left(
			\norm{\nabla u_{k}}_{L^2(\Omega)}
			+
			\norm{T u_{k}}_{L^{2}(\partial\Omega)}
		\right)
	\end{displaymath}
	Consideremos la sucesión \(u'_k =
	u_k/\norm{u_k}_{L^{2}(\Omega)}\). Se sigue de la desigualdad
	anterior que
	\begin{displaymath}\label{star2}
	   	\left(
			\norm{\nabla u'_k}_{L^2(\Omega)}
			+
			\norm{T u'_k}_{L^{2}(\partial\Omega)}
		\right)
		\le \frac{1}{k}.
		\tag{\(\star\star\)}
	\end{displaymath}
	En particular, se tiene que
	\begin{displaymath}
		\norm{\nabla u'_k}_{L^2(\Omega)} 
		\xrightarrow{k\to\infty} 0
	\end{displaymath}
	Más aún, dado que la sucesión \((u'_k)_{k\in\N}\) es acotada en
	\(H^{1}(\Omega)\). Dado que \(H^{1}(\Omega) \subset\subset
	L^{2}(\Omega)\), existe una subsucesión \((u'_{k_j})_{j\in\N}\)
	que es convergente a \(u \in L^{2}(\Omega)\). 

	Queremos ver que \(u\in H^{1}(\Omega)\). La
	fórmula de integración por partes nos dice que para \(\psi\in
	\CC^{\infty}_{C}(\Omega)\) se tiene que
	\begin{displaymath}
		\abs{ \int_{\Omega} u \, D\psi }
		=
		\abs{
		\lim_{k_j\to\infty} 
		\int_{\Omega} u'_{k_j}\, D\psi
		}
		=
		\abs{
		\lim_{k_j\to\infty} 
		\int_{\Omega} Du'_{k_j}\, \psi
		}
		\le
		\lim_{k_j\to\infty} C \frac{1}{k_j^2}
		\to 0.
	\end{displaymath}
	Por lo que \(u\in H^{1}(\Omega)\) con \(Du = 0\) c.t.p. en
	\(\Omega\). Como \(\Omega\) es conexo, \(u\) debe ser constante.
	Las funciones constantes son suaves hasta la frontera, así que
	\(Tu = u\vert_{\partial\Omega}\). Pero entonces
	\begin{displaymath}
		\norm{u\vert_{\partial\Omega}}_{L^{2}(\partial\Omega)} 
		= 
		\norm{Tu}_{L^2(\partial\Omega)} 
		= 
		\lim_{k_j \to\infty} \norm{Tu'_{k}}_{L^{2}(\partial\Omega}
		= 0.
	\end{displaymath}
	Y por lo tanto \(u \equiv 0\) en \(\Omega\). Esto contradice que
	\(\norm{u}_{L^{2}(\Omega)} = \lim_{k_j \to \infty}
	\norm{u'_k}_{L^{2}(\Omega)} = 1\), así que la
	desigualdad~\eqref{eq1} debe ser cierta y con ello 
	concluimos el resultado.
\end{Solucion}

\begin{Problema}
	Sea \(\Omega\subset\R^n\) un abierto conexo acotado con frontera
	\(\CC^1\) y \(f\in L^{2}(\Omega)\). Considere el problema
	\begin{equation}\label{eq2}
	\begin{caligned}
		-\Delta u &= f &&, \text{ en } \Omega\\
		\partial_{\n} u + u &= 0 &&, \text{ en } \partial\Omega.
	\end{caligned}
	\end{equation}
	Decimos que \(u\in H^{1}(\Omega)\) es una solución débil del
	problema si 
	\begin{displaymath}
		\int_{\Omega} \nabla u \, \nabla v
		+
		\int_{\partial\Omega} T(u)\, T(v)
		=
		\int_{\Omega} fv
		\quad\forall v\in H^{1}(\Omega).
	\end{displaymath}
	\begin{enumerate}[label=(\alph*)]
		\item
		Demuestre que \(u\in \CC^{2}(\overline\Omega)\) es solución
		clásica de~\eqref{eq2} si y sólo si \(u\) es solución débil
		de~\eqref{eq2}.

		\item 
		Demuestre que~\eqref{eq2} posee una única solución débil en
		\(H^{1}(\Omega)\).

		\textit{Sugerencia: utilice~\eqref{eq1}.}

		\item
		Suponga que existe \(w\in H^{2}(\Omega)\) tal que 
		\begin{displaymath}
			T(\nabla w) \cdot \n
			=
			-T(u),
		\end{displaymath}
		donde \(u\) es la única solución débil de~\eqref{eq2} en
		\(H^{1}(\Omega)\). Muestre que \(h \coloneqq u - w\) es
		solución débil de
		\begin{displaymath}
		\begin{caligned}
			-\Delta h &= g &&, \text{ en } \Omega\\
			\partial_{\n} h &= 0 &&, \text{ en } \partial\Omega,
		\end{caligned}
		\end{displaymath}
		donde \(g\) es una función en \(L^{2}(\Omega)\) tal que
		\(\int_{\Omega} g = 0\).
	\end{enumerate}
\end{Problema}
\begin{Solucion}
	\begin{enumerate}[label=(\alph*)]
	\item \framebox{\(\Rightarrow\):} Supongamos que \(u\) es solución
	clásica. Sea \(v\in C^{\infty}(\overline\Omega)\). Luego,
	\begin{align*}
		-\Delta u = f
		&\Rightarrow
		-\Delta u \, v = f v
		\\&\Rightarrow
		-\int_{\Omega} \Delta u\, v
		=
		\int_{\Omega} f\, v
		\\&\Rightarrow
		\int_{\Omega} \nabla u\cdot \nabla v
		-
		\int_{\partial\Omega} v \partial_{\n} u 
		=
		\int_{\Omega} f v
		\shortintertext{Usamos la condición de borde \(\partial_{\n} u
		= -u\).}
		\\&\Rightarrow
		\int_{\Omega} \nabla u\cdot \nabla v
		+
		\int_{\partial\Omega} v u
		=
		\int_{\Omega} f v
	\end{align*}
	Como \(u\) es continua hasta la frontera,
	\(u\vert_{\partial\Omega} = Tu\). Lo mismo pasa para \(v\). De
	esta forma tenemos la formulación buscada para \(v\) suave. Si
	\(v\in H^{1}(\Omega)\), consideremos una aproximación
	\((v_n)_{n\in\N}\) por funciones suaves. De esta forma tenemos que
	\begin{align*}
		&\lim_{n\to \infty}
		\left(
		\int_{\Omega} \nabla u\cdot \nabla v_n
		+
		\int_{\partial\Omega} Tv_n\, Tu
		=
		\int_{\Omega} f v_n
		\right)
		\\&\Rightarrow	
		\int_{\Omega} \lim_{n\to\infty} \nabla u\cdot \nabla v_n
		+
		\int_{\partial\Omega} \lim_{n\to\infty} Tv_n\, Tu
		=
		\int_{\Omega} \lim_{n\to\infty} f v_n
		\\&\Rightarrow	
		\int_{\Omega} \nabla u\cdot \nabla v
		+
		\int_{\partial\Omega} Tv\, Tu
		=
		\int_{\Omega} f v.
	\end{align*}
	Donde usamos que el operador de traza es continuo.

	\framebox{\(\Leftarrow\):} Supongamos que vale la formulación
	débil. 
	\begin{displaymath}
		\int_{\Omega} \nabla u \nabla v
		+
		\int_{\partial\Omega} Tu\, Tv
		=
		\int_{\Omega} f\,v
		\quad\forall v\in H^{1}(\Omega).
	\end{displaymath}
	Como \(u\in\CC^{2}(\overline\Omega)\) tenemos que
	\begin{displaymath}
		\int_{\Omega} \nabla u \, \nabla v
		=
		\int_{\Omega} (-\Delta u)\, v
		+
		\int_{\Omega} (\partial_{\n} Tu) \, Tv
		\quad\forall v\in H^{1}(\Omega).
	\end{displaymath}
	Reemplazando esto en la formulación débil nos deja con
	\begin{displaymath}
		\int_{\Omega} (-\Delta u)\, v
		+
		\int_{\partial\Omega} (\partial_{\n} Tu + Tu)\, Tv
		=
		\int_{\Omega} fv
		\quad\forall v\in H^{1}(\Omega).
	\end{displaymath}
	Consideremos
	\(\Omega_{\epsilon} \coloneqq \left\{ x\in\Omega\colon
	\textrm{dist}(x,\partial\Omega) > \epsilon \right\}\). Luego,
	\begin{displaymath}
		\int_{\Omega_{\epsilon}} (-\Delta u)\, v
		=
		\int_{\Omega_{\epsilon}} fv
		\qquad 
		\forall v\in \CC^{\infty} \text{ tal que } 
		\supp(v) \subset \Omega_{\epsilon}.
	\end{displaymath}
	Se sigue que \(-\Delta u = f\) c.t.p. en \(\Omega_{\epsilon}\)
	pues podemos aproximar funciones en \(H^{1}(\Omega_{\epsilon})\) por las 
	suaves.
	Dado que \(-\Delta u\) es continua (por hipótesis), 
	la igualdad se mantiene en 
	\(\Omega_{\epsilon}\). Tomando \(\epsilon\to 0\) tenemos que la
	igualdad vale en todo \(\Omega\). En particular se tiene que
	\begin{displaymath}
		\int_{\partial\Omega} (\partial_{\n} Tu + Tu)\, Tv
		=
		0
		\quad\forall v\in H^{1}(\Omega).
	\end{displaymath}
	De esta forma, \(u\) resuelve el
	sistema \(-\Delta u = f\) en \(\Omega\) y \(\partial_{\n} u + u =
	0\) en \(\partial\Omega\).

	\item Sean \(v\) y \(u\) en \(H^{1}\) dos soluciones. Luego,
	para cualquier \(\psi \in H^{1}\) (restando ambas formulaciones)
	tenemos que
	\begin{displaymath}\label{star1}
		\int_{\Omega} (\nabla v - \nabla u) \nabla \psi
		+ 
		\int_{\partial\Omega} (Tv - Tu) T\psi
		= 0.
		\tag{\(\star\)}
	\end{displaymath}
	Por~\eqref{eq1} tenemos que
	\begin{displaymath}
		\norm{v - u}_{L^{2}(\Omega)}
		\le
		C
		\left(
			\norm{\nabla v - \nabla u}_{L^2(\Omega)}
			+
			\norm{Tv - Tu}_{L^2(\partial\Omega)}
		\right)
		= 0.
	\end{displaymath}
	Donde la última igualdad se obtiene poniendo 
	\(\psi = u - v \in H^{1}(\Omega)\) en~\eqref{star1}. De esta forma, dado
	que las funciones y sus derivadas coinciden c.t.p. en \(\Omega\),
	concluimos que \(u = v\) en \(H^{1}(\Omega)\).

	\item \framebox{No me salió}
	Debemos probar la igualdad
	\begin{displaymath}\label{PD}
		\int_{\Omega} \nabla (u-w)\, \nabla v
		+
		\int_{\partial\Omega} T(u-w)\, Tv
		=
		\int_{\Omega} g\,v
		\quad\forall v\in H^{1}(\Omega).
		\tag{PD}
	\end{displaymath}

	Sabemos \(u\) es solución débil de~\eqref{eq2}, por lo tanto:
	\begin{displaymath}\label{dato1}
		\int_{\Omega} \nabla u\, \nabla v
		+
		\int_{\partial\Omega} Tu\, Tv
		=
		\int_{\Omega} f\, v
		\quad\forall v\in H^{1}(\Omega).
		\tag{\(\dag1\)}
	\end{displaymath}
	Además, tenemos que
	\begin{displaymath}\label{dato2}
		\int_{\Omega} g = 0
		\tag{\(\dag2\)}
	\end{displaymath}
	y
	\begin{displaymath}\label{dato3}
		\partial_{\n} T(\nabla w) = -Tu.
		\tag{\(\dag3\)}
	\end{displaymath}
	Por otro lado, \(w\in H^{2}(\Omega)\), así que
	\begin{displaymath}\label{dato4}
		\int_{\Omega} \nabla w\, \nabla v
		=
		\int_{\Omega} -\Delta w\, v
		+
		\int_{\partial\Omega} \partial_{\n} Tw\, Tv
		\quad\forall v\in H^{1}(\Omega).
		\tag{\(\dag4\)}
	\end{displaymath}
	Notar que \(w\in H^{1}(\Omega)\), así que poniendo \(v=w\)
	en~\eqref{dato1} tenemos que
	\begin{displaymath}\label{dd1}
		\int_{\Omega} \nabla u\, \nabla w
		+
		\int_{\partial\Omega} Tu\, Tw
		=
		\int_{\Omega} f\, w
		\tag{\(\ddag1\)}
	\end{displaymath}
	Reemplazando \(v = u\) en~\eqref{dato4} y usando~\eqref{dato2}
	\begin{displaymath}\label{dd2}
		\int_{\Omega} \nabla w\, \nabla u
		=
		\int_{\Omega} -\Delta w\, u
		-
		\int_{\partial\Omega} \abs{Tu}^2
		\tag{\(\ddag2\)}
	\end{displaymath}
	Juntando~\eqref{dd1} y~\eqref{dd2} tenemos que
	\begin{displaymath}
		\int_{\Omega}
			-\Delta w\, u
		+
		\int_{\partial\Omega}
			(Tw - Tu) Tu
		=
		\int_{\Omega} f w
	\end{displaymath}



	\end{enumerate}
\end{Solucion}

\end{document}
