%! tex root = T2.tex
\documentclass[11pt]{article}

% Page set up
\usepackage[margin=3cm]{geometry}

% Document font and symbols
%\usepackage{mathptmx} % Times New Roman
\usepackage{amsmath,amsfonts,amsthm,amssymb}
\usepackage{esint}

% Language formatting
\usepackage[utf8]{inputenc}
\usepackage[T1]{fontenc}
\usepackage[spanish, es-noshorthands]{babel}

% Graph and Drawings
\usepackage{tikz,tikz-cd,float}
\usepackage{wrapfig}
\usetikzlibrary{babel}
\usepackage{pgfplots}
\usepackage[framemethod=default]{mdframed}
\usepackage{framed}

% Tools
\usepackage{multicol}
\usepackage{array}
\usepackage{hyperref}
\usepackage{xcolor}
\usepackage{etoolbox,mathtools}
\usepackage[shortlabels]{enumitem}

% Mdframed template
\mdfsetup{innertopmargin=5pt,%
	frametitlealignment=\raggedright,%
	frametitlefont=\texttt,%
	linewidth=2pt,%
	topline=false,rightline=false, bottomline=false,%
	frametitleaboveskip=\dimexpr-1\ht\strutbox\relax,}

% Problem Env
\newcounter{ProblemCounter}
\newenvironment{Problema}[1][]{%
	\vspace{1em}
	\refstepcounter{ProblemCounter}%
	\noindent\hspace{-12ex}{\texttt{PROBLEMA~\theProblemCounter}~}
	%\begin{tikzpicture}
	%	\draw[thick] (0,0) -- (\linewidth, 0)
	%	node[midway, fill=white] 
	%	{\texttt{PROBLEMA~\theProblemCounter}};
	%\end{tikzpicture}
	\fontfamily{ptm}\selectfont
	}{
	\ \newline
	%\begin{tikzpicture}
	%	\draw[thick] (0,0) -- (\linewidth, 0);
	%\end{tikzpicture}
	%\vspace{1ex}
	}

% Solution Env
\newenvironment{Solucion}[1][]
{%
	\vspace{1ex}
	\noindent\hspace{-10ex}{{\ttfamily SOLUCIÓN}~}
}%
{%
	%\hfill\(\blacksquare\)
}

% -- Aligned Cases
\newenvironment{caligned}
	{\begin{cases} \begin{aligned}}
	{\end{aligned} \end{cases}}

% Resize abs and norm
\DeclarePairedDelimiter{\abs}{\lvert}{\rvert}
\DeclarePairedDelimiter{\norm}{\|}{\|}
\makeatletter
\let\oldabs\abs
\def\abs{\@ifstar{\oldabs}{\oldabs*}}
\let\oldnorm\norm
\def\norm{\@ifstar{\oldnorm}{\oldnorm*}}
\makeatother

% Shortcuts
\def\V{\mathbb{V}}
\def\N{\mathbb{N}}
\def\Z{\mathbb{Z}}
\def\Q{\mathbb{Q}}
\def\R{\mathbb{R}}
\def\C{\mathbb{C}}
\def\CC{\mathcal{C}}
\def\F{\mathbb{F}}
\def\FF{\mathcal{F}}
\def\A{\mathbb{A}}
\def\n{\hat{\textbf{n}}}
\def\loc{\textrm{loc}}
\def\supp{\textrm{supp}\,}

\newcommand{\header}[2]
{
\begin{minipage}{.15\textwidth}
	\includegraphics[width=\textwidth,height=\textheight,keepaspectratio]{LogoUC}
\end{minipage}
\hspace{1em}
\begin{minipage}{.75\textwidth}
	\vspace{2em}
	{\scshape
	Pontificia Universidad Católica de Chile\\
	Facultad de Matemáticas\\
	Docente: Carlos Román\\
	Ayudante: Santiago González
	}
\end{minipage}

\medskip

\begin{center}
	{\textbf{MAT2505 - Ecuaciones Diferenciales Parciales}} \\[1ex]
	{{\large Tarea #1 - Sebastián Sánchez}}
\end{center}

\medskip
}


\begin{document}

\header{2}{}

% Problema 1
\begin{Problema}(Fórmula de Poisson para la bola)
	Sea \(g\in \mathcal{C}(\partial B(0,r))\) y \(u\) definida por
	\begin{displaymath}
		u(x)
		=
		\frac{r^2 - \abs{x}^2}{n \alpha(n) r}
		\int_{\partial B(0,r)}
			\frac{g(y)}{\abs{x-y}^n} \, dS(y)
		\qquad
		x\in B^{\circ}(0,r).
	\end{displaymath}
	Demuestre que
	\begin{enumerate}[leftmargin=.3\textwidth,itemsep=2pt]
		\item \(u\in \mathcal{C}^{\infty}(B(0,r)^{\circ})\),
		\item \(\Delta u = 0\) en \(B(0,r)^{\circ}\),
		\item \(\lim_{x\to x_0} u(x) = g(x_0)\) para todo \(x_0 \in
			\partial B(0,r)\).
	\end{enumerate}
\end{Problema}
\begin{Solucion}
	Denotemos por \(B_{r} \coloneqq B(0,r)\).
	Notando que
	\begin{displaymath}
		\partial_{\n(y)} G(x,y)
		=
		-\frac{r^2 - \abs{x}^2}{n \alpha(n) r}
			\frac{1}{\abs{x-y}^n}
	\end{displaymath}
	podemos reescribir \(u\) como 
	\begin{displaymath}
		u(x) 
		=
		- \int_{\partial B_{r}}
			g(y) \partial_{\n(y)} G(x,y) \, dS(y).
	\end{displaymath}
	Dado que la función de Green es simétrica (sim) y \(\mathcal{C}^{\infty}\), 
	tenemos que (usando TCD para intercambiar límites):
	\begin{align*}
		\Delta_{x} u(x)
		&=
		\Delta_{x}\left(
		- \int_{\partial B_{r}}
			g(y) \partial_{\n(y)} G(x,y), dS(y)
		\right)
		\\&\overset{TCD}{=}
		- \int_{\partial B_{r}}
			g(y) \partial_{\n(y)} \Delta_{x} G(x,y), dS(y)
		\\&\overset{\text{sim}}{=}
		- \int_{\partial B_{r}}
			g(y) \partial_{\n(y)} 
			\underbrace{\Delta_{y} G(x,y)}_{=0}, dS(y)
		= 0.
	\end{align*}
	Es decir, la función \(u\) es armónica y por lo tanto
	\(\mathcal{C}^{\infty}(B_{r}^{\circ})\).

	Falta ver que al acercarse al borde nos acercamos a \(g\).
	Fijemos \(y_0 \in \partial B_{r}\). Queremos ver que
	\begin{displaymath}
		\lim_{B_{r}^{\circ} \ni x \to y_0} \abs{u(x) - g(y_0)} = 0.
	\end{displaymath}

	Sea \(\epsilon > 0\). Como \(g\) es continua, existe \(\delta >
	0\) tal que
	\begin{displaymath}
		\abs{g(y_0) - g(y)} < \epsilon
		\quad \forall y \in 
		\underbrace{
			B(y_0, \delta)
		}_{\eqqcolon B_{\delta}} \cap \partial B.
	\end{displaymath}
	Usando que
	\begin{displaymath}\label{af1}
		\int_{\partial B_{r}} k(x,y) \, dS(y) = 1,
		\tag{\(\star\)}
	\end{displaymath}
	tenemos que 
	\begin{align*}
		\abs{ u(x) - g(y_0) }
		&=
		\abs{
		\int_{\partial B_{r}} 
			k(x,y) ( g(y) - g(y_0) )
			\, dS(y)
		}
		\\&\le
		\abs{
		\int_{\partial B_{r} \cap B_{\delta}} 
			k(x,y) ( g(y) - g(y_0) )
			\, dS(y)
		}
		+ 
		\abs{
		\int_{\partial B_{r} \setminus B_{\delta}} 
			k(x,y) ( g(y) - g(y_0) )
			\, dS(y)
		}
	\end{align*}
	Llamemos a las integrales \(I_1\) e \(I_2\) respectivamente. Por la
	continuidad de \(g\) y dado que el kernel integra uno podemos
	acotar \(I_1\) por \(\epsilon\). En símbolos:
	\begin{displaymath}
		\abs{I_1} \le \epsilon \xrightarrow{\epsilon \to 0} 0.
	\end{displaymath}
	Para acotar \(I_2\), tomemos \(x\in B_{\delta/2}\). Se sigue que
	para \(y \in \partial B_{r} \setminus B_{\delta}\),
	\begin{displaymath}
		\abs{y - x} \ge \frac{1}{2} \abs{y - y_0}.
	\end{displaymath}
	Luego, 
	\begin{align*}
		\abs{I_2} 
		&\le
		\norm{g}_{\infty} 
		\int_{\partial B_{r} \setminus B_{\delta}} k(x,y) \, dS(y)
		\\&=
		\norm{g}_{\infty} 
		\int_{\partial B_{r} \setminus B_{\delta}} 
			\frac{r^2 - \abs{x}^2}{n\alpha(n) r} 
			\frac{1}{\abs{y-x}^n} 
			\, dS(y)
		\\&\le
		\norm{g}_{\infty} 
		\int_{\partial B_{r} \setminus B_{\delta}} 
			\frac{r^2 - \abs{x}^2}{n\alpha(n) r 2^{n}} 
			\frac{1}{\abs{y_0 - y}^n} 
			\, dS(y)
		\\&\le
		\underbrace{
		\norm{g}_{\infty} 
		\frac{1}{n\alpha(n) r 2^{n}} 
		\int_{\partial B_{r} \setminus B_{\delta}} 
			\frac{1}{\abs{y_0 - y}^{n}}
			 \, dS(y)
		}_{\text{constante en \(x\)}}
		\,
		( r^2 - \abs{x}^2 )
	\end{align*}
	Tomando \(x\to y_0\) en particular no da que \(\abs{x} \to r\) y
	por lo tanto \(\abs{I_2} \to 0\). Concluimos entonces que 
	\begin{displaymath}
		\lim_{B_{r}^{\circ} \ni x \to y_0} \abs{u(x) - g(y_0)} = 0.
	\end{displaymath}

	\framebox{Demostración de \(\star\):}
	Sea \(r > 0\) y denotemos por \(B_{r}\) a la bola centrada en el
	origen de radio \(r\).

	Llamemos \(\mathcal{K}\) a la integral del kernel de poisson sobre
	\(\partial B_{r}\). Expadiendo \(\mathcal{K}\) tenemos que
	\begin{align*}
		\int_{\partial B_r} k(x,y) \, dS(y)
		&=
		-\int_{\partial B_r} 
		\partial_{\n(y)} G(x,y) \, dS(y)
		\\&=
		\int_{\partial B_r} 
			\frac{r^2 - \abs{x}^2}{n \alpha(n) r}
			\frac{1}{\abs{x-y}^n}
		\, dS(y)
		\\&=
		\frac{r^2 - \abs{x}^2}{n\alpha(n) r}
		\int_{\partial B_r} 
			\frac{1}{\abs{x-y}^n}
		\, dS(y).
	\end{align*}
	Consideremos ahora \(s < r\). Nótese que si \(x\in \partial
	B_{s}\), entonces el integrando es constante \(\abs{s-r}^{-n}\) y
	por lo tanto \(\mathcal{K} = 1\) en la frontera de \(B_{r}\).
	
	Por otro lado, \(\mathcal{K}\) resuelve el siguiente problema 
	(usando TCD, la simetría y la suavidad de la función de Green), 
	\begin{displaymath}
	\begin{caligned}
		\Delta_{x} \mathcal{K}(x) &= 0
		&&, \text{ para } x\in B_{s}
		\\
		\mathcal{K}(x) &= \alpha
		&&, \text{ para } x\in \partial B_{s}
	\end{caligned}
	\end{displaymath}
	donde \(\alpha\) es una constante.
	Por la unicidad del problema de Poisson se sigue que \(\mathcal{K}
	= \alpha\) en \(\overline{B_{s}}\). Más aún, \(\mathcal{K}(0) =
	1\), así que \(\mathcal{K} = 1\) en \(\overline{B_{s}}\). Basta
	tomar \(s \uparrow r\) y obtenemos el resultado o bien para cada
	\(x\) repetir el procedimiento tomando \(s = \abs{x}\).
\end{Solucion}

% Problema 2
\begin{Problema}
	Sea \(\Omega = \left\{ (x,y) \in \R^2 \mid x^2 + y^2 < a^2
	\right\}\). Denotando por \((r,\theta)\) las coordenadas polares
	en \(\R^2\), consideramos el problema
	\begin{displaymath}
	\begin{aligned}
		u_{xx} + u_{yy} &= 0 &&, (x,y) \in \Omega\\
		u(x,y) &= h(\theta) &&, (x,y) \in \partial\Omega, 
	\end{aligned}
	\end{displaymath}
	donde \(h\) es una función (suave) definida sobre el círculo de
	radio \(a\). La idea de esta pregunta es resolver este problema
	utilizando coordenadas polares. Para ello, proceda de la siguiente
	forma:
	\begin{enumerate}[label=(\alph*), topsep=3pt, itemsep=2pt]
		\item Escriba la ecuación en coordenadas polares. Para ello,
			escriba \(u = u(r,\theta)\), y muestre que 
			\begin{displaymath}
				u_{rr} + \frac{1}{r} u_{r} 
				+ \frac{1}{r^2} u_{\theta\theta} 
				= 0.
			\end{displaymath}
		\item A continuación debe resolver esta ecuación utilizando
			separación de variables. Para ello, busque una solución de
			la forma \(u(r,\theta) = R(r) \Theta(\theta)\), y deduzca
			las ecuaciones que deben satisfacer \(R\) y \(\Theta\).
		\item Note que \(\Theta\) satisface las condiciones de borde
			periódicas para \(r=a\). Usando esto, encuentre las
			soluciones \((\lambda_n, \Theta_n)\) del problema de
			auto-valores y auto-funciones asociado a la función
			\(\Theta\).
		\item Usando los auto-valores \(\lambda_n\) encontrados en la
			parte anterior, resuelva la EDO lineal con coeficientes
			variables satisfecha por \(R = R_{n}\) para cada uno de
			estos auto-valores.
		\item Escriba un candidato a solución de la EDP como una
			combinación lineal infinita (serie), a partir de las
			soluciones de las EDOs resueltas en las partes anteriores,
			\textbf{descartando} aquellas que revientan cuando \(r
			\downarrow 0\) (¿por qué?), y encuentre los coeficientes
			de la serie en términos de \(h\).
		\item Usando que
			\begin{displaymath}
				\cos(n\phi) \cos(n\theta)
				+
				\sin(n\phi) \sin(n\theta)
				=
				\cos(n(\theta - \phi)),
			\end{displaymath}
			expresando \(\cos(n(\theta - \phi))\) en término de
			exponenciales complejas, y la convergencia de la serie
			geométrica, muestre que
			\begin{displaymath}
				u(r,\theta)
				=
				\frac{1}{2\pi}
				\int_{0}^{2\pi}
					h(\phi)
					\frac
					{a^2 - r^2}
					{a^2 - 2 a r \cos(\theta-\phi)+ r^2}
					\, d\phi.
			\end{displaymath}
		\item Finalmente, pasando a coordenadas cartesianas, muestre
			que la solución encontrada satisface la fórmula de Poisson
			para un disco.
	\end{enumerate}
\end{Problema}

\begin{Solucion}
\begin{enumerate}[label=(\alph*), topsep=3pt, itemsep=2pt]
% 2a
\item El cambio desde coordenadas rectangulares a
	coordenadas polares viene dado por:
	\begin{displaymath}
	\begin{aligned}
		T \colon \Omega \setminus \left\{ (0,0) \right\} &\to (0, a) \times [0, 2\pi)\\
		(x,y) &\mapsto 
		\left( \sqrt{x^2 + y^2}, 
		\begin{caligned}
			\arctan(y/x) &&, y \ge 0, x \ne 0\\
			-\arctan(y/x) &&, y \le 0, x\ne 0 \\
			\pi/2 &&, x = 0
		\end{caligned}
		\right)
	\end{aligned}.
	\end{displaymath}
	Por simetría del disco basta considerar el caso del plano superior, esto es 
	\(y > 0\) y \(x \ne 0\). Además, para no andar anotando la \(u\)
	en todas partes trabajaré con los operadores. 

	La derivada de la transformación es
	\begin{displaymath}
		DT(x,y)
		=
		\begin{pmatrix}
			\cos(\theta) & -\frac{1}{r} \sin(\theta) \\
			\sin(\theta) & \frac{1}{r} \cos(\theta)
		\end{pmatrix}.
	\end{displaymath}
	Al aplicar regla de la cadena tenemos
	\begin{displaymath}
		D(u\circ T)
		=
		Du(T) \cdot DT
		=
		\begin{pmatrix}
			\partial_{r} & \partial_{\theta}
		\end{pmatrix}
		\cdot
		\begin{pmatrix}
			\cos(\theta) & \sin(\theta)  \\
			-\frac{1}{r} \sin(\theta) & \frac{1}{r} \cos(\theta)
		\end{pmatrix},
	\end{displaymath}
	y por lo tanto los operadores se leen
	\begin{align*}
		\partial_{x} &= 
		\cos(\theta) \partial_{r} - \frac{1}{r} \sin(\theta) \partial_{\theta}
		\\
		\partial_{y} &=
		\sin(\theta) \partial_{r} + \frac{1}{r} \cos(\theta)
		\partial_{\theta}.
	\end{align*}
	Dado que (aplicando regla del producto)
	\begin{align*}
		\partial_{r} \partial_{x}
		&=
		\cos(\theta) \partial_{r}^2 
		- \frac{1}{r^2} \sin(\theta) \partial_{\theta}
		- \frac{1}{r} \sin(\theta) \partial_{r} \partial_{\theta}
		\\
		\partial_{\theta} \partial_{x}
		&=
		(-\sin(\theta) \partial_{r}) + (\cos(\theta) \partial_{\theta} \partial{r}) 
		- \frac{1}{r} \cos(\theta) \partial_{\theta}
		- \frac{1}{r} \sin(\theta) \partial_{\theta}^2,
	\end{align*}
	tenemos que
	\begin{align*}
		\partial_{x}^2 
		&=
		\cos\theta \left[
		\cos\theta \partial_{r}^2 
		- \frac{1}{r^2} \sin\theta \partial_{\theta}
		- \frac{1}{r} \sin\theta \partial_{r} \partial_{\theta}
		\right]
		\\& - \frac{1}{r} \sin\theta
		\left[
			- \sin(\theta) \partial_{r} 
			+ \cos\theta \partial_{\theta} \partial{r}
			- \frac{1}{r} \cos\theta \partial_{\theta}
			- \frac{1}{r} \sin\theta \partial_{\theta}^2
		\right]
		\\&=
		\cos^2 \theta \partial_{r}^2 
		+ \frac{1}{r} \sin^2\theta \partial_{r} 
		+ \frac{1}{r^2} \sin^2\theta \partial_{\theta}^2.
	\end{align*}
	Donde para simplificar usamos la regularidad de la función (que
	las derivadas parciales cruzadas conmutan). Ahora hacemos lo mismo
	con \(\partial_{y}\).
	\begin{align*}
		\partial_{r} \partial_{y}
		&=
		\sin(\theta) \partial_{r}^2
		-\frac{1}{r^2} \cos\theta \partial_{\theta}
		+ \frac{1}{r} \cos\theta \partial_{r} \partial_{\theta}
		\\
		\partial_{\theta} \partial_{y}
		&=
		\cos\theta \partial_{r}
		+ \sin\theta \partial_{\theta} \partial_{r}
		- \frac{1}{r} \sin\theta \partial_{\theta}
		+ \frac{1}{r} \cos\theta \partial_{\theta}^2,
	\end{align*}
	y obtenemos que 
	\begin{align*}
		\partial_{y}^2 
		&=
		\sin(\theta) \left[
			\sin(\theta) \partial_{r}^2
			-\frac{1}{r^2} \cos\theta \partial_{\theta}
			+ \frac{1}{r} \cos\theta \partial_{r} \partial_{\theta}
		\right]
		\\&+ \frac{1}{r} \cos\theta \left[
			\cos\theta \partial_{r}
			+ \sin\theta \partial_{\theta} \partial_{r}
			- \frac{1}{r} \sin\theta \partial_{\theta}
			+ \frac{1}{r} \cos\theta \partial_{\theta}^2
		\right]
		\\&=
		\sin^2\theta \partial_{r}^2 
		+ \frac{1}{r} \cos^2 \theta \partial_{r}
		+ \frac{1}{r^2} \cos^2\theta \partial_{\theta}^2.
	\end{align*}
	De esta forma, 
	\begin{align*}
		\partial_{x}^2 + \partial_{y}^2
		&=
		\cos^2 \theta \partial_{r}^2 
		+ \frac{1}{r} \sin^2\theta \partial_{r} 
		+ \frac{1}{r^2} \sin^2\theta \partial_{\theta}^2
		\\&+
		\sin^2\theta \partial_{r}^2 
		+ \frac{1}{r} \cos^2 \theta \partial_{r}
		+ \frac{1}{r^2} \cos^2\theta \partial_{\theta}^2
		\\&=
		\partial_{r}^2
		+ \frac{1}{r} \partial_{r}
		+ \frac{1}{r^2} \partial_{\theta}^{2}.
	\end{align*}
	Aplicando el operador a \(u\) y usando la hipótesis tenemos lo
	pedido.
% 2b
\item Buscamos soluciones de la forma \(u(r,\theta) = R(r)
	\Theta(\theta)\). Poniéndolas en la ecuación tenemos que
	\begin{displaymath}
		\Theta(\theta) \partial_{r}^2 R(r)
		+ \Theta(\theta) \frac{1}{r} \partial_{r} R(r)
		+ R(r) \frac{1}{r^2} \partial_{\theta}^{2} \Theta(\theta)
		= 0.
	\end{displaymath}
	Equivalentemente (suponiendo que \(R, \Theta \not\equiv 0\)),
	\begin{displaymath}
		\frac
			{r^2 R''(r)}
			{R(r)}
		+
		\frac
			{r R'(r)}
			{R(r)}
		=
		-
		\frac
			{\Theta''(\theta)}
			{\Theta(\theta)}.
	\end{displaymath}
	Como ambos lados dependen de variables distintas, es necesario que
	igualen una constante. Denotemos tal constante por \(-\lambda\).
	Luego, \(R\) y \(\Theta\) deben resolver,
	\begin{align*}	
		0 &= r^2 R''(r) + r R'(r) + \lambda R(r) \\
		0 &= \Theta''(\theta) - \lambda \Theta(\theta).
	\end{align*}
% 2c
\item 
	Dado que 
	\begin{displaymath}
		R(a) \Theta(\theta) = h(\theta).
	\end{displaymath}
	tenemos que problema es periódico en el siguiente sentido,
	\begin{subequations}
	\begin{align}
		\label{f}
		0 &= \Theta''(\theta) - \lambda \Theta(\theta)
		\\
		\label{f1}
		0 &= \Theta(0) - \Theta(2\pi)
		\\
		\label{f2}
		0 &= \Theta'(0) - \Theta'(2\pi)
	\end{align}.
	\end{subequations}

	La ecuación característica de la EDO~\eqref{f} es 
	\begin{displaymath}
		0 = z^2 - \lambda.
	\end{displaymath}

	\underline{Si \(\lambda = 0\):} entonces \(\Theta(\theta) = m
	\theta + n\). Por las condiciones de borde periódicas,
	\begin{displaymath}
		\eqref{f1}
		\implies
		\Theta(0) = n = 2\pi m + n = \Theta(2\pi)
		\iff m = 0
	\end{displaymath}
	Así que la solución es constante. 

	\underline{Si \(\lambda > 0\):} entonces \(0 = (z -
	\sqrt{\lambda})(z + \sqrt{\lambda})\). Así, las soluciones son de
	la forma
	\begin{displaymath}
		\Theta(\theta)
		= 
		A \exp(\sqrt{\lambda} \theta)) + 
		B \exp(-\sqrt{\lambda} \theta).
	\end{displaymath}
	Por las condiciones de periocidad, 
	\begin{align*}
		\eqref{f1} &\implies
		\Theta(0) = A + B 
		= 
		A\, e^{\sqrt{\lambda} 2\pi} +
		B\, e^{-\sqrt{\lambda} 2\pi}
		=
		\Theta(2\pi)
		\\
		\eqref{f2} &\implies
		\Theta'(0) = A\sqrt{\lambda} - B\sqrt{\lambda} 
		= 
		A\sqrt{\lambda}\, e^{\sqrt{\lambda} 2\pi} -
		B\sqrt{\lambda}\, e^{-\sqrt{\lambda} 2\pi}
		=
		\Theta'(2\pi).
	\end{align*}
	Como \(\lambda > 0\) podemos dividir la segunda ecuación por
	\(\lambda\). Sumando ambas nos deja que,
	\begin{displaymath}
		2A = 2A e^{\sqrt{\lambda} 2\pi}.
	\end{displaymath}
	Si \(A \ne 0\), entonces \(\lambda = 0\). Por lo tanto, \(A = 0\).
	De manera similar concluimos que \(B = 0\). Así, la solución es la
	nula y por lo tanto no hay valores propios.

	\underline{Si \(\lambda < 0\):} La ecuación característica queda
	\begin{displaymath}
		0 = (z - i\sqrt{-\lambda}) (z + i\sqrt{-\lambda})
	\end{displaymath}
	Luego, 
	\begin{displaymath}
		\Theta(\theta) 
		=
		A e^{i\sqrt{-\lambda} \theta} 
		+
		B e^{-i\sqrt{-\lambda} \theta}
		=
		(A+B) \cos(\sqrt{-\lambda} \theta) 
		+
		(A-B) i \sin(\sqrt{-\lambda} \theta).
	\end{displaymath}
	Imponiendo las condiciones de periocidad tenemos que,
	\begin{alignat*}{3}
		\eqref{f1} &\implies
		\Theta(0) 
		= A + B 
		= 
		(A+B) \cos(\sqrt{-\lambda} 2\pi) 
		+ (A-B) i \sin(\sqrt{-\lambda} 2\pi)
		=
		\Theta(2\pi)
		\\
		\eqref{f2} &\implies
		\Theta'(0) 
		= (A-B) i \sqrt{-\lambda} \sin(\sqrt{-\lambda} \theta)
		\\&= 
		-(A+B) \sqrt{-\lambda} \sin(\sqrt{-\lambda} 2\pi) 
		+ (A-B) i \sqrt{-\lambda} \cos(\sqrt{-\lambda} 2\pi)
		=
		\Theta'(2\pi).
	\end{alignat*}
	En la última ecuación podemos dividir por \(\sqrt{-\lambda}\) y
	multiplicar por \(i\). Sumando ambas obtenemos (ignorando el
	argumento de las funciones para ahorrar notación),
	\begin{align*}
		2B = 2B \cos - 2B i\sin.
	\end{align*}
	Si \(B \ne 0\), entonces \(1 = \cos - i \sin\) lo cual (comparando
	parte imaginaria y real) solo puede pasar si \(\sqrt{-\lambda}
	2\pi = 2\pi k\) con \(k \ge 1\); Vale decir, los valores propios
	son:
	\begin{displaymath}
		\lambda_{n} = -n^2.
	\end{displaymath}
	Se sigue que las autofunciones vienen dadas por
	\begin{displaymath}
		\Theta_{n}(\theta) = A_n e^{\theta n i} + B_n e^{-\theta n i}.
	\end{displaymath}

	\item
	Volviendo a la ecuación de \(R\) tenemos que para cada
	\(\lambda_n\) se debe satisfacer:
	\begin{displaymath}
		0 = r^2 R_n''(r) + r R_n'(r) - n^2 R_n(r).
	\end{displaymath}
	Tomemos como ansatz \(R_n = r^N\) para algún \(N\). Luego,
	\begin{align*}
		0 
		&= r^2 N(N-1) r^{N-2} + r N r^{N-1} - n^2 r^{N} 
		\\&= N(N-1) r^{N} + N r^{N} - n^2 r^{N} 
		\\&= r^N (N^2 - n^2)
		\\&= r^{N} (N-n)(N+n).
	\end{align*}
	y las soluciones son de la forma
	\begin{displaymath}
		R_n = A r^{n} + B r^{-n}.
	\end{displaymath}
%2e
	\item 
	Recordamos que las soluciones desacopladas son de la forma
	\begin{align*}
		R_n &= a_n r^n + b_n r^{-n}
		\intertext{y}
		\Theta_{n}(\theta) &= c_n e^{\theta n i} + d_n e^{-\theta n
		i},
	\end{align*}
	para cada \(n \ge 0\) (recordar que las constantes también son
	solución).
	
	Queremos aplicar el Ppio de superposición con las soluciones
	\(u_{n} = R_n \Theta_{n}\). El problema es que la primera relación
	sobre \(R_n\) no está
	completamente definida en todo nuestro dominio. Sin embargo, por
	el mismo Ppio podemos imponer que \(b_n = 0\) y la relación
	seguirá resolviendo la ecuación.
	El candidato a solución entonces es de la forma
	\begin{displaymath}
		u(r, \theta) 
		= \sum_{n\ge 0} 
			r^{n} (x_n\, e^{\theta n i} + y_n\, e^{-\theta n i}).
	\end{displaymath}
	De ser solución, por las condiciones de borde se tiene que
	\begin{displaymath}
		u(a, \theta) 
		= \sum_{n\ge 0} 
			a^{n} (x_n\, e^{\theta n i} + y_n\, e^{-\theta n i})
		= h(\theta).
	\end{displaymath}
	Usando la ortogonalidad de las autofunciones tenemos que al
	aplicar \(\langle \cdot, \pm e^{\theta n i} \rangle\)
	nos deja
	\begin{displaymath}
		x_n = 
		\frac{1}{a^n}
		\frac
			{\langle h(\theta), e^{in\theta} \rangle}
			{\langle e^{in\theta}, e^{in\theta} \rangle}
		=
		\frac{\langle h(\theta), e^{in\theta} \rangle}{2\pi\, a^n}
		\hspace{.5cm} \text{y} \hspace{.5cm}
		y_n = 
		\frac{1}{a^n}
		\frac
			{\langle h(\theta), e^{-in\theta} \rangle}
			{\langle e^{-in\theta}, e^{-in\theta} \rangle}
		=
		\frac{\langle h(\theta), e^{-in\theta} \rangle}{2\pi\, a^n}
		.
	\end{displaymath}
	Denotemos por \(h_n(\theta) = \langle h(\theta), e^{in\theta}
	\rangle\) y \(\tilde{h}_{n}(\theta) = \langle h(\theta),
	e^{-in\theta)} \rangle\).
	De esta forma, el candidato a solución se ve
	\begin{displaymath}
		u(r, \theta) 
		= 
		\sum_{n\ge 0} 
			\frac{r^{n}}{2\pi\, a^n} 
			(h_n e^{\theta n i} + \tilde{h}_n e^{-\theta n i})
		.
	\end{displaymath}
%2f
	\item
	Expandiendo el término anterior nos queda
	\begin{align*}
		u(r, \theta) 
		&= 
		\sum_{n\ge 0} 
			\frac{r^{n}}{2\pi\, a^n} 
			(h_n e^{\theta n i} + \tilde{h}_n e^{-\theta n i})
		\\&=
		\sum_{n\ge 0} 
			\frac{r^{n}}{2\pi\, a^n} 
			\left[
			\int_{0}^{2\pi}
				h(\phi)
				\left(
				e^{in(\theta - \phi)}
				+
				e^{-in(\theta - \phi)}
				\right)
			\, d\phi
			\right].
	\end{align*}
	Intercambiando la sumatoria con la integral donde se puede nos
	deja
	\begin{displaymath}
		u(r, \theta) 
		=
		\frac{1}{\pi}
		\int_{0}^{2\pi}
		h(\phi)
		\left[
		\frac{1}{2} +
		\sum_{n\ge 1} 
		\frac{r^{n}}{2 a^n} 
		\left( e^{in(\theta - \phi)} + e^{-in(\theta - \phi)} \right)
		\right]
		\, d\phi.
	\end{displaymath}
	Dado que,
	\begin{displaymath}
		\sum_{n\ge 1} 
		\frac{r^{n}}{a^n} 
		\left( e^{in(\theta - \phi)} + e^{-in(\theta - \phi)} \right)
		=
		\sum_{n\ge 1} 
			\frac{r^{n}}{a^n} 
			e^{in(\theta - \phi)}
		+
		\sum_{n\ge 1} 
			\frac{r^{n}}{a^n} 
			e^{-in(\theta - \phi)},
	\end{displaymath}
	con las últimas series geométricas convergentes (el módulo del
	argumento de la serie es menor a \(1\)). Se sigue que
	\begin{displaymath}
		\sum_{n\ge 1}
			\frac{r^n}{a^n}
			e^{in(\theta - \phi)}
		=
		\frac
			{(r/a) e^{i(\theta-\phi)}}
			{1 - \frac{r}{a} e^{i(\theta-\phi)}}
		\hspace{.5cm} \text{y} \hspace{.5cm}
		\sum_{n\ge 1}
			\frac{r^n}{a^n}
			e^{-in(\theta - \phi)}
		=
		\frac
			{(r/a) e^{-i(\theta-\phi)}}
			{1 - \frac{r}{a} e^{-i(\theta-\phi)}}.
	\end{displaymath}
	Al volverlas a sumar nos queda,
	\begin{align*}
		\sum_{n\ge 1} 
		\frac{r^{n}}{2 a^n} 
		\left( e^{in(\theta - \phi)} + e^{-in(\theta - \phi)} \right)
		&=
		\frac{1}{2}
		\frac
		{(r/a) e^{i(\theta-\phi)} - (r/a)^2 + (r/a)
		e^{-i(\theta-\phi)} - (r/a)^2}
		{1 - (r/a) e^{i(\theta-\phi)} - (r/a) e^{-i(\theta-\phi)} + (r/a)^2}
		\\&=
		\frac{1}{2}
		\frac
		{ar e^{-i(\theta-\phi)} - ar e^{i(\theta-\phi)} - 2r^2}
		{a^2 - ar e^{-i(\theta-\phi)} - ar e^{i(\theta-\phi)} + r^2}
		\\&=
		\frac
		{ar\cos(\theta-\phi) - r^2}
		{a^2 - 2ar \cos(\theta- \phi) + r^2}.
	\end{align*}
	De esta forma, el integrando queda
	\begin{displaymath}
		\frac{1}{2} +
		\frac
		{ar\cos(\theta-\phi) - r^2}
		{a^2 - 2ar \cos(\theta- \phi) + r^2}
		=
		\frac
		{a^2 - r^2}
		{a^2 - 2ar \cos(\theta- \phi) + r^2}.
	\end{displaymath}
	Juntando todo nos da que
	\begin{displaymath}
		u(r,\theta)
		=
		\frac{1}{2\pi}
		\int_{0}^{2\pi}
			h(\phi)
			\frac
			{a^2 - r^2}
			{a^2 - 2 a r \cos(\theta-\phi)+ r^2}
		\, d\phi.
	\end{displaymath}
% 2g
	\item
	Haciendo el cambio a coordenadas cartesianas sobre los parámetros
	de la función
	\begin{displaymath}
		x = r\cos\theta \hspace{1cm} y = r\sin\theta,
	\end{displaymath}
	nos deja que
	\begin{displaymath}
		u(x,y)
		=
		\frac{1}{2\pi}
		\int_{0}^{2\pi}
			h(\phi)
			\frac
			{a^2 - x^2 - y^2}
			{a^2 - 2 a (x\cos\phi + y\cos\phi)+ x^2 + y^2}
		\, d\phi.
	\end{displaymath}
	Por la simetría basta derivar con respecto a una variable. Dígamos
	\(x\). Por TCD podemos intercambiar la integral con la derivada y
	solo enforcanos en la fracción que tiene \(x\).
	\begin{displaymath}
		\begin{array}{*1{>{\displaystyle}c}p{5cm}}
		\partial_{x}
		\left(
		\frac
			{a^2 - x^2 - y^2}
			{a^2 - 2 a (x\cos\phi + y\cos\phi)+ x^2 + y^2}
		\right)
		\\\rotatebox{90}{=}\\
		\frac
			{(-2x)\,(a^2 - 2a(x\cos\phi + y\cos\phi) + x^2 + y^2) +
			(a^2 - x^2 - y^2)\,(-2a\cos\phi + 2x)}
			{(a^2 - 2 a (x\cos\phi + y\cos\phi)+ x^2 + y^2)^2}
		\\\rotatebox{90}{=}\\
		\frac
			{2a\cos\phi(2x^3 + 2xy - a^2 + x^2 + y^2) - 4x(x^2+y^2)}
			{(a^2 - 2 a (x\cos\phi + y\cos\phi)+ x^2 + y^2)^2}.
	\end{array}
	\end{displaymath}
	(\textit{Respiro profundo}) Derivando de nuevo
	\begin{displaymath}
		\begin{array}{*1{>{\displaystyle}c}p{5cm}}
		\partial_{x}
		\left(
		\frac
			{2a\cos\phi(2x^3 + 2xy - a^2 + x^2 + y^2) - 4x(x^2+y^2)}
			{(a^2 - 2 a (x\cos\phi + y\cos\phi)+ x^2 + y^2)^4}
		\right)
		\\\rotatebox{90}{=}\\
		\frac
			{(2a\cos\phi(6x^2 + 2y + 2x)-4(3x^2+y^2))(\downarrow) 
			+ 2(\downarrow)(-2a\cos\phi + 2x) }
			{(a^2 - 2 a (x\cos\phi + y\cos\phi)+ x^2 + y^2)^4}.
	\end{array}
	\end{displaymath}
\end{enumerate}
\end{Solucion}

% Problema 3
\begin{Problema}
	Use la transformada de Fourier para resolver
	\begin{subequations}
	\begin{alignat}{2}
		\label{eq}
		u_{t} - \kappa u_{xx} &= -u &&, (t,x) \in \R^{+}\times \R
		\\ 
		\label{ic}
		u(0,x) &= \phi(x) &&, x\in \R
	\end{alignat}
	\end{subequations}
\end{Problema}
\begin{Solucion}
	Aplicamos la tranformada de Fourier \(\FF\) (en el espacio) a la
	ecuación~\eqref{eq}. 
	Esto nos deja,
	\begin{displaymath}
		\FF(u_t - \kappa u_{xx})
		=
		\FF(u_t - \kappa D^{(0,2)} u)
		=
		\widehat{u_t} - \kappa (ix)^2 \widehat{u}
		=
		-\widehat{u}
		= \FF(-u).
	\end{displaymath}
	Reordenando (notar que intercambiamos los operadores \(\FF\) y
	\(\partial_{t}\))
	\begin{displaymath}
		\widehat{u}_t = - (1 + \kappa x^2) \widehat{u}.
	\end{displaymath}
	Esto es una EDO en tiempo. La solución general es
	\(
		\widehat{u}
		=
		A e^{-t(1+\kappa x^2)}.
	\)
	Aplicando \(\FF\) a las condiciones iniciales~\eqref{ic} nos da
	que
	\(
		\widehat{u} = \widehat{\phi}
	\)
	para \(t=0\). Por lo tanto, la solución es
	\begin{displaymath}
		\widehat{u} = \widehat{\phi} e^{-t(1+\kappa x^2)}.
	\end{displaymath}
	Aplicando la transformada de Fourier inversa tenemos que
	\begin{displaymath}
		u = 
		\frac{\phi\ast \FF^{-1}(e^{-t(1+\kappa x^2)})}{\sqrt{2\pi}}.
	\end{displaymath}
	Por lo que solo basta calcular \(\FF^{-1}(e^{-t(1+\kappa x^2)})\).
	Vamos a ello,
	\begin{align*}
		\FF^{-1}(e^{-t(1+\kappa x^2)})(x)
		&=
		\frac{1}{\sqrt{2\pi}}
		\int_{\R} e^{i xy} e^{-t(1+\kappa y^2)} \, dy
		\\&=
		\frac{1}{\sqrt{2\pi}^n e^{t}}
		\int_{\R} e^{i xy - t\kappa y^2} \, dy
		\\&=
		\frac{1}{\sqrt{2\pi} e^{t}}
		\frac{\sqrt{\pi}}{\sqrt{t\kappa}} 
		\, e^{-\frac{x^2}{4t\kappa}}
		\\&=
		\exp\left(-\frac{x^2}{4t\kappa}\right)\,
		\frac{1}{e^{t} \sqrt{2t\kappa}}.
	\end{align*}
	De esta forma, la solución se lee
	\begin{align*}
		u(t,x)
		&=
		\frac{1}{\sqrt{2\pi}}
		\phi\ast\FF^{-1}(e^{-t(1+\kappa x^2)})
		\\&=
		\frac{1}{\sqrt{4\pi t\kappa} e^{t}}
		\int_{\R}
			e^{-\frac{(x-y)^2}{4t\kappa}}
			\phi(y)
		\, dy.
	\end{align*}
\end{Solucion}

% Problema 4
\begin{Problema}
	Sea \(\Omega \subset \R^d\) un abierto acotado con frontera
	\(\mathcal{C}^{1}\). Demuestre que el problema
	\begin{displaymath}
	\begin{caligned}
		u_{t} - \Delta_{x} u + u &= 0 &&, (t,x) \in \R^{+}\times \Omega\\
		u &= 0 &&, (t,x) \in \left\{ 0 \right\}\times \Omega \\
		\partial_{\n} u + a u &= 0 &&, (t,x) \in \R^{+}\times \partial\Omega\\
	\end{caligned},
	\end{displaymath}
	donde \(a \ge 0\) es una constante, posee una única solución.

	\textbf{Sugerencia:} Considere el funcional de energía 
	\begin{displaymath}
		E(t) = \int_{\Omega} u^2(t,x) \, dx,
	\end{displaymath}
	y estudie el comportamiento de su derivada.
\end{Problema}
\begin{Solucion}
	Supongamos que \(u_1\) y \(u_2\) son dos soluciones. Denotemos por
	\(u = u_1 - u_2\). Luego, \(u\) satisface el problema
	\begin{displaymath}
	\begin{caligned}
		u_{t} - \Delta_{x} u + u &= 0 &&, (t,x) \in \R^{+}\times \Omega\\
		u &= 0 &&, (t,x) \in \left\{ 0 \right\}\times \Omega \\
		\partial_{\n} u + a u &= 0 &&, (t,x) \in \R^{+}\times \partial\Omega\\
	\end{caligned}.
	\end{displaymath}
	Consideremos el funcional de energía dado por la sugerencia. Por
	TCD tenemos que
	\begin{align*}
		\frac{1}{2} E'(t)
		&=	
		\int_{\Omega} u\, u_{t} \, dx
		\\&=
		\int_{\Omega} u\, (\Delta_{x} u - u) \, dx
		\\&=
		\int_{\Omega} u\, \Delta_{x} u - u^2 \, dx
		\\&=
		-\int_{\Omega} \abs{\nabla u}^2 \, dx
		+\int_{\partial \Omega} u \partial_{\n} u\, dx
		-\int_{\Omega} u^2\, dx
		\\&=
		-\int_{\Omega} \abs{\nabla u}^2 \, dx
		-\int_{\partial \Omega} a u^2\, dS(x)
		-\int_{\Omega} u^2\, dx.
	\end{align*}
	Notando que todos los integrandos son no negativos, concluimos que
	\begin{displaymath}
		E'(t) \le 0,\qquad \forall t \ge 0.
	\end{displaymath}
	Dado que \(u\) se anula en \(t = 0\), tenemos que
	\begin{displaymath}
		0 \le E(t) = E(0) = 0,\qquad \forall t \ge 0
	\end{displaymath}
	Así que la solución \(u = u_1 - u_2 = 0\). Con esto concluimos que
	el problema tiene a lo más una solución.
\end{Solucion}

\end{document}
